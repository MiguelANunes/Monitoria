%% This document created by Scientific Word (R) Version 3.0



\documentclass[12pt, a4paper,portuges]{article}
\usepackage{t1enc}
\usepackage[latin1]{inputenc}
\usepackage[portuges]{babel}
\usepackage{amsmath}
\usepackage{amsfonts}
\usepackage{amssymb}

%\usepackage{graphicx}
\topmargin       0cm
 \headheight      17pt
 \headsep  1cm

\textheight      23cm
\textwidth       16.3cm
\oddsidemargin   2mm
\evensidemargin  2mm
\pagestyle{empty}

\begin{document}
\begin{center}
\framebox[\textwidth][c]{$1a.$ Avalia��o de L�gica Matem�tica  (LMA) -
Joinville, \today}

%%\newline
\end{center}

\vskip1cm Aluno(a): \hrulefill

%%%\noindent

\begin{enumerate}
%%%\setlength{\itemsep}{-5pt}
 \item Fa�a a interpreta��o do valor l�gico de:

 \begin{enumerate}
\setlength{\itemsep}{-5pt}
 \item $3+4=8$ se somente se $5^3=125 $;
 
 \item se $3+4 > 8$ ent�o $5^3 < 125 $;
 
 \item A agua pura � cristalina ou o sol do sistema solar � azul;
 
 \item $3^2 + 4^2 = 5^2$ se somente se $\pi $ �  n�o  for irracional;

\item $5^2=25$ ou  $\pi $ � irracional.
\end{enumerate}


\item Construindo a Tabela Verdade,
identifique se a f�rmula � tautol\'{o}gica,
contingente (satisfat\'{i}vel,
consistente), ou inv\'{a}lida
(contradit�ria, insatisfat�vel):

\begin{enumerate}
\setlength{\itemsep}{-5pt}

\item $(A \leftrightarrow B) \vee  (A  \wedge B)$

\item $(A \vee B)\rightarrow (A \wedge  B)$

%%\item $(B \rightarrow A)\rightarrow (A \rightarrow  B)$

%\item $(A \rightarrow (A \rightarrow B)) \rightarrow  B $

\item $(\sim A \rightarrow  \sim B) \rightarrow (\sim B \rightarrow \sim  A)$
\item $(B \rightarrow  A) \rightarrow  (A \rightarrow B) $
\end{enumerate}

\item Escolha duas f�rmulas abaixo,
e verifique  se apresentam uma rela�\~ao de implica�\~ao l�gica  verdadeira:
\begin{enumerate}
\setlength{\itemsep}{-5pt}

\item $q \Rightarrow p \wedge q
\leftrightarrow q$

%\item $ (x=y \vee x < 4) \wedge x \geq 4
%\Rightarrow x=y $
\item  $ (p \vee q) \wedge \sim q \Rightarrow p $


% \item $(x \neq 0 \rightarrow x=y) \wedge x
% \neq y \Rightarrow x=0 $

\item $(p \rightarrow q) \wedge \sim q \Rightarrow \sim p$


\end{enumerate}

\item Demonstre se as f�rmulas abaixo apresentam uma 
rela�\~ao de  equival�ncia l�gicas verdadeira:
\begin{enumerate}
\setlength{\itemsep}{-5pt}

\item $A \leftrightarrow B \Leftrightarrow
(\sim A \wedge \sim B)\vee( A \wedge B)$

\item $(A \rightarrow (A \rightarrow (A
\rightarrow B ))) \Leftrightarrow A
\rightarrow B $


\end{enumerate}

\item Encontre as seguintes Formas Normais Disjuntiva e 
Conjuntiva para:
\begin{enumerate}
\setlength{\itemsep}{-5pt}
 \item $(p \rightarrow q) \wedge \sim(q \rightarrow p) $
 \item $\sim (\sim p \rightarrow q) \vee (q \rightarrow \sim p) $
\end{enumerate}

\item (cap. 11) Verificar a validade dos
argumentos que se seguem:
\begin{enumerate}
\setlength{\itemsep}{-2pt}
 \item $\{ p\rightarrow \sim q$, $\sim p \rightarrow (r \rightarrow \sim q)$,
 $(\sim s \vee \sim r)\rightarrow \sim \sim q$,
 $\sim s$ \} {\bf $\vdash $} $\sim r$

\item $\{(\sim p\vee q) \rightarrow r$,  $(r \vee s)
\rightarrow \sim t$, $t$ \} 
{\bf $\vdash $} $\sim q$

\item  $\{ p\rightarrow \sim q$, 
 $\sim q \rightarrow \sim s$, 
 $(p \rightarrow \sim s) \rightarrow \sim t$, 
 $r \rightarrow t$ \} 
 {\bf $\vdash $} $\sim r$

\end{enumerate}
O que voce conclui sobre os argumentos dos itens a) e c)?

\end{enumerate}


%\vskip1cm

%\noindent

%

\end{document}
