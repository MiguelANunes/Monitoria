% Document Type: latex
% Master File: sheet1_ND_Prop_FOL.tex
\documentclass[a4paper]{article}

\usepackage[textwidth=13cm]{geometry}

\usepackage{amssymb,latexsym,alltt,eepic,
%showkeys,showtags 
%stmaryrd,
amsmath,amsthm
}

%% Gráficos portables
%\usepackage{tikz}
%\usetikzlibrary{arrows}

%\usepackage{wasysym}
%\usepackage{cmll}
%\usepackage{bussproofs}
%\EnableBpAbbreviations

\pagestyle{myheadings} 
\newcommand{\tstamp}{\mbox{\tt Time-stamp: <2009-05-06 02:10:34
    jmarino localhost.localdomain natprop.tex>}} 
%\newcommand{\tensor}{\otimes}
%\newcommand{\lolli}{\multimap}

%\setlength{\oddsidemargin}{0in}
%\setlength{\evensidemargin}{0in}
%\setlength{\topmargin}{0in}
%\addtolength{\textwidth}{1.2in}
%\addtolength{\textheight}{0.5in}
%

\usepackage[utf8]{inputenc}
\usepackage{fourier}
\renewcommand{\ttdefault}{txtt}

\theoremstyle{definition}
\newtheorem{prob}{Problem}

\newcommand{\imp}{\rightarrow}
\newcommand{\ang}[1]{\langle {#1} \rangle}   %< thing >
\newcommand{\mea}{\llbracket}
\newcommand{\ning}{\rrbracket}
\newcommand{\me}[1]{\mea{#1}\ning}
\newcommand{\NK}{\ensuremath{\mathrm{NK}}}
\newcommand{\NJ}{\ensuremath{\mathrm{NJ}}}
%\newcommand{\land}{\wedge}
%\newcommand{\lor}{\vee}
\newcommand{\then}{\rightarrow}
\newcommand{\ded}{\vdash}
\newcommand{\fl}{\rightarrow}

\begin{document}
\title{Exercises on Natural Deduction and Semantics of Classical Propositional 
       and First Order Logic}
\author{Nonclassical Logics\\
Universidad Politécnica de Madrid}

\maketitle

\thispagestyle{empty}

%
%{\footnotesize
%\tableofcontents
%}
\bibliographystyle{plain}

\noindent
\begin{em}
Solutions must be sent by mail to \texttt{jmarino@fi.upm.es} no later
than April 12, 2010. 
\end{em}

\subsection*{Classical propositional logic}

\begin{prob}
Which of the following are tautologies? Prove the tautologies in the
following list by natural deduction. To prove something of the form
$\alpha \leftrightarrow \beta$ you must give a proof of 
$\alpha \rightarrow \beta$ and another proof of 
$\beta \rightarrow \alpha$.

\begin{enumerate}
\item $(\alpha \vee \alpha) \rightarrow \alpha$.
\item $((\alpha \vee \beta) \vee \gamma) \leftrightarrow 
       (\alpha \vee (\beta \vee \gamma))$.
\item $(\alpha \rightarrow \beta) \leftrightarrow 
       \neg(\neg\alpha \wedge \beta)$.
\item $(\alpha \wedge (\alpha \vee \beta)) \leftrightarrow \alpha$.
\item $\neg(\alpha \vee \beta) \leftrightarrow (\neg\alpha \wedge \neg\beta)$.
\item $\alpha \rightarrow (\beta \rightarrow \gamma) \rightarrow 
       ((\alpha \rightarrow \beta) \rightarrow (\alpha \rightarrow \gamma))$.
\item $(\alpha \rightarrow \beta) \rightarrow 
       ((\alpha \rightarrow \neg\beta) \rightarrow \neg\alpha)$.
\item $\alpha \rightarrow \neg\neg\alpha$.
\item $\neg\neg\alpha \rightarrow \alpha$.
\item $(\alpha \rightarrow \beta) \vee (\beta \rightarrow \alpha)$.
\item $((\alpha \rightarrow \beta) \rightarrow \alpha) \rightarrow
       \alpha$.
\item $(\alpha \rightarrow \beta) \leftrightarrow 
       (\neg\alpha \vee \beta)$.
\item $\alpha \vee \neg\alpha$.
\end{enumerate}
\end{prob}


\begin{prob}\label{prob:haskell-prop}
Define the syntax of propositional logic as a set of datatype
definitions in Haskell. Then, define the semantics of propositional
formulas as a Haskell function from formulas and valuations to truth
values. 
Finally, use that function to check the results of the previous problem.
\end{prob}


\begin{prob}
Prove the \emph{soundness theorem} for classical propositional logic,
i.e., show that if there is a natural deduction proof of $\phi$ from
premises $\Gamma$ (written $\Gamma \vdash_{\text{NK}} \phi$), then
every valuation making $\Gamma$ true also makes $\phi$ true 
(written $\Gamma \models \phi$).
\end{prob}


\begin{prob}
Prove the converse of the previous result, i.e.~\emph{completeness}
for classical propositional logic. 
\end{prob}



\subsection*{Classical first order logic}

\begin{prob}Translate the following syllogisms into (classical) first-order
predicate logic and study their validity:

\begin{enumerate}

\item \begin{tabular}[t]{l} 
     All soldiers are strong;\\
     All soldiers are brave.\\
     $\vdash$ Some strong men are brave.
     \end{tabular}

\item \begin{tabular}[t]{l}
     All soldiers can walk;\\
     Some children are not soldiers.\\
     $\vdash$ Some children cannot walk.
     \end{tabular}

\item \begin{tabular}[t]{l}
     I admire these paintings;\\
     When I admire something, I like to scrutinize it thoroughly.\\
     $\vdash$ I like to scrutinize some of these paintings thoroughly.
     \end{tabular}

\item \begin{tabular}[t]{l}
     All soldiers are brave;\\
     Some spaniards are brave.\\
     $\vdash$ Some spaniards are soldiers.
     \end{tabular}

\item \begin{tabular}[t]{l}
     All bullfighters are brave;\\
     Some spaniards are bullfighters.\\
     $\vdash$ Some spaniards are brave.
     \end{tabular}

\item \begin{tabular}[t]{l}
     Every altruistic person is also generous;\\
     No greedy person is generous.\\
     $\vdash$ No greedy person is altruistic.
     \end{tabular}

\end{enumerate}
\end{prob}


\begin{prob}
Prove the following using natural deduction:
\begin{enumerate}
\item $\forall x(\phi(x) \rightarrow \phi(x))$.
\item $\forall x\phi(x) \rightarrow \exists x\phi(x)$.
\item $\forall x(\phi(x) \wedge \psi(x)) \rightarrow 
       (\forall x\phi(x)) \wedge (\forall x \psi(x))$.
\item $\forall x(\phi \rightarrow \psi(x)) \rightarrow 
       (\phi \rightarrow \forall x \psi(x)),~
       \text{where $x$ is not free in $\phi$}$.
\end{enumerate}
\end{prob}

\begin{prob}
Prove the substitution lemma for first-order logic, stated in the
following way. Let $L = \ang{F, R}$ be a first-order language, and 
${\cal A}=\ang{A, {\cal F}^{\cal A}, {\cal R}^{\cal A}}$ be a
first-order structure over $L$.
Let $\eta:{\sf Vars}\imp A$ be an environment for ${\cal A}$,
and $t$ be a term over the language $L$, possibly with
free variables.

\newcommand{\semeta}[1]{\me{#1}_{\eta}^{\cal A}}

%% \clearpage
%% \noindent
Recall that we denote by $\semeta{t}$ the \emph{interpretation} of
$t$ in ${\cal A}$ under the environment $\eta$, defined as follows:

\begin{itemize}
\item $\semeta{x} = \eta(x)$
\item $\semeta{f(t_1,\ldots,t_n)} = 
      f^{\cal A}(\semeta{t_1},\ldots,\semeta{t_n})$
\end{itemize}

\noindent
Recall also, the definition of the truth value 
$\semeta{\varphi}$ of a formula, by induction on the structure of
$\varphi$ (see e.g.~your class notes).

\begin{enumerate}

\item First prove the substitution lemma for {\em terms} $t$, that
      is to say, that 

      \[ \semeta{t[u/x]} = \me{t}_{\eta[x:=\semeta{u}]}. \] 

      This says that the interpretation of a term $t$ in a structure
      ${\cal A}$ under environment $\eta$ is the same if we substitute
      a term $u$ for every free occurrence of a variable $x$ in $t$,
      or if we alter the environment to assign the denotation
      $\semeta{u}$ to $x$.

\item Now prove the same lemma for {\em formulas} $\varphi$,
      that is to say

      \[\semeta{\varphi[u/x]} = \me{\varphi}_{\eta[x:=\semeta{u}]},\] 

      or, equivalently

      \[ {\cal A} \models_{\eta} \varphi[u/x] \mbox{ iff }
         {\cal A} \models_{\eta[x:=\semeta{u}]} \varphi \]

      where $\varphi$ is a formula and $u$ is a term. You must do this by
      induction on the structure of the formula $\varphi$. The base case,
      i.e.~the case where $\varphi$ is an atomic formula
      $r(t_1,\ldots,t_n)$ makes use of what you showed in the first part
      of this problem. 

\end{enumerate}
\end{prob}


\begin{prob}
Formalize the first-order language for the \emph{scissors-rock-paper}
game and define the first-order structure that represents its
behaviour. 

Use this to falsify the following sentence:

\[ \forall y. \exists x. r(x, y) \implies 
   \exists x. \forall y. r(x, y) . \]
\end{prob}


\begin{prob}
In the spirit of Prob.~{\ref{prob:haskell-prop}}, define the semantics
of first order formulas as a Haskell function from formulas, a first
order structure and an environment, to truth values. Using it,
formalize the counterexample in the previous problem.
\end{prob}


\begin{prob}
Prove the soundness theorem for predicate logic (classical or
intuitionistic). In more detail: show that if $\Gamma \vdash_{NK} A$ 
($A$ is derivable using natural deduction from -- some or all of -- 
the premisses in $\Gamma$) then $\Gamma \models A$, 
i.e.~for every structure ${\cal A}$ in which all the formulas in
$\Gamma$ hold, $A$ is also true.

You must do this by induction on the length of the given natural
deduction proof, considering all ten possibilities for the last rule
used in the proof ($\wedge$-intro, $\wedge$-elim, etc.). 
\end{prob}



\end{document}
