\documentclass[12pt]{article}
\usepackage[a4paper,left=27mm,right=27mm,top=10mm,bottom=15mm]{geometry}
\usepackage{graphicx,url}
\usepackage{color}
\usepackage{amssymb,comment}
\usepackage[utf8]{inputenc}
\usepackage[brazilian]{babel}
\usepackage[T1]{fontenc}

% Setting configuration for the text format
%\renewcommand{\contentsname}{Table of Contents}
%\renewcommand{\bibname}{References}
%\titleformat{\chapter}[display]{\normalfont\huge\bfseries}{\filleft\chaptername\ \thechapter}{5pt}{\filleft\Huge}
%\sloppy

\title{Lógica Matemática -- $1^a$ Avaliação}
\author{Rogério Eduardo da Silva e Claudio Cesar de Sá}
\date{\today}

\graphicspath{{/figures/}}   
\DeclareGraphicsExtensions{{.jpg},{.png}}

\begin{document}


\pagestyle{empty}
\maketitle


%\begin{LARGE}
%
%\textcolor{red}{Rogério ... a prova estah bem dimensionada ...
%basta trocar alguns operadores das questoes 1 e 2 (para não repetir o semestre passado) e tudo pronto.... \textbf{já acertei as questões 4 e 5 -- confira do livro}}
%\end{LARGE}


\begin{flushright}
``\textit{Se A é o sucesso, então A é igual a X mais Y mais Z. O trabalho é X; Y é o lazer; e Z é manter a boca fechada.}''\\ (Albert Einstein)
\end{flushright}

\begin{flushleft}
\textbf{Nome}: \rule{10cm}{0.2mm} \textbf{Turma}: \rule{1cm}{0.2mm}
\end{flushleft}
%\textbf{Matrícula}: \rule{7cm}{0.3mm} 
% %\rule{\linewidth}{1.


\begin{enumerate}
%\setlength{\itemsep}{-1pt}

\begin{comment}
\item (1.0 pt) Determina o valor lógico das fórmulas abaixo:

\begin{itemize}
\setlength{\itemsep}{-3pt}
\item $-2 < 0 \leftrightarrow \pi^2 < 0 \wedge $ Roma é a capital da Itália
\item $3+4=7 \vee 13 $ é um número primo $\rightarrow \sqrt 16 > 2$
\item $3^2 + 4^2 = 5^2  ~\wedge $ Tóquio não fica no Japão $\rightarrow ( \pi > 2.04 \leftrightarrow 2 \neq 3 )$
\item Na Espanha se fala português $\wedge~ 2^3 - 4 > 5^2 - 10 \vee 5 \neq 3+3$
\end{itemize}
\end{comment}

\item (1.0 pt) Determinar por tabela-verdade se a fórmula abaixo é uma {\bf tautologia}, {\bf contradição} (ou insatisfatível) ou {\bf contingência} (ou satisfatível): 

\begin{enumerate}
\setlength{\itemsep}{-2pt}


\item $p \leftrightarrow (p \rightarrow q \vee \sim q)$


\item $(p \rightarrow q) \rightarrow (p \vee r \leftrightarrow q \veebar r)$


\item $\sim (p \wedge \sim q) \rightarrow  p$ 


\item $ (p \rightarrow \sim p) \wedge (\sim p \rightarrow  p) $ 

\item $ (q \rightarrow \sim q) \vee (\sim q \rightarrow  q) $
\end{enumerate}


\item (3.0 pts) Determine as formas normais \underline{mais simples} (FNC e FND) equivalentes para as fórmulas abaixo: 
\begin{enumerate}
\setlength{\itemsep}{-2pt}

\item $(\sim p \vee q)  \leftrightarrow  (q \wedge \sim p) $

\item $(\sim p \wedge \sim q) \leftrightarrow p$

\item $(p \vee q) \rightarrow (p \wedge q) $

\end{enumerate}

Finalmente, reescreva-as em sua forma dual de cada uma das FNCs 
e FNDs resultantes.


\item (3.0 pts) Utilizando as propriedades e equivalências
fornecidas na página seguinte verifique {\bf SE} 
essas fórmulas apresentam uma relaç\~ao de implicaç\~ao lógica  
verdadeira:

\begin{enumerate}
\setlength{\itemsep}{-2pt}

\item $p \vee (p \wedge q) \Rightarrow p$
% % pagina 64 exercio 3 B


\item $q \leftrightarrow p \vee  q \Rightarrow p \rightarrow q$
% % pagina 64 exercio 3 D

\item $(p \rightarrow q)  \vee (p \rightarrow r) \Rightarrow p \rightarrow q \vee r $
% % pagina 64 exercio 3 F


%\item $q \Rightarrow p \wedge q \leftrightarrow q$

%\item  $ (p \vee q) \wedge \sim q \Rightarrow p $


%\item $(p \leftrightarrow \sim q) \Rightarrow (p \rightarrow q)$
%% pagina 54 Ex 3 = contingencia

%\item $q \Rightarrow p \vee q \leftrightarrow p$
%% pag. 54 ex. 2(b) = TAUTOLOGIA


%\item $(p \rightarrow q) \Rightarrow p \wedge r \rightarrow q $
%% pagina 80 Ex 10

\end{enumerate}

\item (3.0 pts) Utilizando as propriedades e algumas equivalências
fornecidas na página seguinte, demonstre {\bf SE} as equivalências abaixo 
se aplicam:

\begin{enumerate}
\setlength{\itemsep}{-2pt}

%\item $P \rightarrow Q \Leftrightarrow P \vee Q \rightarrow  Q$ %% Ex 12 da pag 80

%\item  $(p \rightarrow q) \vee (p \rightarrow r) \Leftrightarrow p  \rightarrow  (q \vee r) $ %% Ex 16 da pag 80

%\item $P \uparrow Q \Leftrightarrow ((P\downarrow P)\downarrow (Q\downarrow Q)) \downarrow ((P\downarrow P)\downarrow(Q\downarrow Q))$

%\item $p \wedge q \rightarrow r \Leftrightarrow p \rightarrow (q \rightarrow r) $  {\scriptsize (sim, é para demonstrar a regra EI)}
%%(Regra da Exportação e Importação) %% Ex 14 da pag 80

%\item  $(p \rightarrow r) \wedge (q \rightarrow r) \Leftrightarrow (p \vee  q) \rightarrow r $ %% Ex 15 da pag 80

%\item $(p \rightarrow r)  \vee (q \rightarrow s) \Leftrightarrow p \wedge q \rightarrow  r \vee s $  %% Ex 17 da pag 81

%\item $P \vee Q \Leftrightarrow (P \rightarrow Q) \rightarrow P$ % Ex 9 pag 66


\item $p \wedge (p \vee q) \Leftrightarrow p $
%Ex 3(a) pag 64

\item $(p \downarrow q) \downarrow  (p \downarrow q) \Leftrightarrow p \vee q $
%Ex 6 b  pag 64


\item $(p \rightarrow q) \rightarrow r \Leftrightarrow p \wedge \sim r \rightarrow \sim q$ % Ex 3(g) pag 64


\end{enumerate}






\end{enumerate}
\newpage

%Argumentos válidos fundamentais:
%\begin{description}
%\item[Adição (AD)] $P \vdash P \vee Q$ ou $P \vdash Q \vee P$
%\item[Simplificação (SIMP)] $P \wedge Q \vdash P$ ou $P \wedge Q \vdash Q$
%\item[Conjunção (CONJ)] $P, Q \vdash P \wedge Q$ ou $P, Q \vdash Q \wedge P$
%\item[Absorção (ABS)] $P \rightarrow Q \vdash P \rightarrow (P \wedge Q)$
%\item[Modus Ponens (MP)] $P \rightarrow Q, P \vdash Q$
%\item[Modus Tollens (MT)] $P \rightarrow Q, \sim Q \vdash \sim P$
%\item[Silogismo Disjuntivo (SD)] $P \vee Q, \sim P \vdash Q$ ou $P \vee Q, \sim Q \vdash P$
%\item[Silogismo Hipotético (SH)] $P \rightarrow Q, Q\rightarrow R \vdash P\rightarrow R$
%\item[Dilema Construtivo (DC)] $P\rightarrow Q, R\rightarrow S, P \vee R \vdash Q\vee S$
%\item[Dilema Destrutivo (DD)] $P\rightarrow Q, R\rightarrow S, \sim Q\vee\sim S \vdash \sim P \vee\sim R$
%\end{description}
%\end{enumerate}

\underline{Equivalências Notáveis}:
\begin{description}
\item[Idempotência (ID):] $p\Leftrightarrow p\wedge p$ ou $p\Leftrightarrow p\vee p$
\item[Comutação (COM):] $p\wedge q\Leftrightarrow q\wedge p$ ou $p\vee q\Leftrightarrow q\vee p$
\item[Associação (ASSOC):] $p\wedge(q\wedge r)\Leftrightarrow (p\wedge q)\wedge r$ ou $p\vee(q\vee r)\Leftrightarrow (p\vee q)\vee r$ 
\item[Distribuição (DIST):] $p\wedge(q\vee r)\Leftrightarrow (p\wedge q)\vee (p \wedge r)$ ou $p\vee(q\wedge r)\Leftrightarrow (p\vee q)\wedge (p\vee r)$
\item[Dupla Negação (DN):] $p\Leftrightarrow\sim\sim p$
\item[De Morgan (DM):] $\sim(p \wedge q) \Leftrightarrow \sim p \vee\sim q$ ou $\sim(p \vee q) \Leftrightarrow \sim p \wedge\sim q$
\item[Condicional (COND):] $p\rightarrow q \Leftrightarrow\sim p \vee q$

\item[Bicondicional (BICOND):] $p\leftrightarrow q \Leftrightarrow (p\rightarrow q)\wedge(q\rightarrow p)$

\item[Contraposição (CP):] $p\rightarrow q \Leftrightarrow \sim q\rightarrow\sim p$

\item[Exportação-Importação (EI):] $p\wedge q\rightarrow r \Leftrightarrow p\rightarrow(q\rightarrow r)$

\item[Tautologia:] $p\vee \sim p \Leftrightarrow  \blacksquare  $

\item[Contradição:] $ p\wedge \sim p \Leftrightarrow \square $

\item[Conectivos de Scheffer:] $p \uparrow q \Leftrightarrow \sim p \vee \sim q$ e $p \downarrow q \Leftrightarrow \sim p \wedge \sim q$ 

\item[Ou-exclusivo (X-or):] $p \veebar q \Leftrightarrow (p \vee q) \wedge\sim (p \wedge q)$ Obs.: $\veebar = \oplus$

\end{description}
%\bibliographystyle{ieeetr} % ieeetr or acm or apalike or alpha or splncs
%\bibliography{LMArefs.bib}

\end{document}
