\documentclass[a4paper,12pt]{article}
\usepackage[T1]{fontenc}
\usepackage[utf8]{inputenc}
\usepackage{lmodern}
\usepackage[brazil]{babel}

\usepackage{comment, color} %%% 
\usepackage{graphicx, url}
\usepackage{amsmath}
\usepackage{amsfonts}
\usepackage{amssymb}
%%%\usepackage[normalem]{ulem}

\topmargin       0.15cm
\headheight      0pt
\headsep         -0.5cm
\textheight      25cm
\textwidth       16.7cm
\oddsidemargin   -5mm
\evensidemargin  -5mm
\baselineskip    -13pt

\begin{document}
\framebox[15cm][c]{$3^a$ Avaliação de Lógica Matemática  (LMA) - Joinville, \today}

%\author{Rogério Eduardo da Silva e Claudio Cesar de Sá}
%\date{\today}

\vskip 0.5cm Acadêmico(a): \hrulefill%%%

%%%\noindent Algumas questões desta prova vieram de \url{http://www.cs.utsa.edu/~bylander/cs2233/index.html}

\begin{enumerate}
%\setlength{\itemsep}{-1pt}

\item Construa as duas fórmulas abaixo em suas respectivas FNC e FND\footnote{alguns alunos ficaram com dúvidas neste importante tópico do curso}:
\begin{enumerate}
\setlength{\itemsep}{-3pt}
\item  $(p\rightarrow \sim q)  \vee (\sim q \rightarrow p) $
\item  $p \leftrightarrow q $
\end{enumerate}

\item Verificar a validade dos teoremas abaixo, usando um dos seguintes métodos de prova:  \textbf{dedução natural}  (regras de inferência diretas e propriedades lógicas), ou pela \textbf{contradição}, ou   método da demonstração \textbf{indireta} (escolha Duas das Tres abaixo):


\begin{enumerate}
\setlength{\itemsep}{-2pt} 
 \item $\{ p\rightarrow \sim q$, $\sim p \rightarrow (r \rightarrow \sim q)$, $(\sim s \vee \sim r)\rightarrow \sim \sim q$, $\sim s$ \} {\bf $\vdash $} $\sim r$

\item $\{(\sim p\vee q) \rightarrow r$,  $(r \vee s)\rightarrow \sim t$, $t$ \} {\bf $\vdash $} $\sim q$

\item $\{ p\rightarrow q$, $q \leftrightarrow s$, $t\vee (r\wedge \sim s)$ \} {\bf $\vdash $} $p \rightarrow t$


%%\item  $\{ p\rightarrow \sim q$, $\sim p \rightarrow (r \rightarrow \sim q)$, $ (\sim s \vee \sim r)\rightarrow \sim \sim q$, $ \sim s\}$ $\vdash \sim r $




\end{enumerate}

\item Considere cada uma das proposições atômicas abaixo:
\begin{enumerate}
\setlength{\itemsep}{-2pt} 
\item $y < 0$
\item $y = 0$
\item $y > 0$
\item $x < y$
\item $x > y$
\item $x = y$
\end{enumerate}
Quais destas proposições deveriam ser escolhidas e combinadas, para
demostrar ou concluir a proposição $x < 0$? Faça suas escolhas e exiba esta demonstração.


\item \textbf{\textcolor{red}{Rogério, que tal esta questao???? terias que traduzir. Achei esta questao muito legal.}}\\ Consider the game of rock, paper, scissors.
  With two players, we will use the following six propositions:

\begin{center}
\begin{tabular}{ll}
$r_1$ & Player 1 chooses rock. \\
$p_1$ & Player 1 chooses paper. \\
$s_1$ & Player 1 chooses scissors. \\
$r_2$ & Player 2 chooses rock. \\
$p_2$ & Player 2 chooses paper. \\
$s_2$ & Player 2 chooses scissors.
\end{tabular}
\end{center}

\begin{enumerate}
\setlength{\itemsep}{-2pt} 
\item Express as a proposition: ``Each player must choose at least one
  of rock, paper, or scissors.''

\item Express as a proposition: ``Each player cannot choose more than
  one of rock, paper, or scissors.''

\item (100 pts., shared extra credit) Express as short as possible:
``Each player must choose exactly one of rock, paper, or scissors.''

\item Express as a proposition: ``The players tie.''  Assume you don't
  have to worry about the previous rule.

\item Express as a proposition: ``Player 1 wins.''
Assume you don't have to worry about the rule in part (c).
\end{enumerate}

\item Determine o valor verdade $\{V, F \}$ (a interpretação $\Phi $)
de cada uma das fórmulas abaixo, em seu respectivo domínio.
Faça os cálculos em separado e preencha a tabela abaixo.
(Determine the truth value of each statement for each domain.)

\begin{center}
\begin{tabular}{l|l|l|l|l} \hline \hline
 & \multicolumn{4}{c}{Domínios} \\ \hline
 & Num. Reais & Reais Positivos & Inteiros & Inteiros Positivos \\ \hline
$\exists x (x = -x)$ & & & & \\ \hline
$\forall x (2x \leq 3x)$ & & & & \\ \hline
$\exists x (x^2 = 2)$ & & & & \\ \hline
$\forall x (x \leq x^2)$ & & & & \\ \hline
$\forall x \exists y (xy = 1)$ & & & & \\ \hline \hline
\end{tabular}
\end{center}


\item Seja o enunciado: ``{\em \ldots para todo caminho definido de $x$ até $z$ e arco entre $z$ e $y$, então há um caminho entre $x$ e $y$. Sabe-se que todo arco 
entre $x$ e $y$ é também um caminho entre $x$ e $y$}''. Sabe-se ainda que há arcos
definidos pelas fórmulas: $arco(a,b)$, $arco(a,c)$, $arco(b,d)$,  e  $arco(c,d)$. 
Prove que é possível ir de um ponto $a$ a $e$ definido por  um 
$caminho(a,e)$ como verdade. Desta vez vamos fornecer
a fórmulas referente ao texto acima, as quais são dadas por:
\begin{enumerate}
\setlength{\itemsep}{-2pt} 
  \item  $\forall x \forall y \forall z ( caminho(x,z) \wedge arco(z,y) \rightarrow  caminho(x,y) )$
\item $\forall x \forall y ( arco(x,y)  \rightarrow  caminho(x,y) )$
\item   $arco(a,b)$
\item   $arco(a,c)$
\item   $arco(b,d)$
\item   $arco(c,d)$
\item   $arco(d,e)$
\end{enumerate}
Deduza tal caminho como verdade, indicando todas instâncias
das variáveis,  PU's, PE's e regras de inferências
utilizadas. Faça um grafo (flechas e nós) orientado para ver o que estás calculando.

\end{enumerate}

\newpage 
%%%%%%%%%%%%%%%%%%%%%%%%%%%%%%%
\underline{{\Large Equivalências Notáveis}}:
\begin{description}
\setlength{\itemsep}{-4pt}

\item[Idempotência (ID):] $P\Leftrightarrow P\wedge P$ ou $P\Leftrightarrow P\vee P$
\item[Comutação (COM):] $P\wedge Q\Leftrightarrow Q\wedge P$ ou $P\vee Q\Leftrightarrow Q\vee P$
\item[Associação (ASSOC):] $P\wedge(Q\wedge R)\Leftrightarrow (P\wedge Q)\wedge R$ ou $P\vee(Q\vee R)\Leftrightarrow (P\vee Q)\vee R$ 
\item[Distribuição (DIST):] $P\wedge(Q\vee R)\Leftrightarrow (P\wedge Q)\vee (P \wedge R)$ ou $P\vee(Q\wedge R)\Leftrightarrow (P\vee Q)\wedge (P\vee R)$
\item[Dupla Negação (DN):] $P\Leftrightarrow\sim\sim P$
\item[De Morgan (DM):] $\sim(P \wedge Q) \Leftrightarrow \sim P \vee\sim Q$ ou $\sim(P \vee Q) \Leftrightarrow \sim P \wedge\sim Q$
\item[Equivalência da Condicional (COND):] $P\rightarrow Q \Leftrightarrow\sim P \vee Q$

\item[Bicondicional (BICOND):] $P\leftrightarrow Q \Leftrightarrow (P\rightarrow Q)\wedge(Q\rightarrow P)$

\item[Contraposição (CP):] $P\rightarrow Q \Leftrightarrow \sim Q\rightarrow\sim P$

\item[Exportação-Importação (EI):] $P\wedge Q\rightarrow R \Leftrightarrow P\rightarrow(Q\rightarrow R)$

\item[Contradição:] $P\wedge \sim P \Leftrightarrow \square $

\item[Tautologia:] $ P\vee \sim P \Leftrightarrow \blacksquare    $


\end{description}

\underline{{\Large Regras Inferencias Válidas (Teoremas)}}:
\begin{description}
\setlength{\itemsep}{-4pt}
\item[Adição (AD):] $P \vdash P \vee Q$ ou $P \vdash Q \vee P$
\item[Simplificação (SIMP):] $P \wedge Q \vdash P$ ou $P \wedge Q \vdash Q$
\item[Conjunção (CONJ)] $P, Q \vdash P \wedge Q$ ou $P, Q \vdash Q \wedge P$
\item[Absorção (ABS):] $P \rightarrow Q \vdash P \rightarrow (P \wedge Q)$
\item[Modus Ponens (MP):] $P \rightarrow Q, P \vdash Q$
\item[Modus Tollens (MT):] $P \rightarrow Q, \sim Q \vdash \sim P$
\item[Silogismo Disjuntivo (SD):] $P \vee Q, \sim P \vdash Q$ ou $P \vee Q, \sim Q \vdash P$
\item[Silogismo Hipotético (SH):] $P \rightarrow Q, Q\rightarrow R \vdash P\rightarrow R$
\item[Dilema Construtivo (DC):] $P\rightarrow Q, R\rightarrow S, P \vee R \vdash Q\vee S$
\item[Dilema Destrutivo (DD):] $P\rightarrow Q, R\rightarrow S, \sim Q\vee\sim S \vdash \sim P \vee\sim R$
\end{description}
%\end{enumerate}

\begin{flushleft}
\underline{Observações}:
\begin{enumerate}
\setlength{\itemsep}{-2pt}
\item Qualquer dúvida, desenvolva a questão e deixe tudo
explicado, detalhadamente, que avaliaremos o seu conhecimentos sobre
 o assunto;
 \item \underline{Clareza e legibilidade};

\end{enumerate}
\end{flushleft}
\end{document}
