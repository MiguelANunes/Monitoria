%% This document created by Scientific Word (R) Version 3.0



\documentclass[12pt, a4paper,portuges]{article}
\usepackage{t1enc}
\usepackage[latin1]{inputenc}
\usepackage[portuges]{babel}
\usepackage{amsmath}
\usepackage{amsfonts}
\usepackage{amssymb}

%\usepackage{graphicx}
\topmargin       -1cm
 \headheight      17pt
 \headsep  1cm

\textheight      24cm
\textwidth       16.3cm
\oddsidemargin   2mm
\evensidemargin  2mm
\pagestyle{empty}

\begin{document}
\begin{center}
\framebox[\textwidth][c]{$1a.$ L�gica e Programa��o em L�gica  (LPL) -
Joinville, \today}
%%\newline
\end{center}

\vskip1cm Aluno(a): \hrulefill

%%%\noindent

\begin{enumerate}
\setlength{\itemsep}{-5pt}
 \item Fa�a a interpreta��o do valor l�gico de:
\begin{enumerate}
\setlength{\itemsep}{-5pt}

 \item $3+4=8$ se somente se $5^3=125 $;

 \item $3^2 + 4^2 = 5^2$ se
somente se $\pi $ �  n�o  for irracional;

\item $5^2=25$ ou  $\pi $ � irracional.
\end{enumerate}

 \item Identificar e simbolizar as seguintes proposi��es matem�ticas:

 \begin{enumerate}
\setlength{\itemsep}{-5pt}

 \item ``{\em x � menor que 5 e maior
que 7 ou x n�o � igual a 6}'';

 \item ``{\em
Se x � menor que 5 e maior que 3, ent�o x �
igual a 4}'';

\item ``{\em  � falso que Carlos fala ingl�s
ou alem�o, mas que n�o fala franc�s}''.

\item ``{\em Se L�gica � importante, e a
Vida tamb�m, ent�o devo estudar L�gica}''.
\end{enumerate}


\item Construindo a Tabela Verdade,
identifique se a f�rmula � tautol\'{o}gica,
contingente ( satisfat\'{i}vel,
consistente), ou inv\'{a}lida
(contradit�ria, insatisfat�vel):

\begin{enumerate}
\setlength{\itemsep}{-5pt}

%\item $(A \leftrightarrow B) \vee  (A  \wedge B)$

%\item $(A \vee B)\rightarrow (A \wedge  B)$

\item $(B \rightarrow A)\rightarrow (A \rightarrow  B)$

\item $(A \rightarrow (A \rightarrow B)) \rightarrow  B $

\item $((B \rightarrow  A) \rightarrow  A \rightarrow (B ) $
\end{enumerate}

\item Demonstre se as f�rmulas abaixo
apresentam implica��es l�gicas:
\begin{enumerate}
\setlength{\itemsep}{-5pt}

\item $q \Rightarrow p \wedge q
\leftrightarrow q$

\item $ (x=y \vee x < 4) \wedge x \geq 4
\Rightarrow x=y $

\item $(x \neq 0 \rightarrow x=y) \wedge x
\neq y \Rightarrow x=0 $

\end{enumerate}

\item Demonstre se as f�rmulas abaixo
apresentam  equival�ncias l�gicas:
\begin{enumerate}
\setlength{\itemsep}{-5pt}

\item $A \leftrightarrow B \Leftrightarrow
(\sim A \wedge \sim B)\vee( A \wedge B)$

\item $(A \rightarrow (A \rightarrow (A
\rightarrow B )))) \Leftrightarrow A
\rightarrow B $


\end{enumerate}

\item Encontre as seguintes Formas Normais:
\begin{description}
\setlength{\itemsep}{-5pt}
 \item [Disjuntiva] para: $(p \rightarrow q) \wedge \sim(q \rightarrow p) $
 \item [Conjuntiva] para: $\sim (\sim p \rightarrow q) \vee (q \rightarrow \sim p) $
\end{description}

\item (cap. 11) Verificar a validade dos
argumentos que se seguem:
\begin{enumerate}
\setlength{\itemsep}{-2pt}
 \item $p\rightarrow \sim q$, $\sim p \rightarrow (r \rightarrow \sim q)$,
 $(\sim s \vee \sim r)\rightarrow \sim \sim q$,
 $\sim s$ $\vdash $ $\sim r$

\item $\sim p\vee q \rightarrow r$,  $r \vee s
\rightarrow \sim t$, t {\bf $\vdash $} $\sim q$

\end{enumerate}

\end{enumerate}


%\vskip1cm

%\noindent

%

\end{document}
