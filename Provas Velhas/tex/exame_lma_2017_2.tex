\documentclass[11pt, a4paper,final]{article}
\usepackage{t1enc}
\usepackage[utf8]{inputenc} %%% garante mactosh
\usepackage[portuges]{babel}
\usepackage{amsmath}
\usepackage{amsfonts}
\usepackage{amssymb}
\usepackage{comment,color, fancybox} %%% 
\usepackage{graphicx, url}

%\usepackage{graphicx}
\topmargin       0cm
\headheight      0pt
\headsep  0cm
\textheight      25cm
\textwidth       16.7cm
\oddsidemargin   -3mm
\evensidemargin  -3mm
\pagestyle{empty}


\begin{document}

\begin{large}
\begin{center}

\shadowbox{
\begin{minipage}[c]{10cm}
\begin{center}
\sf
Exame Final -- 2017/2\\
L\'ogica Matem\'atica  (LMA)\\
Professores: Rogério  ($T_B$) e Claudio ($T_A$) 
\end{center}
\end{minipage}
} 
\end{center}
\end{large} 

%\vskip1cm 
\textbf{Aluno(a)}: \noindent\rule{0.7\textwidth}{1pt} TURMA: \noindent\rule{0.05\textwidth}{1pt} 
%%%\noindent

\begin{flushright}
``{\em  Nenhuma prática é tão boa como uma boa teoria.}''\\
.... de algum matemático teórico
\end{flushright}
	
%\textbf{\textcolor{red}{Rogério: a prova está BEM DIMENSIONADA, contudo as questões precisam ser recicladas. No mais intocável. Há uma questão NOVA ... veja abaixo.}}

\begin{enumerate}
%\setlength{\itemsep}{-2pt}

\item {\bf (2.0 pts)} Determinar as Formas Normais Disjuntiva e Conjuntiva para as fórmulas abaixo:

\begin{description}
\setlength{\itemsep}{-2pt}
 \item [1.]  $( (p \wedge q) \rightarrow q )~  \wedge~   (q \rightarrow \sim (p \wedge q))  $
 
 \item [2.]  $(\sim p \vee q)\:\: \leftrightarrow \:\: (\sim q \wedge p) $
\end{description}
Qual o tipo dessas fórmulas? (contingência, contradição ou tautologia). Forneça as duais, da FNC ou  da FND.

%% OK
%\item {\bf (1.5 pt)} Efetuar a prova  direta (natural) para validade dos argumentos que se seguem: 

\item {\bf (2.0 pts)} Aplique um método de prova indicado (natural, condicional ou absurdo--contradição), nas questões abaixo e resolva a validade dos argumentos que se seguem: 
%%%\textcolor{red}{nem mexi}

\begin{enumerate}
\item $\{p\rightarrow  q \: , \:\:\: q \leftrightarrow s \: , \:\:\:
 t \vee ( r \wedge \sim s)\: , \:\:\: p \} \vdash  t $ \hskip 1cm (absurdo ou natural)
 %% pag 133 exercicio 8
 
\item  $\{ ( \sim p \vee q) \rightarrow r \: ,
  \:\: (r \vee s)  \rightarrow \sim t \: ,
    \:\: t \:   \} \vdash \: \:  \sim  q $ \hskip 1cm (absurdo ou natural)
 %% pag 136 exercicio 15

\item $\{p\rightarrow \sim q \: , \:\:\: \sim p \rightarrow (r \rightarrow \sim q)  \: , \:\:\: (\sim s \vee \sim r)\rightarrow \sim \sim q  \: , \:\:\: \sim s  \} \vdash  \sim r $ \hskip 1cm (absurdo ou natural)

\item $ \{ p \vee q \rightarrow r,~ s \rightarrow \sim r \wedge \sim t,~ s \vee u \} ~\vdash~ p \rightarrow u $ \hskip 1cm (condicional)
\end{enumerate}

\begin{comment}
\begin{enumerate}

 
\item  $\{ p \vee q \: ,
  \:\: q \rightarrow r \: ,
    \:\: p \rightarrow s \: ,
     \:\: \sim s   \} \vdash r \wedge (p \vee q) $
\end{enumerate}

% Caso esse conjunto não derive um teorema, que mudanças
% você faria nas premissas para derivar $r$?

%% OK
\item {\bf (1.0 pt)} Utilizando o método de  {\em demonstração por condicional}  a validade do   argumento $ p \rightarrow u $, a partir das premissas:
%%%\rule{0.25\textwidth}{1pt}\\
$$ \{ p \vee q \rightarrow r,~ s \rightarrow \sim r \wedge \sim t,~ s \vee u \} ~\vdash~ p \rightarrow u $$
%pagina 153 -- 1l
\end{comment}  


\item {\bf (2.0 pts)} Considere o seguinte conjunto de fórmulas: 

\begin{tabular}{ll}
  % after \\: \hline or \cline{col1-col2} \cline{col3-col4} ...
1 &  $\forall x\forall y:~ q(x,y) \wedge r(y) \rightarrow p(y) $ \\
2 &  $\forall x:~  q(x,x) \rightarrow p(x)  $ \\
3 &  $\forall x\exists y:~  s(x) \rightarrow q(x,y) $ \\
4 &  $r(b)$ \\ 
5 &  $s(a)$ \\
%6 &  $s(b)$ \\
\end{tabular}\\
Utilizando as propriedades da LPO (por exemplo: PU, GU, GE e PE), 
demonstre: $\sim p(a)$ ou $p(a)$ e $\sim p(b)$ ou $p(b)$.
O domínio é dado por $D=\{a,b\}$. 

\begin{comment}

\item {\bf (1.5 pts)} Considere a seguinte interpretação em um domínio dado por $D=\{a,b\}$. 

\begin{tabular}{c |c | c | c } \hline \hline 
p(a,a) & p(a,b)  & p(b,a) & p(b,b)  \\  \hline 
 V & F & F & V \\ \hline \hline 
\end{tabular}

Determine o valor verdade ($f_{aval}$  ou $\Phi $), passo-a-passo, das seguintes fórmulas:

\begin{enumerate}
\itemsep -2pt
\item $\forall x \exists y:~ \:\: p(x,y) $
\item $\forall x \forall y:~ \:\: p(x,y) $
\item $\exists x \forall y:~ \:\: p(x,y) $
\item $\exists y:~ \:\: \sim p(a,y) $
\item $\forall x \forall y:~ \:\: (p(x,y) \rightarrow p(y,x)) $
\item $\forall x:~  \:\: p(x,x) $
\end{enumerate}
%Retirado do livro do Chang-Lee -- pagina 42 -- estava na apostila como exercício proposto

\newpage
\end{comment}

%\textbf{\textcolor{red}{Rogério: inclui esta questão NOVA ... que poderia entrar no lugar desta de baixo???? Pois a anterior é próxima 4 a 6. Que tal?}}
  

\item {\bf (2.0 pts.)} Dado o código abaixo, indique precisamente a sua saída, após   a execução do predicado \texttt{main}:
%\textcolor{red}{questao muito boa ... falta modifica-la}

\begin{tiny}
\begin{verbatim}
index(-)      
     f1(1).
     f1(2).
	
index(-)  
     f2(2).
     f2(3).
	
index(-) 
     f3(1).
     f3(2).
     f3(3).
     f3(4).
    
calculo( X, 1, R ) ?=> R = X.
calculo( X, Y, R ) => Y1 = Y-1, calculo( X, Y1, R1 ), R = R1 + X.

regra( X, Y, R ) => f1(X), f2(Y), calculo( X, Y, R ), f3(R).

%%% esta regra tem backtracking
main ?=> regra(X,Y,R), printf("\n X: %d \tY: %d \tResp: %d", X,Y,R), false.
main =>  printf("\n\n FIM DOS FATOS \n\n") , true.
\end{verbatim}
\end{tiny}

%\vskip 0.3cm

%OBS.: O predicado ``\texttt{false}''  é usado apenas para forçar o PICAT retornar todas as respostas usando o ({\em backtracking}). 

\newpage
\item {\bf (2.0 pts.)} Implementar em PICAT (\textbf{usando notação da programação em lógica}, isto é, apenas fórmulas predicativas) a funções definidas recursivamente por:

\begin{enumerate}
	\item Implemente os predicados {\tt dimensoes} (fato) e {\tt volume} (regra) para o cálculo do volume de uma caixa d'água de dimensões 
	$Largura \times Altura \times Profundidade$ que complementem a regra para o predicado {\tt main} abaixo:
	
	{\tt main => dimensoes(L,A,P), volume(L,A,P,Resp), printf("Volume: \%d",Resp).}
	
   % \item Implemente o predicado do cálculo do máximo divisor comum entre dois números, definidos pela função recursiva abaixo:\\
  % $ \operatorname{maior\_divisor}(x,y) =
  % \begin{cases}
 % x & \mbox{if } y = 0 \\
% \operatorname{maior\_divisor}(y, \operatorname{resto\_divisao}(x,y)) & \mbox{se } y > 0 \\
 % \end{cases}
% $\\
% dado: $\operatorname{resto\_divisao}(x,y) = mod(x,y)$

% \item Idem para função que calcula o número de movimentos da torre de Hanoi:\\
% $
% \operatorname{hanoi}(n) =
 % \begin{cases}
 % 1 & \mbox{if } n = 1 \\
 % 2\times\operatorname{hanoi}(n-1) + 1 & \mbox{if } n > 1\\
 % \end{cases}
% $ 

%\item Nas duas implementações anteriores calcule a sequência para  \texttt{resto\_divisao(12,3)} e 
%\texttt{hanoi(4)}.
  
 % \item Crie um predicado que converta um número decimal em seu correspondente binário de acordo com o procedimento descrito 
	% a seguir:
		% \begin{enumerate}
		% \item Dado o número decimal $N$ a ser convertido divida-o sucessivas vezes por 2 até $N=0$
		% \item Apresente todos os restos das divisões na ordem inversa que foram produzidos (ver exemplo para $N=25$ na figura abaixo)
		% \item \underline{OBS}.: em PICAT o resto da divisão é dado pelo comando {\tt X mod Y}. 
		
		% Exemplo: no {\tt R = 10 mod 3} a variável $R$ recebe o valor 1
		% \end{enumerate}
 
	% \begin{center}
	% \includegraphics[width=6cm]{dec-bin.png}
	% \end{center}
  
	\item Crie um programa em PICAT que, dado um número $X$, determine se o dado valor é primo ou não.
	
	\begin{footnotesize}
	{\tt main => numero(X),(primo(X),printf("\%d eh primo",X));printf("\%d nao eh primo",X). }
	\end{footnotesize}
	
	Lembrando que o resto de uma divisão X por Y é obtido por:\\ \texttt{Resto = X mod Y}  ou   \texttt{Resto = mod(X, Y)}
	
\end{enumerate}

%Em todos os casosa acima, ilustre com valores  numéricos (pequenos) que seu código funciona.
%\item {\bf (1.5 pts)} \textbf{\textcolor{red}{Substituir essa questão, por outro conteudo ou até mesmo podes retirar em face das duas anteriores --- ou DIMINUIR OU MUDAR ESTA}}
%No mundo do {\em faz de conta}, há uma
%companhia aérea que apresenta vôos entre diferentes
%aeroportos. Os vôos podem ser diretos ou com conexão.
%Seja o mapa de vôos diretos:

%\begin{center}
%\begin{tabular}{cc}
%1. voo\_direto(a,b) & \hskip 2cm 2. voo\_direto(a,c)\\
%3. voo\_direto(b,d) & \hskip 2cm 4. voo\_direto(c,e)\\
%5. voo\_direto(d,f) & \hskip 2cm 6. voo\_direto(e,f)\\
%7. voo\_direto(a,g) & \hskip 2cm 8. voo\_direto(g,h)\\
%9. voo\_direto(h,f) & \hskip 2cm 10. voo\_direto(h,e)\\
%\end{tabular}
%\end{center}

%Adicionando ao mapa acima  as regras de conexão abaixo, demonstre  que existe uma possível
%conexão  entre os aeroportos  $a$ até $f$.
%\begin{description}
%\setlength{\itemsep}{-2pt}
%  \item[11.]  $\forall x \forall y \forall z ( conexao(x,z) \wedge voo\_direto(z,y) \rightarrow  conexao(x,y) )$
%\item[12.] $\forall x \forall y (voo\_direto(x,y)  \rightarrow  conexao(x,y) )$
%\end{description}
  



\end{enumerate}

%%\makebox[\linewidth]{\rule{.8\paperwidth}{2pt}} \\

\noindent\rule{\textwidth}{4pt}

%%%%%%%%%%%%%%%%%%%%%%%%%%%%%%%


\underline{{\large Equivalências Notáveis}}:

{\small
\begin{description}
\setlength{\itemsep}{-2pt}

\item[Idempotência (ID):] $P\Leftrightarrow P\wedge P$ ou $P\Leftrightarrow P\vee P$
\item[Comutação (COM):] $P\wedge Q\Leftrightarrow Q\wedge P$ ou $P\vee Q\Leftrightarrow Q\vee P$
\item[Associação (ASSOC):] $P\wedge(Q\wedge R)\Leftrightarrow (P\wedge Q)\wedge R$ ou $P\vee(Q\vee R)\Leftrightarrow (P\vee Q)\vee R$ 
\item[Distribuição (DIST):] $P\wedge(Q\vee R)\Leftrightarrow (P\wedge Q)\vee (P \wedge R)$ ou $P\vee(Q\wedge R)\Leftrightarrow (P\vee Q)\wedge (P\vee R)$
\item[Dupla Negação (DN):] $P\Leftrightarrow\sim\sim P$
\item[De Morgan (DM):] $\sim(P \wedge Q) \Leftrightarrow \sim P \vee\sim Q$ ou $\sim(P \vee Q) \Leftrightarrow \sim P \wedge\sim Q$
\item[Equivalência da Condicional (COND):] $P\rightarrow Q \Leftrightarrow\sim P \vee Q$

\item[Bicondicional (BICOND):] $P\leftrightarrow Q \Leftrightarrow (P\rightarrow Q)\wedge(Q\rightarrow P)$

\item[Contraposição (CP):] $P\rightarrow Q \Leftrightarrow \sim Q\rightarrow\sim P$

\item[Exportação-Importação (EI):] $P\wedge Q\rightarrow R \Leftrightarrow P\rightarrow(Q\rightarrow R)$

\item[Contradição:] $P\wedge \sim P \Leftrightarrow \square $

\item[Tautologia:] $ P\vee \sim P \Leftrightarrow \blacksquare    $

\item[Negações para LPO:] $ \sim \forall px \Leftrightarrow \exists \sim px $

\item[Negações para LPO:] $ \sim \exists px \Leftrightarrow \forall \sim px $

\end{description}

\underline{{\large Regras Inferencias Válidas (Teoremas)}}:
\begin{description}
\setlength{\itemsep}{-2pt}
\item[Adição (AD):] $P \vdash P \vee Q$ ou $P \vdash Q \vee P$
\item[Simplificação (SIMP):] $P \wedge Q \vdash P$ ou $P \wedge Q \vdash Q$
\item[Conjunção (CONJ)] $P, Q \vdash P \wedge Q$ ou $P, Q \vdash Q \wedge P$
\item[Absorção (ABS):] $P \rightarrow Q \vdash P \rightarrow (P \wedge Q)$
\item[Modus Ponens (MP):] $P \rightarrow Q, P \vdash Q$
\item[Modus Tollens (MT):] $P \rightarrow Q, \sim Q \vdash \sim P$
\item[Silogismo Disjuntivo (SD):] $P \vee Q, \sim P \vdash Q$ ou $P \vee Q, \sim Q \vdash P$
\item[Silogismo Hipotético (SH):] $P \rightarrow Q, Q\rightarrow R \vdash P\rightarrow R$
\item[Dilema Construtivo (DC):] $P\rightarrow Q, R\rightarrow S, P \vee R \vdash Q\vee S$
\item[Dilema Destrutivo (DD):] $P\rightarrow Q, R\rightarrow S, \sim Q\vee\sim S \vdash \sim P \vee\sim R$
\end{description}
%\end{enumerate}

\begin{flushleft}
\underline{Observações}:
\begin{enumerate}
\setlength{\itemsep}{-2pt}
\item Qualquer dúvida, desenvolva a questão e deixe tudo
explicado, detalhadamente, que avaliaremos o seu conhecimentos sobre
 o assunto;
 \item \underline{Clareza e legibilidade};

\end{enumerate}
\end{flushleft}
\noindent Boas férias!
}\\
\noindent\rule{\textwidth}{4pt}

\end{document}
