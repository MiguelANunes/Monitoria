
\documentclass[12pt, a4paper]{article}
\usepackage{t1enc}
\usepackage[latin1]{inputenc}
\usepackage[brazil,portuges]{babel}
\usepackage{amsmath}
\usepackage{amsfonts}
\usepackage{amssymb}

%\usepackage{graphicx}
\topmargin       0cm
 \headheight      0pt
 \headsep  0cm

\textheight      24cm
\textwidth       16.7cm
\oddsidemargin   -2mm
\evensidemargin  -2mm
\pagestyle{empty}

\begin{document}
\begin{center}
\fbox{
\framebox[\textwidth][c]{$1a.$ Prova -- L�gica e Programa��o em L�gica  (LPL)}
}
%%\newline
\end{center}

\vskip1cm Aluno(a): \hrulefill

%%%\noindent

\begin{enumerate}
%\setlength{\itemsep}{-5pt}
\item Transforme as f�rmulas abaixo em f�rmulas equivalentes reescritas como
$\wedge $ e $\sim$; $\vee $ e $\sim$; e $\rightarrow $ e $\sim$ :
\begin{enumerate}
\setlength{\itemsep}{-3pt}
\item $(p \wedge (p \rightarrow q)) \rightarrow q$
\item $(\sim q \wedge (p \rightarrow q)) \rightarrow \sim p$

\end{enumerate}

\item Com base na quest�o anterior, das f�rmulas obtidas com 
$\wedge $ e $\sim$; $\vee $ e $\sim$; obtenha as respectivas duais
destas 06 f�rmulas. Estas duais equivalentes s�o equivalentes? Quais?
 Porqu�?

\item Construindo a Tabela Verdade, identifique se a f�rmula � tautol\'{o}gica,
contingente (satisfat\'{i}vel, consistente), ou inv\'{a}lida
(contradit�ria, insatisfat�vel):

\begin{enumerate}
\setlength{\itemsep}{-3pt}

\item $(\sim p \leftrightarrow \sim q) \vee  (p  \leftrightarrow q)$

\item $(\sim p \vee  \sim q) \rightarrow (p \wedge  q)$

\item $(p \rightarrow \sim q) \rightarrow (q \rightarrow  \sim p)$

%\item $(A \rightarrow (\sim A \rightarrow B)) \rightarrow  B $
%\item $(B \rightarrow  A) \rightarrow  (A  \rightarrow \sim B)  $
\end{enumerate}


\item Encontre as  Formas Normais (FN) das  f�rmulas  abaixo:
\begin{description}
\setlength{\itemsep}{-3pt}
 \item [Disjuntiva] para: $(q \rightarrow p) \wedge \sim (q \rightarrow p) $
 \item [Conjuntiva] para: $\sim (\sim p \rightarrow q) \vee (\sim q \rightarrow \sim p) $
 
\end{description}

\item Demonstre se as f�rmulas abaixo
apresentam  equival�ncias l�gicas:
\begin{enumerate}
\setlength{\itemsep}{-3pt}

%\item $p \leftrightarrow p \wedge q \Leftrightarrow p \rightarrow q$

\item $(p \rightarrow r) \vee (q \rightarrow r) \Leftrightarrow  p  \wedge q  \rightarrow r $

\item $(p \rightarrow q) \wedge (p \rightarrow r) \Leftrightarrow p \rightarrow q \wedge r $

\end{enumerate}


\item Demonstrar que o conjunto das
proposi��es abaixo geram uma contradi��o (isto �,
derivam ma inconsist�ncia, i. �: $\Box \Leftrightarrow (\sim x \wedge x)$).
 Escolher uma das quest�es abaixo:
\begin{enumerate}
\setlength{\itemsep}{-2pt}

\item %%\vskip 11pt
\begin{tabular}{ll}
  % after \\: \hline or \cline{col1-col2} \cline{col3-col4} ...
    1 &  $p \vee (q \wedge r) $ \\
    2 &  $p \rightarrow q$ \\
    3 &  $s \rightarrow r$ \\ 
    4 &  $ \sim r $ 
\end{tabular}
\item \vskip 11pt
\begin{tabular}{ll}
  % after \\: \hline or \cline{col1-col2} \cline{col3-col4} ...
    1 &  $ p \vee s \rightarrow q$ \\
    2 &  $q \rightarrow \sim r$ \\
    3 &  $t \rightarrow p$ \\
    4 &  $t \wedge r $
\end{tabular}

\end{enumerate}

\item Verificar a validade dos argumentos (leia-se, estes s�o teoremas l�gicos) que se seguem:
\begin{enumerate}
\setlength{\itemsep}{-2pt}
 \item $ p \rightarrow q$, $ r \rightarrow s$, $(q \vee s) \rightarrow  \sim t$,
 $t$ $\vdash $ $\sim p \wedge \sim r$
 %% livro pag 137

\item $\sim p \vee q \rightarrow r$,  $(r \vee s) \rightarrow \sim t$, $t$  $\vdash $ $\sim q$
%% livro pag 136
\end{enumerate}
%%%%%%%%%%%%%%%%% fim 

\end{enumerate}

PS: Legibilidade e organiza��o na prova!

\end{document}
