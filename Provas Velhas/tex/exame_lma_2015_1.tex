
\documentclass[11pt, a4paper,final]{article}
\usepackage{t1enc}
\usepackage[utf8]{inputenc} %%% garante mactosh
\usepackage[portuges]{babel}
\usepackage{amsmath}
\usepackage{amsfonts}
\usepackage{amssymb}
\usepackage{comment, color} %%% 

%\usepackage{graphicx}
\topmargin       -1cm
\headheight      0pt
\headsep  0cm
\textheight      27cm
\textwidth       16.7cm
\oddsidemargin   -5mm
\evensidemargin  -5mm
\pagestyle{empty}


\begin{document}
\begin{center}
\begin{tabular}{||c||} \hline \hline 
{\Large Logica Matemática  (LMA)}  \\
\mbox{\hskip 2cm  UDESC/DCC -- \today  \hskip 2cm }
\\
Exame Final   \\ \hline \hline
\end{tabular}
\end{center}
%\vskip1cm 
\textbf{Aluno(a)}: \noindent\rule{0.7\textwidth}{1pt} TURMA: \noindent\rule{0.05\textwidth}{1pt} 
%%%\noindent

\begin{flushright}
``{\em  Logic is a systematic method of coming to the wrong conclusion with confidence.}''\\
Samuel Butler (1835--1902)
\end{flushright}

%\textbf{\textcolor{red}{Rogério: a prova está BEM DIMENSIONADA, contudo as questões precisam ser recicladas. No mais intocável. Há uma questão NOVA ... veja abaixo.}}

\begin{enumerate}
%\setlength{\itemsep}{-2pt}

\item {\bf (2.0 pts)} Determinar as Formas Normais Disjuntiva e Conjuntiva para:

\begin{description}
\setlength{\itemsep}{-2pt}
 \item [1.]  $( p \rightarrow q) \wedge  (\sim q \rightarrow p) $
 \item [2.]  $(\sim p \rightarrow \:\: \sim q) \vee (\sim q \rightarrow p) $
\end{description}
Qual o tipo dessas fórmulas? (contingência, contradição ou tautologia). Forneça as duais, da FNC ou  da FND (ou seja, apenas uma).

%% OK
%\item {\bf (1.5 pt)} Efetuar a prova  direta (natural) para validade dos argumentos que se seguem: 

\item {\bf (2.0 pts)} Escolha e indique um método de prova (natural, condicional ou absurdo) e resolva a validade dos argumentos que se seguem: 

\begin{enumerate}
\item $\{p\rightarrow  q \: , \:\:\: q \leftrightarrow s \: , \:\:\:
 t \vee ( r \wedge \sim s)\: , \:\:\: p \} \vdash  t $ 
 %% pag 133 exercicio 8
 
\item  $\{ ( \sim p \vee q) \rightarrow r \: ,
  \:\: (r \vee s)  \rightarrow \sim t \: ,
    \:\: t \:   \} \vdash \: \:  \sim  q $
 %% pag 136 exercicio 15

\item $\{p\rightarrow \sim q \: , \:\:\: \sim p \rightarrow (r \rightarrow \sim q)  \: , \:\:\: (\sim s \vee \sim r)\rightarrow \sim \sim q  \: , \:\:\: \sim s  \} \vdash  \sim r $ 

    
\end{enumerate}

\begin{comment}
\begin{enumerate}

 
\item  $\{ p \vee q \: ,
  \:\: q \rightarrow r \: ,
    \:\: p \rightarrow s \: ,
     \:\: \sim s   \} \vdash r \wedge (p \vee q) $
\end{enumerate}
\end{comment}  
% Caso esse conjunto não derive um teorema, que mudanças
% você faria nas premissas para derivar $r$?

%% OK
\item {\bf (1.0 pt)} Utilizando o método de  {\em demonstração por condicional}  a validade do   argumento $ p \rightarrow u $, a partir das premissas:
%%%\rule{0.25\textwidth}{1pt}\\
$$ \{ p \vee q \rightarrow r,~ s \rightarrow \sim r \wedge \sim t,~ s \vee u \} ~\vdash~ p \rightarrow u $$
%pagina 153 -- 1l




\item {\bf (1.5 pts)} Considere o seguinte conjunto de f\'ormulas: 
%%%{\bf \textcolor{red}{é boa esta questão ... manteria !}}

\begin{tabular}{ll}
  % after \\: \hline or \cline{col1-col2} \cline{col3-col4} ...
1 &  $\forall x\forall y (q(x,y) \wedge r(y) \rightarrow p(y)) $ \\
2 &  $\forall x  (q(x,x) \rightarrow p(x))  $ \\
3 &  $\forall x \exists y ( s(x) \rightarrow q(x,y)) $ \\
4 &  $r(b)$ \\ 
5 &  $s(a)$ \\
6 &  $s(b)$ \\
\end{tabular}\\
Utilizando as propriedades da LPO (por exemplo: PU, GU, GE e PE), demonstre: $\sim p(a)$ ou $p(a)$ e $\sim p(b)$ ou $p(b)$.
O domínio é dado por $D=\{a,b\}$. 


\item {\bf (1.5 pts)} Considere a seguinte interpretação em um domínio dado por $D=\{a,b\}$. 

\begin{tabular}{c |c | c | c } \hline \hline 
p(a,a) & p(a,b)  & p(b,a) & p(b,b)  \\  \hline 
 V & F & F & V \\ \hline \hline 
\end{tabular}

Determine o valor verdade ($f_{aval}$  ou $\Phi $), passo-a-passo, das seguintes fórmulas:

\begin{enumerate}
\itemsep -2pt
\item $\forall x \exists y \:\: p(x,y) $
\item $\forall x \forall y \:\: p(x,y) $
\item $\exists x \forall y \:\: p(x,y) $
\item $\exists y \:\: \sim p(a,y) $
\item $\forall x \forall y \:\: (p(x,y) \rightarrow p(y,x)) $
\item $\forall x  \:\: p(x,x) $
\end{enumerate}
Retirado do livro do Chang-Lee -- pagina 42


%\textbf{\textcolor{red}{Rogério: inclui esta questão NOVA ... que poderia entrar no lugar desta de baixo???? Pois a anterior é próxima 4 a 6. Que tal?}}


\begin{comment}
\item {\bf (1.5 pts)} No mundo do {\em faz de conta}, há uma
companhia aérea que apresenta vôos entre diferentes
aeroportos. Os vôos podem ser diretos ou com conexão.
Seja o mapa de vôos diretos:


\begin{center}
\begin{tabular}{cc}
1. voo\_direto(a,b) & \hskip 2cm 2. voo\_direto(a,c)\\
3. voo\_direto(b,d) & \hskip 2cm 4. voo\_direto(c,e)\\
5. voo\_direto(d,f) & \hskip 2cm 6. voo\_direto(e,f)\\
7. voo\_direto(a,g) & \hskip 2cm 8. voo\_direto(g,h)\\
9. voo\_direto(h,f) & \hskip 2cm 10. voo\_direto(h,e)\\
\end{tabular}
\end{center}

Dada as regras de conexão abaixo, demonstre  que existe uma possível
conexão  entre os aeroportos  $a$ até $f$.
\begin{description}
\setlength{\itemsep}{-2pt}
  \item[11.]  $\forall x \forall y \forall z ( conexao(x,z) \wedge voo\_direto(z,y) \rightarrow  conexao(x,y) )$
\item[12.] $\forall x \forall y (voo\_direto(x,y)  \rightarrow  conexao(x,y) )$
\end{description}
\end{comment}

\item {\bf (1.0 pt)} Expresse {\bf em Prolog} o seguinte texto e resolva: 

``\textit{Tweety  é um pássaro. Goldie  é um peixe. Molie é uma minhoca. Pássaros gostam de minhocas. Gatos gostam de peixes. Gatos gostam de pássaros.
 O meu gato come tudo o que gosta. O meu gato chama-se Silvester.}''

%``\textit{Laranja é uma fruta. Lisa é uma sabiá. Molie é uma minhoca. Todo sabiá é um pássaro. Todos os pássaros comem minhocas ou frutas.}''

\begin{enumerate}
  \item Escreva o texto acima em fórmulas de primeira ordem e, em seguida, um código equivalente em Prolog;
 \item Use uma das formalizações do item anterior para determinar {\bf tudo} o que Silvester come.
 % \item Da resposta do item anterior, a resposta é razoável? Se não for, verifique se o problema está na especificação original ou na sua tradução para Prolog, corrija o seu programa e veja uma nova resposta para o que Silvester come.
\end{enumerate}

\item {\bf (1.0 pt)} Apresente \textbf{todas as saídas na ordem} para a execução do predicado {\texttt go}:
\begin{verbatim}
	t(*) .
	t(#).
	r(a).
	r(b) .
	r(c).
	s(1).
	s(2).
	p(X, Y, Z) :- t( X ) , r( Y ) , s( Z ) .
	go :- p(X, Y, Z), write(X), write(Y), write(Z).
\end{verbatim}


\end{enumerate}

%%\makebox[\linewidth]{\rule{.8\paperwidth}{2pt}} \\

\noindent\rule{0.8\textwidth}{2pt}

%%%%%%%%%%%%%%%%%%%%%%%%%%%%%%%


\underline{{\large Equivalências Notáveis}}:

{\small
\begin{description}
\setlength{\itemsep}{-2pt}

\item[Idempotência (ID):] $P\Leftrightarrow P\wedge P$ ou $P\Leftrightarrow P\vee P$
\item[Comutação (COM):] $P\wedge Q\Leftrightarrow Q\wedge P$ ou $P\vee Q\Leftrightarrow Q\vee P$
\item[Associação (ASSOC):] $P\wedge(Q\wedge R)\Leftrightarrow (P\wedge Q)\wedge R$ ou $P\vee(Q\vee R)\Leftrightarrow (P\vee Q)\vee R$ 
\item[Distribuição (DIST):] $P\wedge(Q\vee R)\Leftrightarrow (P\wedge Q)\vee (P \wedge R)$ ou $P\vee(Q\wedge R)\Leftrightarrow (P\vee Q)\wedge (P\vee R)$
\item[Dupla Negação (DN):] $P\Leftrightarrow\sim\sim P$
\item[De Morgan (DM):] $\sim(P \wedge Q) \Leftrightarrow \sim P \vee\sim Q$ ou $\sim(P \vee Q) \Leftrightarrow \sim P \wedge\sim Q$
\item[Equivalência da Condicional (COND):] $P\rightarrow Q \Leftrightarrow\sim P \vee Q$

\item[Bicondicional (BICOND):] $P\leftrightarrow Q \Leftrightarrow (P\rightarrow Q)\wedge(Q\rightarrow P)$

\item[Contraposição (CP):] $P\rightarrow Q \Leftrightarrow \sim Q\rightarrow\sim P$

\item[Exportação-Importação (EI):] $P\wedge Q\rightarrow R \Leftrightarrow P\rightarrow(Q\rightarrow R)$

\item[Contradição:] $P\wedge \sim P \Leftrightarrow \square $

\item[Tautologia:] $ P\vee \sim P \Leftrightarrow \blacksquare    $

\item[Negações para LPO:] $ \sim \forall px \Leftrightarrow \exists \sim px $

\item[Negações para LPO:] $ \sim \exists px \Leftrightarrow \forall \sim px $

\end{description}

\underline{{\large Regras Inferencias Válidas (Teoremas)}}:
\begin{description}
\setlength{\itemsep}{-2pt}
\item[Adição (AD):] $P \vdash P \vee Q$ ou $P \vdash Q \vee P$
\item[Simplificação (SIMP):] $P \wedge Q \vdash P$ ou $P \wedge Q \vdash Q$
\item[Conjunção (CONJ)] $P, Q \vdash P \wedge Q$ ou $P, Q \vdash Q \wedge P$
\item[Absorção (ABS):] $P \rightarrow Q \vdash P \rightarrow (P \wedge Q)$
\item[Modus Ponens (MP):] $P \rightarrow Q, P \vdash Q$
\item[Modus Tollens (MT):] $P \rightarrow Q, \sim Q \vdash \sim P$
\item[Silogismo Disjuntivo (SD):] $P \vee Q, \sim P \vdash Q$ ou $P \vee Q, \sim Q \vdash P$
\item[Silogismo Hipotético (SH):] $P \rightarrow Q, Q\rightarrow R \vdash P\rightarrow R$
\item[Dilema Construtivo (DC):] $P\rightarrow Q, R\rightarrow S, P \vee R \vdash Q\vee S$
\item[Dilema Destrutivo (DD):] $P\rightarrow Q, R\rightarrow S, \sim Q\vee\sim S \vdash \sim P \vee\sim R$
\end{description}
%\end{enumerate}

\begin{flushleft}
\underline{Observações}:
\begin{enumerate}
\setlength{\itemsep}{-2pt}
\item Qualquer dúvida, desenvolva a questão e deixe tudo
explicado, detalhadamente, que avaliaremos o seu conhecimentos sobre
 o assunto;
 \item \underline{Clareza e legibilidade};

\end{enumerate}
\end{flushleft}
\noindent Boas férias!
}
%% detalhamento.
%\noindent

\end{document}
