\documentclass[a4paper,12pt]{article}
\usepackage[T1]{fontenc}
\usepackage[utf8]{inputenc} %% isto garante compatibilidade com seu MAC
\usepackage{lmodern}
\usepackage[brazil]{babel}

\usepackage{comment, color} %%% 
\usepackage{graphicx, url}
\usepackage{amsmath}
\usepackage{amsfonts}
\usepackage{amssymb}
%%%\usepackage[normalem]{ulem}

\topmargin       0.15cm
\headheight      0pt
\headsep         -0.5cm
\textheight      25cm
\textwidth       16.7cm
\oddsidemargin   -5mm
\evensidemargin  -5mm
\baselineskip    -13pt

\begin{document}
\framebox[15cm][c]{$3^a$ Avaliação de Lógica Matemática  (LMA) - Joinville, \today}

%\author{Rogério Eduardo da Silva e Claudio Cesar de Sá}
%\date{\today}

\vskip 0.5cm Acadêmico(a): \hrulefill%%%

%%%\noindent Algumas questões desta prova vieram de \url{http://www.cs.utsa.edu/~bylander/cs2233/index.html}

\begin{enumerate}
%\setlength{\itemsep}{-1pt}

%\item Construa as duas fórmulas abaixo em suas respectivas FNC e FND\footnote{alguns alunos ficaram com dúvidas neste importante tópico do curso}:
%%%\textcolor{red}{Rogério: Não mexi aqui mas sugiro esta questao}
%\begin{enumerate}
%\setlength{\itemsep}{-3pt}
%\item  $(p\rightarrow \sim q)  \wedge (\sim q \rightarrow p) $
%\item  $p \leftrightarrow \sim q $
%\end{enumerate}
%
%\item Verificar a validade dos teoremas abaixo, usando um dos seguintes métodos de prova:  \textbf{dedução natural}  (regras de inferência diretas e propriedades lógicas), ou pela \textbf{contradição}, ou   método da demonstração \textbf{indireta} (escolha Duas das Tres abaixo):\\
%\textcolor{red}{Rogério: Não mexi aqui ... confira ou pegue alguma outra}
%
%\begin{enumerate}
%\setlength{\itemsep}{-2pt} 
% \item $\{ p\rightarrow \sim q$, $\sim p \rightarrow (r \rightarrow \sim q)$, $(\sim s \vee \sim r)\rightarrow \sim \sim q$, $\sim s$ \} {\bf $\vdash $} $\sim r$
%
%\item $\{(\sim p\vee q) \rightarrow r$,  $(r \vee s)\rightarrow \sim t$, $t$ \} {\bf $\vdash $} $\sim q$
%
%\item $\{ p\rightarrow q$, $q \leftrightarrow s$, $t\vee (r\wedge \sim s)$ \} {\bf $\vdash $} $p \rightarrow t$
%
%
%%%\item  $\{ p\rightarrow \sim q$, $\sim p \rightarrow (r \rightarrow \sim q)$, $ (\sim s \vee \sim r)\rightarrow \sim \sim q$, $ \sim s\}$ $\vdash \sim r $
%
%
%\end{enumerate}


\item {\bf (2.0 pt)} Determine o valor verdade $\{V, F \}$ (a interpretação $\Phi $)
de cada uma das fórmulas abaixo em seu respectivo domínio.
Faça os cálculos em separado e preencha a tabela abaixo.
%(Determine the truth value of each statement for each domain.)\\
%\textbf{\textcolor{red}{Rogério: Já alterei as questoes do semestre anterior ...}}
\begin{center}
\begin{tabular}{l|l|l|l|l} \hline \hline
 & \multicolumn{4}{c}{Domínios} \\ \hline
 & Num. Reais & Reais Positivos & Inteiros & Inteiros Positivos \\ \hline
$\exists x (x = -x^2)$ & & & & \\ \hline
$\forall x (3x \leq 3x)$ & & & & \\ \hline
$\exists x (x^2 = 16)$ & & & & \\ \hline
$\forall x (x^4 \geq x^2)$ & & & & \\ \hline
$\forall x \exists y (xy = 2)$ & & & & \\ \hline \hline
\end{tabular}
\end{center}


%\item {\bf (1.0 pt)} Escreva claramente uma descrição para as fórmula abaixo, indicando
%quais são logicamente equivalentes entre elas (da ``a'' a ``e''), e quais não são.
%Explique suas respostas.
%%% pagina 371 do livro em ingles que estah no Dropbox
%%\textcolor{red}{Rogério: esta é nova ... veja se está bom}
%\begin{enumerate}
%\setlength{\itemsep}{-3pt}
%  \item  $\exists x \sim p(x)$ 
%  \item  $ \exists x \forall y (p(y) \rightarrow y = x)$
%  \item  $ \exists x \forall y (p(y) \leftrightarrow y = x)$
%   \item $ \forall x \forall y (p(x) \wedge p(y)) \rightarrow x = y)$
%   \item $ \forall x \forall y (p(x) \wedge p(y)) \leftrightarrow x = y)$
%\end{enumerate}

\item {\bf (1.5 pt)} Aplicando De Morgan aos
quantificadores das fórmulas de LPO, dar a
negação das seguintes sentenças lógicas:
\begin{enumerate}
\setlength{\itemsep}{-2pt}
% \item $ \forall x \exists y (\sim p(x) \wedge \sim q(y))$
 \item $ \exists x \forall y (p(x) \vee \sim q(y))$
% \item $ \exists x \forall y (p(x)\rightarrow q(y))$
 \item $ \forall x \exists y (\sim p(x) \vee \sim q(y))$
  \item  $ \exists x \forall y (p(y) \rightarrow y = x)$
  \item  $ \forall x \exists y (p(y) \leftrightarrow y = x)$
\end{enumerate}

\item {\bf (3.5 pts)} Seja o conjunto das seguintes fórmulas em lógica de primeira-ordem (LPO):\\
\begin{center}
\begin{tabular}{ll}
 \hline \hline
  % after \\: \hline or \cline{col1-col2} \cline{col3-col4} ...
    1. &  $\forall x\forall y (bebe(y) \wedge genitores(x,y) \rightarrow orgulhoso(x)) $ \\
    2. &  $\forall x\forall y ( pai(x,y) \rightarrow genitores(x,y) ) $ \\
    3. &  $\forall x\forall y ( mae(x,y) \rightarrow genitores(x,y) ) $ \\
    4. &  $ pai(adam, maria) $ \\
    5. &  $ bebe(maria) $ \\
    6. &  $ mae(beatriz, maria) $ \\ \hline \hline
 \end{tabular}
\end{center}
 \\
Na sequência abaixo, resolva as seguintes questões:
\begin{enumerate}
\setlength{\itemsep}{-3pt}
\item {\bf (1.0 pt)} Interprete textualmente o significado de cada fórmula acima
\item {\bf (2.0 pts)} Utilizando as propriedades da LPO, PU's, PE's e regras de inferências, deduza que \texttt{beatriz} e \texttt{adam}
são pais orgulhosos de seu bebe chamado \texttt{maria}.
\end{enumerate}
PS: Indique claramente cada passo realizado.

\item {\bf (1.5 pt)} Em Prolog, a partir da relação \texttt{pai(X,Y)}, construa a regras para deduzir \texttt{avo(X,Y)} e \texttt{tio(X,Y)}

\item {\bf (1.5 pt)} Analise o código Prolog apresentado a seguir e informe qual seria a sequência de respostas válidas para a inferência {\tt person(X,Y).} {\small (na mesma ordem que seria apresentada pelo Prolog)}:

\begin{center}
\begin{tabular}{ll}
 \hline \hline
  % after \\: \hline or \cline{col1-col2} \cline{col3-col4} ...
    1. &  $record(maejoana,1940).$ \\
    2. &  $record(zeninguem, 2001).$ \\
    3. &  $record(fulana,1955).$ \\
    4. &  $record(fulano, 1992).$\\
    5. &  $person(X,old) :- record(X,Y), Y=<1955.$ \\
    6. &  $person(X,adult) :- record(X,Y), Y>1955, Y=<2000.$ \\
    7. &  $person(X,young) :- record(X,Y), Y>2000$ \\ \hline \hline
 \end{tabular}
\end{center}

\end{enumerate}


%%%%%%%%%%%%%%%%%%%%%%%%%%%%%%%
\underline{{\large Equivalências Notáveis}}:
\begin{description}
\setlength{\itemsep}{-4pt}

\item[Idempotência (ID):] $P\Leftrightarrow P\wedge P$ ou $P\Leftrightarrow P\vee P$
\item[Comutação (COM):] $P\wedge Q\Leftrightarrow Q\wedge P$ ou $P\vee Q\Leftrightarrow Q\vee P$
\item[Associação (ASSOC):] $P\wedge(Q\wedge R)\Leftrightarrow (P\wedge Q)\wedge R$ ou $P\vee(Q\vee R)\Leftrightarrow (P\vee Q)\vee R$ 
\item[Distribuição (DIST):] $P\wedge(Q\vee R)\Leftrightarrow (P\wedge Q)\vee (P \wedge R)$ ou $P\vee(Q\wedge R)\Leftrightarrow (P\vee Q)\wedge (P\vee R)$
\item[Dupla Negação (DN):] $P\Leftrightarrow\sim\sim P$
\item[De Morgan (DM):] $\sim(P \wedge Q) \Leftrightarrow \sim P \vee\sim Q$ ou $\sim(P \vee Q) \Leftrightarrow \sim P \wedge\sim Q$
\item[Equivalência da Condicional (COND):] $P\rightarrow Q \Leftrightarrow\sim P \vee Q$

\item[Bicondicional (BICOND):] $P\leftrightarrow Q \Leftrightarrow (P\rightarrow Q)\wedge(Q\rightarrow P)$

\item[Contraposição (CP):] $P\rightarrow Q \Leftrightarrow \sim Q\rightarrow\sim P$

\item[Exportação-Importação (EI):] $P\wedge Q\rightarrow R \Leftrightarrow P\rightarrow(Q\rightarrow R)$

\item[Contradição:] $P\wedge \sim P \Leftrightarrow \square $

\item[Tautologia:] $ P\vee \sim P \Leftrightarrow \blacksquare    $

\item[Negações para LPO:] $ \sim \forall px \Leftrightarrow \exists \sim px $
\item[Negações para LPO:] $ \sim \exists px \Leftrightarrow \forall \sim px $
\end{description}

\underline{{\large Regras Inferencias Válidas (Teoremas)}}:
\begin{description}
\setlength{\itemsep}{-4pt}
\item[Adição (AD):] $P \vdash P \vee Q$ ou $P \vdash Q \vee P$
\item[Simplificação (SIMP):] $P \wedge Q \vdash P$ ou $P \wedge Q \vdash Q$
\item[Conjunção (CONJ)] $P, Q \vdash P \wedge Q$ ou $P, Q \vdash Q \wedge P$
\item[Absorção (ABS):] $P \rightarrow Q \vdash P \rightarrow (P \wedge Q)$
\item[Modus Ponens (MP):] $P \rightarrow Q, P \vdash Q$
\item[Modus Tollens (MT):] $P \rightarrow Q, \sim Q \vdash \sim P$
\item[Silogismo Disjuntivo (SD):] $P \vee Q, \sim P \vdash Q$ ou $P \vee Q, \sim Q \vdash P$
\item[Silogismo Hipotético (SH):] $P \rightarrow Q, Q\rightarrow R \vdash P\rightarrow R$
\item[Dilema Construtivo (DC):] $P\rightarrow Q, R\rightarrow S, P \vee R \vdash Q\vee S$
\item[Dilema Destrutivo (DD):] $P\rightarrow Q, R\rightarrow S, \sim Q\vee\sim S \vdash \sim P \vee\sim R$
\end{description}
%\end{enumerate}

\begin{flushleft}
\underline{Observações}:
\begin{enumerate}
\setlength{\itemsep}{-2pt}
\item Qualquer dúvida, desenvolva a questão e deixe tudo
explicado, detalhadamente, que avaliaremos o seu conhecimentos sobre
 o assunto;
 \item \underline{Clareza e legibilidade};

\end{enumerate}
\end{flushleft}
\end{document}
