\documentclass[12pt, a4paper,final]{article}
\usepackage{t1enc}
\usepackage[utf8]{inputenc}
\usepackage[portuges,brazilian]{babel}

\usepackage{amsmath}
\usepackage{amsfonts}
\usepackage{amssymb}
\usepackage{comment,color, fancybox} %%% 
%\usepackage{tikz}
%\usepackage{comment, color } %%% inclui

%%%\usepackage{graphicx,url}
\topmargin       0cm 
\headheight      0pt 
\headsep         0cm
\textheight      24cm
\textwidth       16.7cm
\oddsidemargin   -2mm
\evensidemargin  -2mm
\pagestyle{empty}

%%%\graphicspath{{/figures/}}   
%%\DeclareGraphicsExtensions{{.jpg},{.png}}
%%https://logicproblems.org/problems/

\begin{document}


\begin{large}
\begin{center}

\shadowbox{
\begin{minipage}[c]{10cm}
\begin{center}
\sf
$2^{\underline{a}}$ Avalia\c c\~ao de L\'ogica Matem\'atica  (LMA)\\
%Professores: Claudio ($T_A$) e Rogério ($T_B$) \\
Joinville, \today
\end{center}
\end{minipage}
} %% 

\end{center}
\end{large} 

%{\bf\textcolor{red}{Rogerio ... amanha pegamos novas questoes do livro}}


\vskip 0.5cm Acad\^emico(a) : \rule{10cm}{0.4pt} Turma:  $\frac{[ ~~ ] ~ T_A - Claudio}{[ ~~ ] ~ T_B - Rogerio}$ 
\noindent

\begin{enumerate}
%\setlength{\itemsep}{-1pt}

\itemsep 1cm

\item Verificar a \textbf{validade dos argumentos} (dedu\c c\~ao natural ou direta) que se seguem:
%\textbf{\textcolor{red}{Confira .... com as paginas do livro}}
\begin{enumerate}


%\item $\{r  \rightarrow t,~ t \rightarrow \sim s, ~ (r \rightarrow \sim s) \rightarrow q, ~ p  \} ~\vdash~ p \wedge q $
%pagina 153 -- 1c



%\item $\{\sim p  \vee \sim s,~ q \rightarrow \sim r, ~ t \rightarrow ~ (r \wedge s) , ~ t  \} ~\vdash~ \sim (p \vee q) $
%pagina 153 -- 1g



\item $\{q \rightarrow p, ~ t \vee s, ~ q \vee \sim s, \sim(p \vee r)  \} ~\vdash~ t $
%pagina 153 -- 1k


\item $\{ p \vee q \rightarrow r,~ s \rightarrow \sim r \wedge \sim t,~ s \vee u, ~ p \} ~\vdash~  u$ 
%pagina 153 -- 1l -- MODIFICADA ...

\item $\{ p \rightarrow q,~ r \rightarrow t,~ s \rightarrow r,~ p \vee s \} ~\vdash~ \sim q \rightarrow t $
% pagina 153 -- 1m




%\item $\{ p \vee \sim q,~ \sim p,~ \sim (p \wedge r) \rightarrow q \} ~\vdash~ r$
%% pagina 110 -- 1h

%\item $\{ \sim (p \vee q),~ \sim p \wedge \sim q \rightarrow r \wedge s,~ s \rightarrow r \} ~\vdash~ r$
%% pagina 111 -- 4b modificado

%\item $\{ p \vee q,~ q \rightarrow r,~ \sim r \vee s,~ \sim p \} ~\vdash~ s$
%% pagina 110 -- 2c

%\item $\{ p \rightarrow q, \:\: p \vee (\sim \sim r \wedge \sim \sim  q), s \rightarrow \sim r, \sim (p \wedge q) \}$ {\bf $\vdash $} $\sim (s \wedge q)$
 %% pagina 116 -- 11

%\item $\{ p \rightarrow q, \:\: \sim r \rightarrow (s \rightarrow t),\:\: r \vee (p \vee s), \:\: \sim r \}$ {\bf $\vdash $} $q \vee t$
 %% pagina 115 -- 9


%\item $\{ p \rightarrow q, \:\: q \rightarrow r, \:\: r \rightarrow s, \:\: \sim s, \:\: p \vee t \}$ {\bf $\vdash $} $t$
 %% pagina 117 -- 13


%\item $\{ p \rightarrow q, \:\: q \rightarrow r, \:\: p \vee s, \:\: s \rightarrow t, \:\: \sim t \}$ {\bf $\vdash $} $r$
 %% pagina 118 -- 16 convertendo as comparaçoes em letras


%\item $\{ p \vee q, \:\: q \rightarrow r, \:\: p \rightarrow s,  \:\: \sim s  \}$ {\bf $\vdash $} $r \wedge (p \vee q)$
 %% pagina 126 -- D


%%\item $\{ r \rightarrow t, \:\:  s \rightarrow q, \:\: t \vee q \rightarrow \sim p, \:\:  r \vee s \}$ {\bf $\vdash $} $ \sim p$
 %% pagina 126 -- m


%\item $\{ p \wedge q, p \rightarrow r,  r \wedge s \rightarrow \sim t,  q \rightarrow s   \}$ {\bf $\vdash $} $\sim t$
 %% pagina 121--  letra m

%\item $\{ p \wedge \sim q, r \rightarrow q,  r \vee s,  p \vee s \rightarrow t   \}$ {\bf $\vdash $} $ t $
 %% pagina 122--  letra e
 
%%\item $\{ p, p \rightarrow q, p \wedge q \leftrightarrow t \vee s, \sim s \} \vdash t$
  %% criação: prof. Rogerio
  
\end{enumerate}

%\textcolor{red}{Vamo fazer algo totalmente NOVO}

 \item Utilizando o m\'etodo de {\bf demonstra\c c\~ao condicional}, demonstre a validade das consequ\^encias abaixo: 
 %\textbf{\textcolor{red}{NAO MEXI NESTAS DAQUI }}
 
\begin{enumerate}

\item $\{ p \wedge \sim q \rightarrow s, \sim (s \vee u), q \rightarrow r \} ~\vdash~ p \rightarrow q \wedge r$
% retirado de http://www.dma.uem.br/jrgeronimo/fundamentos/logica_word.pdf pag 19

%\item $\{(p \rightarrow q) \vee r, ~ (s \vee t) \rightarrow ~ \sim r,   ~ s \vee (t \wedge u) \} ~\vdash~ p \rightarrow q $ 
 % pagina 154 -- 3E

\item $\{ \sim p \rightarrow (q \rightarrow r), s \vee (r \rightarrow t), p \rightarrow s \} ~ \vdash ~ \sim s \rightarrow (q \rightarrow t)$
% retirado de http://jeiks.net/wp-content/uploads/2013/10/LogMat_Slide-14.pdf slide 6

%\item $\{(p \rightarrow q) \wedge \sim(r \wedge \sim s), ~ s \rightarrow ~  (t \vee u), ~ \sim u \} ~\vdash~ r \rightarrow t $ 
 % pagina 154 -- 3F
 

\item $\{ \sim p \vee q, \sim q, \sim r \rightarrow s, \sim p \rightarrow (s \rightarrow \sim t) \} ~ \vdash ~ t \rightarrow r$
% retirado de http://jeiks.net/wp-content/uploads/2013/10/LogMat_Slide-14.pdf slide 9

%\item $\{(p \vee \sim q),~ q,  ~ r \rightarrow ~ \sim s,~
%p \rightarrow (\sim s \rightarrow t) \} ~\vdash~ \sim t \rightarrow \sim r $ 
 % pagina 154 -- 3G
 
 
%\item $\{p \wedge q \rightarrow \sim r,\:\: ~ r \vee (s \wedge t),\:\: p \leftrightarrow ~ q \} ~\vdash~  p \rightarrow s $ 
 % pagina 134 -- 11

%\item $ \{p \vee (q \rightarrow r), \sim r \} \vdash q \rightarrow p$
% fonte: http://jeiks.net/wp-content/uploads/2013/10/LogMat_Slide-14.pdf (slide 5)

%\item $ \{\sim p \rightarrow (q \rightarrow r), s \vee (r \rightarrow t), p \rightarrow s \} \vdash \sim s \rightarrow (q \rightarrow t)$
% fonte: http://jeiks.net/wp-content/uploads/2013/10/LogMat_Slide-14.pdf (slide 6)

%\item $\{r \rightarrow t, ~ t \rightarrow ~ \sim s,   ~ (r \rightarrow \sim s) \rightarrow q \} ~\vdash~  p \rightarrow (p \wedge q) $ 
 % pagina 153 -- 1c


%\item $\{p \rightarrow q, \:\: q \leftrightarrow ~ s, \:\: t \vee (r \wedge \sim s) \} ~\vdash~ p \rightarrow t $ 
  % pagina 148 -- 5

%\item $\{r \vee s, ~ \sim t \rightarrow ~ \sim p, ~ r \rightarrow \sim q \} ~\vdash~ \sim (p \wedge q) \rightarrow (s \wedge t) $ 
  % pagina 153 -- 1I

%\item $\{ q \rightarrow p,\:\: t \vee s,\:\: q \vee\sim s \} ~\vdash~ \sim (p \vee r) \rightarrow t$
% pagina 153 -- 1k


\end{enumerate}


\item  Demonstrar que o conjunto das proposi\c c\~oes abaixo geram uma contradi\c c\~ao ({\bf demons\-tra\c c\~ao por absurdo ou indireta}),  (isto \'e, derivam uma inconsist\^encia do tipo: ($\Box \Leftrightarrow (\sim x \wedge x)$)
%Escolha duas provas para fazer das 3 que seguem  abaixo:


%\textbf{\textcolor{red}{Confira .... com as paginas do livro}}
\begin{enumerate}


%\item $\{ ( p \rightarrow q), ~ q \leftrightarrow s,~ t\vee (r \wedge \sim s) \} ~\vdash~  p \rightarrow t $
%% Pagina 155 6 B 



%\item $\{ \sim (p \rightarrow q) \vee ( s  \rightarrow ~ \sim r),~ q\vee s,~ p \rightarrow \sim s  \} ~\vdash~  \sim r \vee \sim s $
%% Pagina 155 6 D

%\item $\{\ \sim (p\:\: \rightarrow \:\: \sim q) \rightarrow ((r \:\: \leftrightarrow \:\: s) \vee t),\:\:  p,\:\:  q,\:\:  \sim t    ,\:\:  r \} ~\vdash~    s $
%% Pagina 155 6 G MODIFICADA

\item $\{\ (p \wedge q) \leftrightarrow \sim r, \:\:  \sim r \rightarrow \sim p, \:\: \sim q \rightarrow \sim r   \} ~\vdash~   q $
% pagina 154 -- 4I 

\item $\{\ \sim p \vee \sim  q,\:\:  r \vee s \rightarrow p,\:\:  q \vee \sim s,\:\: \sim r  \} ~\vdash~   \sim (r \vee s) $
% pagina 154 -- 4J 


\item $\{\ (p \rightarrow q) \vee r,\:\:  s \vee t \rightarrow \sim r,\:\:  s \vee (t \wedge u)  \} ~\vdash~   p \rightarrow q $
% pagina 155 -- 4L


%\item $\{\ (p \rightarrow q) \wedge r,\:\: q \vee s \rightarrow t \wedge u,\:\: v \rightarrow s,\:\: v \vee p \} ~\vdash~  t \vee x $
% pagina 147 -- 3 convertendo as comparaçoes em letras


%\item $\{\ \} ~\vdash~  $

%\item $\{\sim (p \rightarrow q) \vee (s \rightarrow\sim r),~ q \vee s,~ p \rightarrow\sim s \} ~\vdash~ \sim r \vee\sim s$
% pagina 155 -- 6d 

%\item $\{\sim p \rightarrow \sim q \vee r,~ s \vee (r \rightarrow t),~ p \rightarrow s,~ \sim s \} ~\vdash~ q \rightarrow t$
% pagina 155 -- 6c

%\item $\{ \sim p \vee \sim q,~ r \vee s \rightarrow p,~ q \vee\sim s,~ \sim r ~\vdash~ \sim(r \vee s)$
% pagina 154 -- 4I

%\item $\{ \sim r \vee \sim s, \:\: q \rightarrow s ~\vdash~ r \rightarrow \sim q$
% pagina 153 -- 1a TRIVIAL !!!

%{\bf\textcolor{red}{Claudio: falta alterar os 3 itens da questão 3. O resto já está OK}}




%\item $\{ p \vee q \rightarrow r,~ s \rightarrow \sim r \wedge \sim t,~ s \vee u,\:\: p   \} ~\vdash~ p \rightarrow u$
%pagina 153 -- 1l  MODIFICADA

%\item $\{ p \rightarrow q,~ r \rightarrow t,~ s \rightarrow r,~ p \vee s ,\:\: \sim q  \} ~\vdash~  t $
% pagina 153 -- 1m


\end{enumerate}


%\item (0.4 pts) Você passou metade do semestre estudando a relação de implicação $A \Rightarrow B$, validando-a
%por pelo menos dois métodos: demonstrando que $X/A \Rightarrow X/B$ e  $A \rightarrow B \equiv \blacksquare $. 
%Assim, explique qual relação de $A \Rightarrow B$ com a definição de teorema lógico $A \vdash B$? Em outras palavras, qual a definição conceitual de teorema lógico?
%Exemplifique se for o caso.


\end{enumerate}


\newpage
\underline{{\Large Equival\^encias Not\'aveis}}:
\begin{description}
\setlength{\itemsep}{-4pt}

\item[Idempot\^encia (ID):] $p\Leftrightarrow p\wedge p$ ou $p\Leftrightarrow p\vee p$
\item[Comuta\c c\~ao (COM):] $p\wedge q\Leftrightarrow q\wedge p$ ou $p\vee q\Leftrightarrow q\vee p$
\item[Associa\c c\~ao (ASSOC):] $p\wedge(q\wedge r)\Leftrightarrow (p\wedge q)\wedge r$ ou $p\vee(q\vee r)\Leftrightarrow (p\vee q)\vee r$ 
\item[Distribui\c c\~ao (DIST):] $p\wedge(q\vee r)\Leftrightarrow (p\wedge q)\vee (p \wedge r)$ ou $p\vee(q\wedge r)\Leftrightarrow (p\vee q)\wedge (p\vee r)$
\item[Dupla Nega\c c\~ao (DN):] $p\Leftrightarrow\sim\sim p$
\item[De Morgan (DM):] $\sim(p \wedge q) \Leftrightarrow \sim p \vee\sim q$ ou $\sim(p \vee q) \Leftrightarrow \sim p \wedge\sim q$
\item[Equival\^encia da Condicional (COND):] $p\rightarrow q \Leftrightarrow\sim p \vee q$

\item[Bicondicional (BICOND):] $p\leftrightarrow q \Leftrightarrow (p\rightarrow q)\wedge(q\rightarrow p)$

\item[Contraposi\c c\~ao (CP):] $p\rightarrow q \Leftrightarrow \sim q\rightarrow\sim p$

\item[Exporta\c c\~ao-Importa\c c\~ao (EI):] $p\wedge q\rightarrow r \Leftrightarrow p\rightarrow(q\rightarrow r)$

\item[Contradi\c c\~ao:] $p\wedge \sim p \Leftrightarrow \square $

\item[Tautologia:] $ p\vee \sim p \Leftrightarrow \blacksquare    $

\item [Absor\c c\~ao:] $\begin{array}{l}p \wedge (p \vee q) \Leftrightarrow p\\p \vee (p \wedge q) \Leftrightarrow p\end{array}$

\end{description}

\underline{{\Large Regras Infer\^encias V\'alidas (Teoremas)}}:
\begin{description}
\setlength{\itemsep}{-4pt}
\item[Adi\c c\~ao (AD):] $p \vdash p \vee q$ ou $p \vdash q \vee p$
\item[Simplifica\c c\~ao (SIMP):] $p \wedge q \vdash p$ ou $p \wedge q \vdash q$
\item[Conjun\c c\~ao (CONJ)] $p,~ q \vdash p \wedge q$ ou $p, q \vdash q \wedge p$
\item[Absor\c c\~ao (ABS):] $p \rightarrow q \vdash p \rightarrow (p \wedge q)$
\item[Modus Ponens (MP):] $p \rightarrow q,~ p \vdash q$
\item[Modus Tollens (MT):] $p \rightarrow q,~ \sim q \vdash \sim p$
\item[Silogismo Disjuntivo (SD):] $p \vee q,~ \sim p \vdash q$ ou $p \vee q,~ \sim q \vdash p$
\item[Silogismo Hipot\'etico (SH):] $p \rightarrow q,~ q\rightarrow r \vdash p\rightarrow r$
\item[Dilema Construtivo (DC):] $p\rightarrow q,~ r\rightarrow s,~ p \vee r \vdash q\vee s$
\item[Dilema Destrutivo (DD):] $p\rightarrow q,~ r\rightarrow s,~ \sim q\vee\sim s \vdash \sim p \vee\sim r$
\end{description}

%\end{enumerate}

\begin{flushleft}
\underline{Observa\c c\~oes}:
\begin{enumerate}
\setlength{\itemsep}{-2pt}
\item Qualquer d\'uvida, desenvolva a quest\~ao e deixe tudo
explicado, detalhadamente,
 que avaliaremos o seu conhecimentos sobre
 o assunto;\item \underline{Clareza e legibilidade};

\end{enumerate}
\end{flushleft}
\end{document}
