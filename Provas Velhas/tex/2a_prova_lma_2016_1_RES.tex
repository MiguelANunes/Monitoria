\documentclass[12pt, a4paper,final]{article}
\usepackage{t1enc}
\usepackage[utf8]{inputenc}
\usepackage[portuges,brazilian]{babel}

\usepackage{amsmath}
\usepackage{amsfonts}
\usepackage{amssymb}
\usepackage{comment,color, fancybox} %%% 
%\usepackage{tikz}
%\usepackage{comment, color } %%% inclui

%%%\usepackage{graphicx,url}
\topmargin       0cm 
\headheight      0pt 
\headsep         0cm
\textheight      24cm
\textwidth       16.7cm
\oddsidemargin   -2mm
\evensidemargin  -2mm
\pagestyle{empty}

%%%\graphicspath{{/figures/}}   
%%\DeclareGraphicsExtensions{{.jpg},{.png}}
%%https://logicproblems.org/problems/

\begin{document}


\begin{large}
\begin{center}

\shadowbox{
\begin{minipage}[c]{10cm}
\begin{center}
\sf
$2^{\underline{a}}$ Avalia\c c\~ao de L\'ogica Matem\'atica  (LMA)\\
Professores: Claudio ($T_B$) e Rogério  ($T_A$) \\
Joinville, \today
\end{center}
\end{minipage}
} %% 

\end{center}
\end{large} 

%{\bf\textcolor{red}{Rogerio ... amanha pegamos novas questoes do livro}}

\begin{flushright}
``É capaz quem pensa que é capaz.''\\{\footnotesize  Buda}
\end{flushright}


\vskip 0.5cm Acad\^emico(a) : \rule{10cm}{0.4pt} Turma:  \rule{1cm}{0.5pt}
\noindent

\begin{enumerate}
%\setlength{\itemsep}{-1pt}

\itemsep 1cm

\item Verificar a \textbf{validade dos argumentos} (dedu\c c\~ao natural) que se seguem: %(escolha duas das 3 para desenvolver):

\begin{enumerate}

\item $\{ p \vee \sim q,~ \sim p,~ \sim (p \wedge r) \rightarrow q \} ~\vdash~ r$
%% pagina 110 -- 1h

\item $\{ \sim (p \vee q),~ \sim p \wedge \sim q \rightarrow r \wedge s,~ s \rightarrow t \} ~\vdash~ r \wedge t$
%% pagina 111 -- 4b modificado

\item $\{ p \vee q,~ q \rightarrow r,~ \sim r \vee s,~ \sim p \} ~\vdash~ s$
%% pagina 110 -- 2c

%\item $\{ p \rightarrow q, \:\: p \vee (\sim \sim r \wedge \sim \sim  q), s \rightarrow \sim r, \sim (p \wedge q) \}$ {\bf $\vdash $} $\sim (s \wedge q)$
 %% pagina 116 -- 11

%\item $\{ p \rightarrow q, \:\: \sim r \rightarrow (s \rightarrow t),\:\: r \vee (p \vee s), \:\: \sim r \}$ {\bf $\vdash $} $q \vee t$
 %% pagina 115 -- 9


%\item $\{ p \rightarrow q, \:\: q \rightarrow r, \:\: r \rightarrow s, \:\: \sim s, \:\: p \vee t \}$ {\bf $\vdash $} $t$
 %% pagina 117 -- 13



%\item $\{ p \rightarrow q, \:\: q \rightarrow r, \:\: p \vee s, \:\: s \rightarrow t, \:\: \sim t \}$ {\bf $\vdash $} $r$
 %% pagina 118 -- 16 convertendo as comparaçoes em letras


%\item $\{ p \vee q, \:\: q \rightarrow r, \:\: p \rightarrow s,  \:\: \sim s  \}$ {\bf $\vdash $} $r \wedge (p \vee q)$
 %% pagina 126 -- D


%\item $\{ r \rightarrow t, \:\:  s \rightarrow q, \:\: t \vee q \rightarrow \sim p, \:\:  r \vee s \}$ {\bf $\vdash $} $ \sim p$
 %% pagina 126 -- m


%\item $\{ p \wedge q, p \rightarrow r,  r \wedge s \rightarrow \sim t,  q \rightarrow s   \}$ {\bf $\vdash $} $\sim t$
 %% pagina 121--  letra m

%\item $\{ p \wedge \sim q, r \rightarrow q,  r \vee s,  p \vee s \rightarrow t   \}$ {\bf $\vdash $} $ t $
 %% pagina 122--  letra e

\end{enumerate}

%\textcolor{red}{Vamo fazer algo totalmente NOVO}

 \item Utilizando o m\'etodo de {\bf demonstra\c c\~ao condicional}, demonstre a validade das consequ\^encias abaixo:
 
\begin{enumerate}

\item $\{p \wedge q \rightarrow \sim r,\:\: ~ r \vee (s \wedge t),\:\: p \leftrightarrow ~ q \} ~\vdash~  p \rightarrow s $ 
 % pagina 134 -- 11

\item $\{ p \vee q \rightarrow r,~ s \rightarrow \sim r \wedge \sim t,~ s \vee u \} ~\vdash~ p \rightarrow u$
%pagina 153 -- 1l

\item $\{ p \rightarrow q,~ r \rightarrow t,~ s \rightarrow r,~ p \vee s \} ~\vdash~ \sim q \rightarrow t $
% pagina 153 -- 1m




%\item $\{r \rightarrow t, ~ t \rightarrow ~ \sim s,   ~ (r \rightarrow \sim s) \rightarrow q \} ~\vdash~  p \rightarrow (p \wedge q) $ 
 % pagina 153 -- 1c


%\item $\{p \rightarrow q, \:\: q \leftrightarrow ~ s, \:\: t \vee (r \wedge \sim s) \} ~\vdash~ p \rightarrow t $ 
  % pagina 148 -- 5

%\item $\{r \vee s, ~ \sim t \rightarrow ~ \sim p,
%   ~ r \rightarrow \sim q \} ~\vdash~ \sim (p \wedge q) \rightarrow (s \wedge t) $ 
  % pagina 153 -- 1I

%\item $\{(p \rightarrow q) \vee r, ~ s \vee t \rightarrow ~ \sim r,   ~ s \vee (t \wedge u) \} ~\vdash~ p \rightarrow q $ 
 % pagina 154 -- 3E

%\item $\{ q \rightarrow p,\:\: t \vee s,\:\: q \vee\sim s \} ~\vdash~ \sim (p \vee r) \rightarrow t$
% pagina 153 -- 1k






\end{enumerate}



\item Demonstrar que o conjunto das proposi\c c\~oes abaixo geram uma contradi\c c\~ao ({\bf demons\-tra\c c\~ao por absurdo ou indireta}),  (isto \'e, derivam uma inconsist\^encia do tipo: ($\Box \Leftrightarrow (\sim x \wedge x)$)
%Escolha duas provas para fazer das 3 que seguem  abaixo:

\begin{enumerate}

%\item $\{\ (p \wedge q) \leftrightarrow \sim r, \:\:  \sim r \rightarrow \sim p, \:\: \sim q \rightarrow \sim r   \} ~\vdash~   q $
% pagina 154 -- 4I 

%\item $\{\ (p \rightarrow q) \wedge r,\:\: q \vee s \rightarrow t \wedge u,\:\: v \rightarrow s,\:\: v \vee p \} ~\vdash~  t \vee x $
% pagina 147 -- 3 convertendo as comparaçoes em letras


%\item $\{\ \} ~\vdash~  $

%\item $\{\sim (p \rightarrow q) \vee (s \rightarrow\sim r),~ q \vee s,~ p \rightarrow\sim s \} ~\vdash~ \sim r \vee\sim s$
% pagina 155 -- 6d 

%\item $\{\sim p \rightarrow \sim q \vee r,~ s \vee (r \rightarrow t),~ p \rightarrow s,~ \sim s \} ~\vdash~ q \rightarrow t$
% pagina 155 -- 6c

%\item $\{ \sim p \vee \sim q,~ r \vee s \rightarrow p,~ q \vee\sim s,~ \sim r ~\vdash~ \sim(r \vee s)$
% pagina 154 -- 4I

%\item $\{ \sim r \vee \sim s, \:\: q \rightarrow s ~\vdash~ r \rightarrow \sim q$
% pagina 153 -- 1a TRIVIAL !!!


\item $\{\ \sim (p\:\: \rightarrow \:\: \sim q) \rightarrow ((r \:\: \leftrightarrow \:\: s) \vee t),\:\:  p,\:\:  q,\:\:  \sim t    ,\:\:  r \} ~\vdash~    s $
%% Pagina 155 6 g MODIFICADA



\item $\{ p \vee q \rightarrow r,~ s \rightarrow \sim r \wedge \sim t,~ s \vee u   \} ~\vdash~ p \rightarrow u$
%pagina 153 -- 1l  MODIFICADA

\item $\{ p \rightarrow q,~ r \rightarrow t,~ s \rightarrow r,~ p \vee s ,\:\: \sim q  \} ~\vdash~  t $
% pagina 153 -- 1m


\end{enumerate}

\end{enumerate}


\newpage
\underline{{\Large Equival\^encias Not\'aveis}}:
\begin{description}
\setlength{\itemsep}{-4pt}

\item[Idempot\^encia (ID):] $P\Leftrightarrow P\wedge P$ ou $P\Leftrightarrow P\vee P$
\item[Comuta\c c\~ao (COM):] $P\wedge Q\Leftrightarrow Q\wedge P$ ou $P\vee Q\Leftrightarrow Q\vee P$
\item[Associa\c c\~ao (ASSOC):] $P\wedge(Q\wedge R)\Leftrightarrow (P\wedge Q)\wedge R$ ou $P\vee(Q\vee R)\Leftrightarrow (P\vee Q)\vee R$ 
\item[Distribui\c c\~ao (DIST):] $P\wedge(Q\vee R)\Leftrightarrow (P\wedge Q)\vee (P \wedge R)$ ou $P\vee(Q\wedge R)\Leftrightarrow (P\vee Q)\wedge (P\vee R)$
\item[Dupla Nega\c c\~ao (DN):] $P\Leftrightarrow\sim\sim P$
\item[De Morgan (DM):] $\sim(P \wedge Q) \Leftrightarrow \sim P \vee\sim Q$ ou $\sim(P \vee Q) \Leftrightarrow \sim P \wedge\sim Q$
\item[Equival\^encia da Condicional (COND):] $P\rightarrow Q \Leftrightarrow\sim P \vee Q$

\item[Bicondicional (BICOND):] $P\leftrightarrow Q \Leftrightarrow (P\rightarrow Q)\wedge(Q\rightarrow P)$

\item[Contraposi\c c\~ao (CP):] $P\rightarrow Q \Leftrightarrow \sim Q\rightarrow\sim P$

\item[Exporta\c c\~ao-Importa\c c\~ao (EI):] $P\wedge Q\rightarrow R \Leftrightarrow P\rightarrow(Q\rightarrow R)$

\item[Contradi\c c\~ao:] $P\wedge \sim P \Leftrightarrow \square $

\item[Tautologia:] $ P\vee \sim P \Leftrightarrow \blacksquare    $

\item [Absorção:] $\begin{array}{l}p \wedge (p \vee q) \Leftrightarrow p\\p \vee (p \wedge q) \Leftrightarrow p\end{array}$

\end{description}

\underline{{\Large Regras Infer\^encias V\'alidas (Teoremas)}}:
\begin{description}
\setlength{\itemsep}{-4pt}
\item[Adi\c c\~ao (AD):] $P \vdash P \vee Q$ ou $P \vdash Q \vee P$
\item[Simplifica\c c\~ao (SIMP):] $P \wedge Q \vdash P$ ou $P \wedge Q \vdash Q$
\item[Conjun\c c\~ao (CONJ)] $P, Q \vdash P \wedge Q$ ou $P, Q \vdash Q \wedge P$
\item[Absor\c c\~ao (ABS):] $P \rightarrow Q \vdash P \rightarrow (P \wedge Q)$
\item[Modus Ponens (MP):] $P \rightarrow Q, P \vdash Q$
\item[Modus Tollens (MT):] $P \rightarrow Q, \sim Q \vdash \sim P$
\item[Silogismo Disjuntivo (SD):] $P \vee Q, \sim P \vdash Q$ ou $P \vee Q, \sim Q \vdash P$
\item[Silogismo Hipot\'etico (SH):] $P \rightarrow Q, Q\rightarrow R \vdash P\rightarrow R$
\item[Dilema Construtivo (DC):] $P\rightarrow Q, R\rightarrow S, P \vee R \vdash Q\vee S$
\item[Dilema Destrutivo (DD):] $P\rightarrow Q, R\rightarrow S, \sim Q\vee\sim S \vdash \sim P \vee\sim R$
\end{description}

%\end{enumerate}

\begin{flushleft}
\underline{Observa\c c\~oes}:
\begin{enumerate}
\setlength{\itemsep}{-2pt}
\item Qualquer d\'uvida, desenvolva a quest\~ao e deixe tudo
explicado, detalhadamente,
 que avaliaremos o seu conhecimentos sobre
 o assunto;\item \underline{Clareza e legibilidade};

\end{enumerate}
\end{flushleft}
\end{document}
