\documentclass[12pt]{article}
\usepackage[a4paper,left=27mm,right=27mm,top=10mm,bottom=15mm]{geometry}
\usepackage{graphicx,url}
\usepackage{comment,color, fancybox} %%% 
\usepackage{amssymb}
\usepackage[utf8]{inputenc}
\usepackage[brazilian]{babel}
\usepackage[T1]{fontenc}


\graphicspath{{/figures/}}   
\DeclareGraphicsExtensions{{.jpg},{.png}}


\begin{document}


\begin{large}
\begin{center}

\shadowbox{
\begin{minipage}[c]{10cm}
\begin{center}
{\sf $1^{\underline{a}}$ Avaliação de Lógica Matemática  (LMA)}\\
Professores: Claudio ($T_A$) e Rogério  ($T_B$) \\
Joinville, \today
\end{center}
\end{minipage}
} %% 
\end{center}
\end{large} 
%\author{Rogério Eduardo da Silva e Claudio Cesar de Sá}
%\date{\today}

\vskip 0.2cm Acad\^emico(a) : \rule{10cm}{0.5pt} Turma:  \rule{1cm}{0.5pt}

\pagestyle{empty}

\begin{flushright}
``\textit{Quando jovens, \/aprendemos.
Quando velhos,  \/entendemos.}''\\ (Albert Einstein)
\end{flushright}

\begin{enumerate}
\itemsep 15pt
\item (1.0 pt) Determinar por tabela-verdade se a fórmula abaixo é uma {\bf tautologia}, {\bf contradição} (ou insatisfatível) ou {\bf contingência} (ou satisfatível, ou consistente): 

\begin{enumerate}
\setlength{\itemsep}{-2pt}
\item $(p \wedge \sim q) \leftrightarrow  (p \vee q) $

\item $p \wedge q \rightarrow ~~(q \oplus p)$

\item $ (\sim q \vee \sim p) \wedge \sim (p \wedge q \rightarrow p) $ 

\item $(\sim p \wedge q) \leftrightarrow ( p \vee \sim q)$ 

\end{enumerate}

\item (0.5 pts) Desenhe a estrutura hierárquica de cada um das fórmulas acima.
\textcolor{red}{Achei isto interessante nos slide e passei para as duas turmas como fazer isto !}

\item (3.0 pts) Determine as formas normais \underline{mais simples} (FNC e FND) equivalentes para as fórmulas abaixo: 
\textcolor{red}{inclui o r em duas formulas pequenas abaixo}
\begin{enumerate}
\setlength{\itemsep}{-2pt}

%\item $((p \wedge q) \vee (r \wedge s)) \vee (\sim q \wedge(p \vee t)) $ %%% 
%\textcolor{red}{peguei de um site}
\item $(p \rightarrow q)  \wedge 
 (\sim p \wedge r) $
% inclui o r

\item $(\sim p \wedge \sim q) \rightarrow (\sim p \rightarrow  q) \vee  ( p \rightarrow  \sim q)$

\item $(\sim r \vee \sim q) \leftrightarrow p $
% pag. 86 exerc. 7k

%%\item $(\sim p \vee \sim q) \rightarrow (p \wedge  q) $

\end{enumerate}

\item (0.5 pt) Das 06 fórmulas
encontradas no item anterior, escolha duas, uma 
FNC ($\mathcal{P}_1$) e sua respectiva FND ($\mathcal{Q}_1$). Obviamente que: $\mathcal{P}_1 \Leftrightarrow \mathcal{Q}_1$. 
Encontre as suas respectivas duais, $\mathcal{P}_2$ e $\mathcal{Q}_2$,  tal que obviamente  $\mathcal{P}_2 \Leftrightarrow \mathcal{Q}_2$. Qual o significado de fórmulas duais?


\item (3.0 pts) Utilizando as propriedades e equivalências
fornecidas na página seguinte
e verifique  se essas fórmulas apresentam uma relaç\~ao de implicaç\~ao lógica  verdadeira:

\textcolor{red}{tem 4 ... deixaria as 4}

\begin{enumerate}
\setlength{\itemsep}{-2pt}

%\item $q \Rightarrow p \wedge q \leftrightarrow q$

%\item  $ (p \vee q) \wedge \sim q \Rightarrow p $

%\item $q \Rightarrow p \wedge q \leftrightarrow p$
% pag. 54 exerc. 2b

%\item $(p \wedge q) \Rightarrow (p \vee q)$
%% pagina 80 Ex 7

%\item $(p \vee q) \Rightarrow (p \wedge q)$
%% um falso da anterior...

\item $(p \vee q) \wedge \sim q \Rightarrow p$
% adaptado do exerc. 5 pag. 54

\item $(p \rightarrow q) \Rightarrow p \wedge r \rightarrow q $
%% pagina 80 Ex 10

\item $(p \rightarrow q)  \Rightarrow ((q \rightarrow r) \rightarrow (p \rightarrow r)) $
\textcolor{red}{ fiz em sala e é válida  }
%% RETIRADO DE http://www.cs.odu.edu/~cs381/cs381content/logic/prop_logic/implications/implication_proof.html

%\item $[(p \leftrightarrow q) \wedge (q \leftrightarrow r)] \Rightarrow (p \leftrightarrow r)$
%\textcolor{red}{DIFICIL ... prova passada}
%http://www.cs.odu.edu/~cs381/cs381content/logic/prop_logic/implications/implication_proof.html


%\item $(p \rightarrow q) \wedge (r \rightarrow s)  \Rightarrow (p \wedge r ) \rightarrow  (q \wedge s) $
%\textcolor{red}{nova .... legal esta }
%% RETIRADO DE http://www.cs.odu.edu/~cs381/cs381content/logic/prop_logic/implications/implication_proof.html

\item $((p \rightarrow q) \wedge (p \rightarrow \sim q)) \rightarrow \sim p \Rightarrow  \blacksquare$
%% MODIFICADO do exemplo 13 PAG 80

\end{enumerate}


\item (2.5 pts) Utilizando as propriedades e algumas equivalências
fornecidas na página seguinte, demonstre as equivalências:

\textcolor{red}{tem 4 ... }

\begin{enumerate}
\setlength{\itemsep}{-2pt}

%\item $p \rightarrow q \Leftrightarrow p \vee q \rightarrow  q$ %% Ex 12 da pag 80

\item  $(p \rightarrow q) \vee (p \rightarrow r) \Leftrightarrow p  \rightarrow  (q \vee r) $ %% Ex 16 da pag 80

%\item $P \uparrow Q \Leftrightarrow ((P\downarrow P)\downarrow (Q\downarrow Q)) \downarrow ((P\downarrow P)\downarrow(Q\downarrow Q))$

%\item $p \wedge q \rightarrow r \Leftrightarrow p \rightarrow (q \rightarrow r) $  (Regra da Exportação e Importação) %% Ex 14 da pag 80

%\item  $(p \rightarrow r) \wedge (q \rightarrow r) \Leftrightarrow (p \vee  q) \rightarrow r $ %% Ex 15 da pag 80

%\item $(p \rightarrow q) \rightarrow r \Leftrightarrow p \wedge \sim r \rightarrow \sim q $
% pag 64 exerc.2g

\item $(p \rightarrow r)  \vee (q \rightarrow s) \Leftrightarrow p \wedge q \rightarrow  r \vee s $  %% Ex 17 da pag 81

%\item $(p \rightarrow q) \wedge (p \rightarrow r) \Leftrightarrow p \rightarrow (q \wedge r)$
%% pagina 86 exercicio f

\item $((p \rightarrow (q \rightarrow r)) \wedge (p \rightarrow q) \wedge p)  \rightarrow r \Leftrightarrow \blacksquare  $
%% UM TEOREMA da pagina 113

\item $((p \rightarrow q) \wedge (p \wedge r)) \rightarrow q \Leftrightarrow \blacksquare  $
%% UM TEOREMA da pagina 113 -- OK

\end{enumerate}


\newpage

\underline{{\large Equivalências Notáveis}}:

{\small
\begin{description}
\setlength{\itemsep}{-1pt}

\item[Idempotência (ID):] $P \Leftrightarrow P\wedge P$ ou $P \Leftrightarrow P\vee P$

\item[Comutação (COM):] $P \wedge Q \Leftrightarrow Q \wedge P$ ou $P\vee Q \Leftrightarrow Q \vee P$

\item[Associação (ASSOC):] $P \wedge(Q \wedge R)\Leftrightarrow (P\wedge Q) \wedge R$ ou $P \vee (Q\vee R) \Leftrightarrow (P\vee Q) \vee R$ 

\item[Distribuição (DIST):] $P\wedge(Q\vee R )\Leftrightarrow (P\wedge Q)\vee (P \wedge R)$ ou $P\vee(Q\wedge R)\:\: \Leftrightarrow (P\vee Q)\wedge (P\vee R)$

\item[Dupla Negação (DN):] $P\Leftrightarrow \:\:  \sim\sim P$

\item[De Morgan (DM):] $\sim(P \wedge Q) \Leftrightarrow  \:\: \sim P \vee\sim Q$ ou $\sim(P \vee Q) \Leftrightarrow \:\:  \sim P \wedge\sim Q$

\item[Equivalência da Condicional (COND):] $P\rightarrow Q \Leftrightarrow\sim P \vee Q$

\item[Bicondicional (BICOND):] $P\leftrightarrow Q \Leftrightarrow (P\rightarrow Q)\wedge(Q\rightarrow P)$

\item[Contraposição (CP):] $P\rightarrow Q \Leftrightarrow \sim Q\rightarrow\sim P$

\item[Exportação-Importação (EI):] $P\wedge Q\rightarrow R \Leftrightarrow P\rightarrow(Q\rightarrow R)$

\item[Contradição:] $P \wedge \sim P \Leftrightarrow \square $

\item[Tautologia:] $ P \vee \sim P \Leftrightarrow \blacksquare    $

\item[Ou-exclusivo:] $ P \oplus Q \Leftrightarrow  \sim (P \leftrightarrow  Q) $

\item[Conectivo de Sheffer (\textit{Not-And}):] $ P \uparrow Q \Leftrightarrow  \:\: \sim (P \wedge Q) $

\item[Conectivo de  Sheffer (\textit{Not-Or}):] $ P \downarrow Q \Leftrightarrow \:\: \sim (P \vee  Q) $

%\item[Negações para LPO:] $ \sim \forall px \Leftrightarrow \exists \sim px $

%\item[Negações para LPO:] $ \sim \exists px \Leftrightarrow \forall \sim px $

\end{description}

\underline{{\large Regras de Inferências Válidas (Teoremas)}}:

\begin{description}
\setlength{\itemsep}{-1pt}
\item[Adição (AD):] $P \vdash P \vee Q$ ou $P \vdash Q \vee P$
\item[Simplificação (SIMP):] $P \wedge Q \vdash P$ ou $P \wedge Q \vdash Q$
\item[Conjunção (CONJ)] $P, Q \vdash P \wedge Q$ ou $P, Q \vdash Q \wedge P$
\item[Absorção (ABS):] $P \rightarrow Q \vdash P \rightarrow (P \wedge Q)$
\item[Modus Ponens (MP):] $P \rightarrow Q, P \vdash Q$
\item[Modus Tollens (MT):] $P \rightarrow Q, \sim Q \vdash \sim P$
\item[Silogismo Disjuntivo (SD):] $P \vee Q, \sim P \vdash Q$ ou $P \vee Q, \sim Q \vdash P$
\item[Silogismo Hipotético (SH):] $P \rightarrow Q, Q\rightarrow R \vdash P\rightarrow R$
\item[Dilema Construtivo (DC):] $P\rightarrow Q, R\rightarrow S, P \vee R \vdash Q\vee S$
\item[Dilema Destrutivo (DD):] $P\rightarrow Q, R\rightarrow S, \sim Q\vee\sim S \vdash \sim P \vee\sim R$
\end{description}
%\end{enumerate}

\begin{flushleft}
\underline{Observações}:
\begin{enumerate}
\setlength{\itemsep}{-2pt}

\item Nas questões 4 e 5, não é para usar a TV (apenas para verificação se for o caso)

\item Qualquer dúvida, desenvolva a questão e deixe tudo
explicado, detalhadamente, que avaliaremos o seu conhecimentos sobre
 o assunto;

 \item \underline{Clareza e legibilidade};

\end{enumerate}
\end{flushleft}
}

\end{enumerate}


\end{document}