\documentclass[a4paper,11pt]{article}
\usepackage[T1]{fontenc}
\usepackage[utf8]{inputenc} %% isto garante compatibilidade com seu MAC
%\usepackage{lmodern}
\usepackage[brazil]{babel}
\usepackage{comment,color, fancybox} %%% 
\usepackage{graphicx, url}
\usepackage{amsmath}
\usepackage{amsfonts}
\usepackage{amssymb}
%%%\usepackage[normalem]{ulem}

\topmargin       0.1cm
\headheight      0pt
\headsep         -0.7cm
\textheight      25cm
\textwidth       16.7cm
\oddsidemargin   -5mm
\evensidemargin  -5mm
\baselineskip    -13pt

\begin{document}
%\framebox[15cm][c]{$3^a$ Avaliação de Lógica Matemática  (LMA) - Joinville, \today}

\begin{large}
\begin{center}

\shadowbox{
\begin{minipage}[c]{10cm}
\begin{center}
\sf
$3^{\underline{a}}$ Avalia\c c\~ao de L\'ogica Matem\'atica  (LMA)\\
Professores: Rogério  ($T_A$) e  Claudio ($T_B$)\\
Joinville, \today
\end{center}
\end{minipage}
} %% 
\end{center}
\end{large} 
%\author{Rogério Eduardo da Silva e Claudio Cesar de Sá}
%\date{\today}

\vskip 0.2cm Acad\^emico(a) : \rule{10cm}{0.7pt} Turma:  \rule{1cm}{0.7pt}
%%%\noindent Algumas questões desta prova vieram de \url{http://www.cs.utsa.edu/~bylander/cs2233/index.html}

%{\bf Atenção: Exame Final dia 07/12 (4a. feira às 17:00 hrs. -- Sala F101)}
{\bf Atenção: Exame Final dia 06/07 (4a. feira às 17:00 hrs. -- Sala F101)}

\begin{enumerate}
\setlength{\itemsep}{13pt}

%{\Huge Rogerio pegar exercicios daqui }
%\url{http://cnx.org/contents/46af3ad0-8636-4653-b64a-ea5f934df9d6@28/Exercises_for_First-Order_Logi}

\item {\bf (1.0 pt.)} Determine o valor verdade $\{V, F \}$ (a interpretação $\Phi $)
de cada uma das fórmulas abaixo em seu respectivo domínio. Dados: $A = \{ 3,  5 \}$, $B = \{ -15, 1, 15\}$ e  $C = \{ 6,  7 \}$.
As questões serão \textbf{apenas} validadas mediante os cálculos em separado.
 Em seguida preencha a tabela abaixo:

\begin{center}
\begin{tabular}{l|c|c} \hline \hline
 & \multicolumn{2}{c}{Domínios} \\ \hline \hline
 & $x \in A ~e~ y \in C$ & $x \in B ~e~ y \in A$  \\ \hline

%$\forall x: ~ (7+x \leq x^2)$ & & --xxx-- & --xxx--  & \\ \hline
%% $\exists x \exists y ((2+x)^2 \geq 24 - y)$ &  --xxx-- & & & --xxx-- \\ \hline
%% $\forall x (x^2 \geq 5)$ & & --xxx-- & --xxx-  & \\ \hline
$\exists y \forall x: ~ (2x+y \neq y^2)$ &  &  \\ \hline
$\forall x \exists y: ~ (xy + 2 \leq 40)$ &   & \\ \hline \hline
\end{tabular}
\end{center}

%\textcolor{red}{Rogerio ... aqui teria que dar uma pensada melhor ... rever as formula propostas}

\item {\bf (1.0 pts.)} Ao contrário  do que você fez na questão anterior, seja o conjunto $0 \leq \mathbb{N}\leq 20$ dos números naturais. Determine o(s) conjunto(s)-verdade ou domínio(s) para o qual a fórmula é \textbf{verdadeira}, para cada  uma das fórmulas abaixo (em resumo, calcule $D_x$ e $D_y$ conforme o caso):

Exemplo: $\forall x: ~ ((2x = 6) \vee (2x = 8)) $,  aqui o valor da resposta é $D=\{3, 4\}$

\begin{enumerate}
\itemsep -2pt   
 %   \item $\forall x . (x^2 - 5x + 6 = 0)$
 %       \item $\exists x . (x^2 - 5x + 6 = 0)$
  % \item $\exists x . (x^2 - 3x = 0)$ 
%  \item $\exists x . ((2x = 6) \vee (2x = 8)) $
  \item $\exists x \exists y: ~ ((2x = 12) \wedge (3y = 12)) $
%      \item $\forall x . ((x - 7) > 4 )$
      \item $\forall x: ~ ((x - 7) \le 4 )$
       
       %\item $\exists y \forall x . (x  \geq y)$

       \item $\exists y \forall x: ~ (x  \geq y + 10)$
       
%     \item $\exists x . \sim (\text{ x é ímpar})$
     \item $\exists x: ~ (\text{x é primo})$ \hspace{1cm} OBS: o $1$ não é primo!
     %%% \item REPENSAR AS FORMULAS.....
 \end{enumerate} 
\item \label{P1} {\bf (2.5 pts.)} Em contação de estória infantis, os tipos clássicos de personagens são chamados {\em arquétipos} e cada um apresenta sempre 
um comportamento padrão específico. Seja o conjunto das seguintes fórmulas em lógica de primeira-ordem (LPO):\\
\begin{tabular}{ll}
\\  \hline \hline
  % after \\: \hline or \cline{col1-col2} \cline{col3-col4} ...
  1. & $personagem(dragao, mau) $ \\
  2. & $personagem(cavaleiro, bom)$ \\
  3. & $personagem(princesa, bom)$ \\
  4. & $captura(dragao, princesa)$ \\
  5. & $armado(cavaleiro)$\\
  6. & $\forall x \exists y: personagem(x, bom) \wedge personagem(y, mau) \wedge ameacado(y) \rightarrow ataca(y, x)$ \\
  7. & $\forall x \exists y \exists z: (personagem(x, bom) \wedge personagem(z, bom) \wedge armado(x) \wedge captura(y, z) \rightarrow ameacado(y) )$ \\
    \hline \hline
 \end{tabular}

%Demonstre quais são os alunos que voltarão no próximo semestre.
\vskip 0.2cm
\underline{PROBLEMA}: Prove que o dragão irá atacar o cavaleiro.

\begin{comment}
\item  \label{P2} {\bf (2.0 pts.)} Na universidade de Berkeley há pré-requisitos entre algumas disciplinas.
Ou seja, \textit{toda disciplina $x$ que é um pré-requisito de alguma disciplina $y$, 
então $x$ deve preceder $y$}. Esta sequência
 de pré-requisitos eventualmente atrasa a graduação de alguns estudantes por lá. Fato similar ocorre por aqui!
 Assim, a situação desta grade-curricular é dada pelo 
 conjunto das seguintes fórmulas em lógica de primeira-ordem (LPO):
\\
  $$\begin{array}{llr} \hline\hline
	(1) & requisito(a, b) & (\text{leia-se: ``a é pré-requisito de b''}) \\
	(2) & requisito(a, c) &  \\
	(3) & requisito(a, d) &  \\
	(4) & requisito(b, e) &  \\
	(5) & requisito(c, e) & \\
			(6) & requisito(d, e) & \\
				(7) & requisito( e, f) & \\
	(8) & \forall x ~\exists y: requisito(x, y) \rightarrow precede(x, y) & \\
	(9) & \forall x ~\exists z ~\exists y: (requisito(x, z) \wedge precede(z, y)) \rightarrow precede(x,y) &\\
	\hline\hline
	\end{array}$$	
 Utilizando as propriedades da LPO, PU's, PE's e regras de inferências,  demonstre que a disciplina  \/`\textbf{a}' \/ deve preceder \/`\textbf{f}'.

\end{comment}

\item {\bf (1.0 pt.)} Dada as formulações
em LPO do problema anterior (questões \ref{P1}),  reescreva-os em Prolog ou Picat.
  
\newpage
\item {\bf (2.0 pts.)} Dado o código abaixo, indique a sua saída precisamente, após   a execução do \texttt{main}. 

%\textcolor{red}{JA MODIFICADA E IMPLEMENTADA ... NESTE DIRETORIO DE PROVAS}

\begin{minipage}{0.45\textwidth}
\begin{tiny}
\begin{center}{\bf PICAT}\end{center}
\begin{verbatim}

index(-)      % fatos instanciados como retorno
     f1(22).
     f1(24).
	
index(-)  % fatos instanciados como retorno
    f2(1).
    f2(2).
	
index(-)     % fatos instanciados como retorno
    f3(13).
    f3(15).
    
regra( X_1, Y_1, Z_1, Resp ) =>     
                      	f1(Z_1),
          	            f2(X_1),
          	            f3(Y_1),
          	            Resp = (X_1 + Y_1 + Z_1 ).

main ?=>     %%% this rule is  backtrackable
   regra(X,Y,Z, R),                            
   printf("\n X: %d \tY: %d \tZ: %d \tResp: %d", X,Y,Z, R)  ,  
   false.

main => 
       printf("\n\n FIM DOS FATOS \n\n") , true.
\end{verbatim}
\end{tiny}
\end{minipage}
\begin{minipage}{0.45\textwidth}
\begin{tiny}
\begin{center}{\bf PROLOG}\end{center}
\begin{verbatim}

    f1(22).
    f1(24).
	
    f2(1).
    f2(2).
	
    f3(13).
    f3(15).
    
regra( X_1, Y_1, Z_1, Resp ) :-     
      f1(Z_1),
      f2(X_1),
      f3(Y_1),
      Resp is (X_1 + Y_1 + Z_1 ).

main :-     %%% this rule is  backtrackable
   regra(X,Y,Z, R),                            
   writef("\n X: "), write(X),
   writef("\t Y: "), write(Y),
   writef("\t Z: "), write(Z), 
   writef("\t R: "), write(R), 
   false.

main :- 
   writef("\n\n FIM DOS FATOS \n\n") , true.
\end{verbatim}
\end{tiny}
\end{minipage}

\vskip 0.3cm

OBS.: O predicado ``\texttt{false / fail}''  é usado  apenas para forçar o PICAT/PROLOG retornarem todas as respostas usando o ({\em backtracking}). 


\item {\bf (2.5 pts.)} Implementar em PICAT (usando notação da programação em lógica) ou PROLOG  a funções recursivas dos 2 enunciados abaixo:
\begin{enumerate}
   \item Implemente o predicado do cálculo do máximo divisor comum entre dois números, definidos pela função recursiva abaixo:\\
  $ \operatorname{maior\_divisor}(x,y) =
  \begin{cases}
 x & \mbox{if } y = 0 \\
\operatorname{maior\_divisor}(y, \operatorname{resto\_divisao}(x,y)) & \mbox{se } y > 0 \\
 \end{cases}
$\\
dado: $\operatorname{resto\_divisao}(x,y) = mod(x,y)$

\item Idem para função que calcula o número de movimentos da torre de Hanoi:\\
$
\operatorname{hanoi}(n) =
 \begin{cases}
 1 & \mbox{if } n = 1 \\
 2\times\operatorname{hanoi}(n-1) + 1 & \mbox{if } n > 1\\
 \end{cases}
$ 
 
\end{enumerate}

\end{enumerate}
%\textcolor{red}{Removi as RIs ... etc ... nao sao mais necessarias neste estagio! colocamos no quadro o que precisarem}

\begin{comment}
  \item  Algo como ....:
  $$ 
  \operatorname{fact}(n) =
 \begin{cases}
  1 & \mbox{if } n = 0 \\
  n \cdot \operatorname{fact}(n-1) & \mbox{se } n > 0 \\
 \end{cases}
$$
   \end{comment}


\begin{comment}
\underline{{\large Equivalências Notáveis}}:
{\footnotesize
\begin{description}
\setlength{\itemsep}{-2pt}

\item[Idempotência (ID):] $P\Leftrightarrow P\wedge P$ ou $P\Leftrightarrow P\vee P$
\item[Comutação (COM):] $P\wedge Q\Leftrightarrow Q\wedge P$ ou $P\vee Q\Leftrightarrow Q\vee P$
\item[Associação (ASSOC):] $P\wedge(Q\wedge R)\Leftrightarrow (P\wedge Q)\wedge R$ ou $P\vee(Q\vee R)\Leftrightarrow (P\vee Q)\vee R$ 
\item[Distribuição (DIST):] $P\wedge(Q\vee R)\Leftrightarrow (P\wedge Q)\vee (P \wedge R)$ ou $P\vee(Q\wedge R)\Leftrightarrow (P\vee Q)\wedge (P\vee R)$
\item[Dupla Negação (DN):] $P\Leftrightarrow\sim\sim P$
\item[De Morgan (DM):] $\sim(P \wedge Q) \Leftrightarrow \sim P \vee\sim Q$ ou $\sim(P \vee Q) \Leftrightarrow \sim P \wedge\sim Q$
\item[Equivalência da Condicional (COND):] $P\rightarrow Q \Leftrightarrow\sim P \vee Q$

\item[Bicondicional (BICOND):] $P\leftrightarrow Q \Leftrightarrow (P\rightarrow Q)\wedge(Q\rightarrow P)$

\item[Contraposição (CP):] $P\rightarrow Q \Leftrightarrow \sim Q\rightarrow\sim P$

\item[Exportação-Importação (EI):] $P\wedge Q\rightarrow R \Leftrightarrow P\rightarrow(Q\rightarrow R)$

\item[Contradição:] $P\wedge \sim P \Leftrightarrow \square $

\item[Tautologia:] $ P\vee \sim P \Leftrightarrow \blacksquare    $

\item[Negações para LPO:] $ \sim \forall x: px \Leftrightarrow \exists x: \sim px $
\item[Negações para LPO:] $ \sim \exists x: px \Leftrightarrow \forall x: \sim px $
\end{description}
}

\underline{{\large Regras Inferencias Válidas (Teoremas)}}:
{\footnotesize
\begin{description}
\setlength{\itemsep}{-2pt}
\item[Adição (AD):] $P \vdash P \vee Q$ ou $P \vdash Q \vee P$
\item[Simplificação (SIMP):] $P \wedge Q \vdash P$ ou $P \wedge Q \vdash Q$
\item[Conjunção (CONJ)] $P, Q \vdash P \wedge Q$ ou $P, Q \vdash Q \wedge P$
\item[Absorção (ABS):] $P \rightarrow Q \vdash P \rightarrow (P \wedge Q)$
\item[Modus Ponens (MP):] $P \rightarrow Q, P \vdash Q$
\item[Modus Tollens (MT):] $P \rightarrow Q, \sim Q \vdash \sim P$
\item[Silogismo Disjuntivo (SD):] $P \vee Q, \sim P \vdash Q$ ou $P \vee Q, \sim Q \vdash P$
\item[Silogismo Hipotético (SH):] $P \rightarrow Q, Q\rightarrow R \vdash P\rightarrow R$
\item[Dilema Construtivo (DC):] $P\rightarrow Q, R\rightarrow S, P \vee R \vdash Q\vee S$
\item[Dilema Destrutivo (DD):] $P\rightarrow Q, R\rightarrow S, \sim Q\vee\sim S \vdash \sim P \vee\sim R$
\end{description}
%\end{enumerate}
\end{comment}

\begin{flushleft}
\underline{Observações}:
\begin{enumerate}
\setlength{\itemsep}{-2pt}
\item Qualquer dúvida, desenvolva a questão e deixe tudo
explicado, detalhadamente, que avaliaremos o seu conhecimentos sobre
 o assunto;
 \item \underline{Clareza e legibilidade};
\end{enumerate}
\end{flushleft}

\end{document}
