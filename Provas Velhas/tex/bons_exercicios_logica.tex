\documentclass[a4paper,12pt]{article}
\usepackage[T1]{fontenc}
\usepackage[utf8]{inputenc}
\usepackage{lmodern}
\usepackage[brazil]{babel}
\usepackage{comment, color} %%% 
\usepackage{graphicx, url}
\usepackage{amsmath}
\usepackage{amsfonts}
\usepackage{amssymb}
\usepackage{colortbl}
\usepackage{multirow}
%%%\usepackage[normalem]{ulem}

\topmargin       0.15cm
\headheight      0pt
\headsep         -0.5cm
\textheight      25cm
\textwidth       16.7cm
\oddsidemargin   -5mm
\evensidemargin  -5mm
\baselineskip    -13pt

\begin{document}
\framebox[15cm][c]{$3^a$ Avaliação de Lógica Matemática  (LMA) - Joinville, \today}

%\author{Rogério Eduardo da Silva e Claudio Cesar de Sá}
%\date{\today}

\vskip 0.5cm Acadêmico(a): \hrulefill%%%


\footnotesize{Obs.: Algumas questões desta prova vieram de \url{www.cs.utsa.edu/~bylander/cs2233/index.html}}

\begin{enumerate}
%\setlength{\itemsep}{-1pt}

\item {\bf Construa as duas fórmulas abaixo em suas respectivas FNC e FND:} %\footnote{alguns alunos ficaram com dúvidas neste importante tópico do curso}:
\begin{enumerate}
\setlength{\itemsep}{-3pt}
\item  $\sim (p\rightarrow \sim q)  \wedge (\sim q \rightarrow p) $
\item  $p \leftrightarrow q $
\end{enumerate}

\item {\bf Verificar a validade dos teoremas abaixo, usando um dos seguintes métodos de prova:  \textit{dedução natural}  (regras de inferência diretas e propriedades lógicas), ou pela \textit{contradição}, ou método da demonstração \textit{condicional} (escolha duas das três abaixo):}


\begin{enumerate}
\setlength{\itemsep}{-2pt} 
 \item $\{ p\rightarrow \sim q$, $\sim p \rightarrow (r \rightarrow \sim q)$, $(\sim s \vee \sim r)\rightarrow \sim \sim q$, $\sim s$ \} {\large $\vdash$} $\sim r$

\item $\{(\sim p\vee q) \rightarrow r$,  $(r \vee s)\rightarrow \sim t$, $t$ \} {\large $\vdash $} $\sim q$

\item $\{ p\rightarrow q$, $q \leftrightarrow s$, $t\vee (r\wedge \sim s)$ \} {\large $\vdash $} $p \rightarrow t$


%%\item  $\{ p\rightarrow \sim q$, $\sim p \rightarrow (r \rightarrow \sim q)$, $ (\sim s \vee \sim r)\rightarrow \sim \sim q$, $ \sim s\}$ $\vdash \sim r $

\end{enumerate}

%\begin{comment}
\item {\bf Considere cada uma das proposições atômicas abaixo:}
\begin{enumerate}
\setlength{\itemsep}{-2pt} 
\item $y < 0$
\item $y = 0$
\item $y > 0$
\item $x < y$
\item $x > y$
\item $x = y$
\end{enumerate}
Quais destas proposições deveriam ser escolhidas e combinadas, para
demostrar ou concluir a proposição $x < 0$? Faça suas escolhas e exiba esta demonstração.



\item {\bf Considere o jogo ``Pedra ($r$), Papel ($p$), Tesoura ($s$)''.
  Com dois jogadores usaremos as seis proposições seguintes:}
  %%Exercício resolvido por Daniel Camargo%%
  
\begin{table}[htb]
\begin{center}
\begin{tabular}{|l|l|l|}
\hline
Jogador 			    & Literais & Significado \\\hline
  \multirow{3}{*}{Jogador 1} & $r_1$ & Escolhe Pedra. \\
			     & $p_1$ & Escolhe Papel. \\
			     & $s_1$ & Escolhe Tesoura. \\\hline
  \multirow{3}{*}{Jogador 2} & $r_2$ & Escolhe Pedra. \\
			     & $p_2$ & Escolhe Papel. \\
			     & $s_2$ & Escolhe Tesoura.\\\hline
\end{tabular}
\end{center}
\end{table}

\begin{enumerate}
%\setlength{\itemsep}{-2pt} 
\item {\bf Expresse como uma proposição: \textit{``Cada jogador deve escolher ao menos uma opção de pedra, papel ou tesoura''}.}\\
 
 \textcolor{red}{Resposta}: O jogador deve escolher pelo menos um, dois ou três literais e não pode escolher nenhum, portanto a primeira linha da T.V. (em que os literais são falso) será Falso e as demais será Verdadeiro.
 \begin{table}[htb]
 \begin{center}
 \begin{minipage}{0.4\linewidth}
 \centering
 \begin{tabular}{|c|c|c|c|}
     \hline
  \rowcolor[rgb]{0.9,0.9,0.9} { $p_1$} & { $r_1$} & { $s_1$} & {\bf Expressão} \\ \hline
  \rowcolor[rgb]{1,0.2,0.2}  $F$   & $F$   & $F$   & $F$ \\ \hline
   $F$   & $F$   & $V$   & $V$ \\ \hline
   $F$   & $V$   & $F$   & $V$ \\ \hline
   $F$   & $V$   & $V$   & $V$ \\ \hline
   $V$   & $F$   & $F$   & $V$ \\ \hline
   $V$   & $F$   & $V$   & $V$ \\ \hline
   $V$   & $V$   & $F$   & $V$ \\ \hline
   $V$   & $V$   & $V$   & $V$ \\ \hline
 \end{tabular}
 \caption{Jogador 1}
 \label{4a1}
 \end{minipage}
 $\wedge$
 \begin{minipage}{0.4\linewidth}
 \centering
 \begin{tabular}{|c|c|c|c|}
     \hline
  \rowcolor[rgb]{0.9,0.9,0.9} { $p_2$} & { $r_2$} & { $s_2$} & {\bf Expressão} \\ \hline
  \rowcolor[rgb]{1,0.2,0.2}  $F$   & $F$   & $F$   & $F$ \\ \hline
   $F$   & $F$   & $V$   & $V$ \\ \hline
   $F$   & $V$   & $F$   & $V$ \\ \hline
   $F$   & $V$   & $V$   & $V$ \\ \hline
   $V$   & $F$   & $F$   & $V$ \\ \hline
   $V$   & $F$   & $V$   & $V$ \\ \hline
   $V$   & $V$   & $F$   & $V$ \\ \hline
   $V$   & $V$   & $V$   & $V$ \\ \hline
 \end{tabular}
 \caption{Jogador 2}
 \label{4a2}
 \end{minipage}
 \end{center}
 \end{table}
 
 
 \begin{itemize}
 \item No conceito de construir uma expressão através de uma TV, deve-se preferencialmente escolher trabalhar com a resposta da expressão que aparece o menor número de vezes.
 
 \item Ao escolhermos a linha em que a expressão é falsa, \textit{e.g.}, em \textcolor{red}{vermelho}, faremos uma relação de disjunção ($\vee$) forçando com que a resposta da Expressão seja Falso.
 
 \item Na tabela \ref{4a1} estamos considerando as opções do jogador 1, portanto a resposta será: $(p_1 \vee r_1 \vee s_1$). 
 
 \item Ao considerarmos a tabela \ref{4a2} do jogador 2, chegaremos à expressão $(p_2 \vee r_2 \vee s_2$). 
 
 \item A resposta final consiste na relação de conjunção entre as expressões dos dois jogadores, chegando à resposta final:
 
 \begin{center} \begin{Large}
  $(r_1 \vee p_1 \vee s_1) \wedge (r_2 \vee p_2 \vee s_2)$
  \end{Large}  \end{center}
  \item Este método de encontrar uma expressão através da manipulação da TV será utilizado para as demais questões.
 \end{itemize}
 
  
\item \label{quatrobee} {\bf Expresse como uma proposição: \textit{``Cada jogador não pode escolher mais de uma opção de pedra, papel ou tesoura''}.}\\
  
 \textcolor{red}{Resposta:} Cada jogador deve escolher no máximo um literal ou nenhum. Vejamos as tabelas ~\ref{4b1} e ~\ref{4b2}: 
 
 \begin{table}[htb]
 \begin{center}
 \begin{minipage}{0.4\linewidth}
 \centering
 \begin{tabular}{|c|c|c|c|}
  \hline
  \rowcolor[rgb]{0.9,0.9,0.9} { $p_1$} & { $r_1$} & { $s_1$} & {\bf Expressão} \\ \hline
  \rowcolor[rgb]{1,0.2,0.2} $F$   & $F$   & $F$   & $V$ \\ \hline
  \rowcolor[rgb]{1,0.2,0.2} $F$   & $F$   & $V$   & $V$ \\ \hline
  \rowcolor[rgb]{1,0.2,0.2} $F$   & $V$   & $F$   & $V$ \\ \hline
   $F$   & $V$   & $V$   & $F$ \\ \hline
  \rowcolor[rgb]{1,0.2,0.2} $V$   & $F$   & $F$   & $V$ \\ \hline
   $V$   & $F$   & $V$   & $F$ \\ \hline
   $V$   & $V$   & $F$   & $F$ \\ \hline
   $V$   & $V$   & $V$   & $F$ \\ \hline
 \end{tabular}
 \caption{Jogador 1}
 \label{4b1}
 \end{minipage}
 $\wedge$
 \begin{minipage}{0.4\linewidth}
 \centering
 \begin{tabular}{|c|c|c|c|}
  \hline
  \rowcolor[rgb]{0.9,0.9,0.9} { $p_2$} & { $r_2$} & { $s_2$} & {\bf Expressão} \\ \hline
  \rowcolor[rgb]{1,0.2,0.2} $F$   & $F$   & $F$   & $V$ \\ \hline
  \rowcolor[rgb]{1,0.2,0.2} $F$   & $F$   & $V$   & $V$ \\ \hline
  \rowcolor[rgb]{1,0.2,0.2} $F$   & $V$   & $F$   & $V$ \\ \hline
   $F$   & $V$   & $V$   & $F$ \\ \hline
  \rowcolor[rgb]{1,0.2,0.2} $V$   & $F$   & $F$   & $V$ \\ \hline
   $V$   & $F$   & $V$   & $F$ \\ \hline
   $V$   & $V$   & $F$   & $F$ \\ \hline
   $V$   & $V$   & $V$   & $F$ \\ \hline
 \end{tabular}
 \caption{Jogador 2}
 \label{4b2}
 \end{minipage}
 \end{center}
\end{table}

\begin{itemize}
   \item Para estas tabelas, encontra-se uma quantidade igual de linhas \textcolor{red}{Verdadeiras} e Falsas.
   \item Ao trabalhar com as linhas verdadeiras, deve-se fazer uma conjunção entre cada literal e uma disjunção entre cada linha da tabela.
   \item Para finalizar deve-se considerar a expressão do jogador 2, fazendo uma conjunção entre os dois jogadores.
\end{itemize}

 Jogador 1:

  $((\sim p_1 \wedge \sim r_1 \wedge \sim s_1) \vee$ Referente à 1a. linha da tabela. \\ 
  $ (\sim p_1 \wedge \sim r_1 \wedge s_1 ) \vee $ Referente à 2a. linha da tabela.\\
  $(\sim p_1 \wedge  r_1 \wedge \sim s_1 ) \vee $ Referente à 3a. linha da tabela. \\
  $ (p_1 \wedge \sim r_1 \wedge \sim s_1))$ Referente à 5a. linha da tabela.  \\

 Jogador 2:
  
  $((\sim p_2 \wedge \sim r_2 \wedge \sim s_2) \vee$ Referente à 1a. linha da tabela. \\ 
  $ (\sim p_2 \wedge \sim r_2 \wedge s_2 ) \vee $ Referente à 2a. linha da tabela.\\
  $(\sim p_2 \wedge  r_2 \wedge \sim s_2 ) \vee $ Referente à 3a. linha da tabela. \\
  $ (p_2 \wedge \sim r_2 \wedge \sim s_2))$ Referente à 5a. linha da tabela.  \\
  
  Resultando na expressão:
  \begin{center}

  $(((\sim p_1 \wedge \sim r_1 \wedge \sim s_1) \vee (\sim p_1 \wedge \sim r_1 \wedge s_1 ) \vee (\sim p_1 \wedge  r_1 \wedge \sim s_1 ) \vee (p_1 \wedge \sim r_1 \wedge \sim s_1))$ \\
  $\wedge$\\
  $((\sim p_2 \wedge \sim r_2 \wedge \sim s_2) \vee (\sim p_2 \wedge \sim r_2 \wedge s_2 ) \vee (\sim p_2 \wedge  r_2 \wedge \sim s_2 ) \vee (p_2 \wedge \sim r_2 \wedge \sim s_2)))$\\
  \end{center}
  Esta expressão pode ser melhorada utilizando a simplificação na tentativa de chegar à uma forma mais reduzida.
 
 
\item {\bf(100 pts., crédito extra) Expresse da maneira mais reduzida possível: \textit{``Cada jogador deve escolher exatamente  uma opção de pedra, papel ou tesoura''}.}\\

\textcolor{red}{Resposta:} Esta questão refere-se à operação de OU EXCLUSIVO (XOR). Vamos às tabelas:

\begin{table}[htb]
\begin{center}
\begin{minipage}[htb]{0.4\linewidth}
\centering
 \begin{tabular}{|c|c|c|c|}
  \hline	
  \rowcolor[rgb]{0.9,0.9,0.9} { $p_1$} & { $r_1$} & { $s_1$} & {\bf Expressão} \\ \hline
   $F$   & $F$   & $F$   & $F$ \\ \hline
  \rowcolor[rgb]{1,0.2,0.2} $F$   & $F$   & $V$   & $V$ \\ \hline
  \rowcolor[rgb]{1,0.2,0.2} $F$   & $V$   & $F$   & $V$ \\ \hline
   $F$   & $V$   & $V$   & $F$ \\ \hline
  \rowcolor[rgb]{1,0.2,0.2} $V$   & $F$   & $F$   & $V$ \\ \hline
   $V$   & $F$   & $V$   & $F$ \\ \hline
   $V$   & $V$   & $F$   & $F$ \\ \hline
   $V$   & $V$   & $V$   & $F$ \\ \hline
\end{tabular}
\caption{Jogador 1}
\label{4c1}
\end{minipage}
$\wedge$
\begin{minipage}[htb]{0.4\linewidth}
\centering
 \begin{tabular}{|c|c|c|c|}
  \hline	
  \rowcolor[rgb]{0.9,0.9,0.9} { $p_2$} & { $r_2$} & { $s_2$} & {\bf Expressão} \\ \hline
   $F$   & $F$   & $F$   & $F$ \\ \hline
  \rowcolor[rgb]{1,0.2,0.2} $F$   & $F$   & $V$   & $V$ \\ \hline
  \rowcolor[rgb]{1,0.2,0.2} $F$   & $V$   & $F$   & $V$ \\ \hline
   $F$   & $V$   & $V$   & $F$ \\ \hline
  \rowcolor[rgb]{1,0.2,0.2} $V$   & $F$   & $F$   & $V$ \\ \hline
   $V$   & $F$   & $V$   & $F$ \\ \hline
   $V$   & $V$   & $F$   & $F$ \\ \hline
   $V$   & $V$   & $V$   & $F$ \\ \hline
\end{tabular}
\caption{Jogador 2}
\label{4c2}
\end{minipage}
\end{center}
\end{table}


\begin{itemize}
   \item Observe que a relação entre os objetos significa que cada jogador pode escolher uma das três opções possíveis para cada jogador:
   \begin{center}
   $((~Tesoura1 ~\underline{\vee} ~Pedra1 ~\underline{\vee} ~Papel1) ~\wedge ~(~Tesoura2 ~\underline{\vee} ~Pedra2 ~\underline{\vee} ~Papel2))$
   \end{center} 
   
   \item Mas deve-se considerar que na tabela ~\ref{4c1}:
   \begin{itemize}
   \item A Tesoura1 será Verdade se $(\sim p_1 \wedge \sim r_1 \wedge s_1)$;
   \item A Pedra1 será Verdade se $(\sim p_1 \wedge r_1 \wedge \sim s_1)$;
   \item O Papel1 será verdade se $(~p_1 \wedge \sim r_1 \wedge \sim s_1)$;
   \end{itemize}
   \item E da mesma forma para o jogador 2 na tabela ~\ref{4c2}...
   \item Substituindo as considerações em cada objeto, chegamos à expressão:
\end{itemize}
\begin{center}
$(((\sim p_1 \wedge \sim r_1 \wedge ~s_1)~\underline{\vee} ~(\sim p_1 \wedge ~r_1 ~\wedge \sim s_1)~\underline{\vee} ~(~p_1 ~\wedge \sim r_1 \wedge \sim s_1))$\\
$\wedge$\\
$((\sim p_2 \wedge \sim r_2 \wedge ~s_2)~\underline{\vee} ~(\sim p_2 \wedge ~r_2 ~\wedge \sim s_2)~\underline{\vee} ~(~p_2 ~\wedge \sim r_2 \wedge \sim s_2)))$\\   
\end{center}


\item {\bf Expresse como uma proposição: \textit{``Os jogadores empatam''}.  Assuma que você não tenha que se importar com a regra anterior.}

\textcolor{red}{Resposta:} Nesta questão desconsideraremos ``não jogar uma opção'' (linha 1). Faremos uma relação de conjunção entre os jogadores e disjunção entre as opções, {\it i.e.}, faremos: \\
``$((tesoura_1 ~e ~tesoura_2) ~ou ~(pedra_1 ~e ~pedra_2) ~ou ~(papel_1 ~e ~papel_2))$''
\begin{table}[htb]
\begin{center}
\begin{minipage}[htb]{0.4\linewidth}
\centering
 \begin{tabular}{|c|c|c|c|}
  \hline	
  \rowcolor[rgb]{0.9,0.9,0.9} { $p_1$} & { $r_1$} & { $s_1$} & {\bf Expressão} \\ \hline
   $F$   & $F$   & $F$   & $F$ \\ \hline
  \rowcolor[rgb]{1,0.2,0.2} $F$   & $F$   & $V$   & $V$ \\ \hline
  \rowcolor[rgb]{1,0.2,0.2} $F$   & $V$   & $F$   & $V$ \\ \hline
   $F$   & $V$   & $V$   & $F$ \\ \hline
  \rowcolor[rgb]{1,0.2,0.2} $V$   & $F$   & $F$   & $V$ \\ \hline
   $V$   & $F$   & $V$   & $F$ \\ \hline
   $V$   & $V$   & $F$   & $F$ \\ \hline
   $V$   & $V$   & $V$   & $F$ \\ \hline
\end{tabular}
\caption{Jogador 1}
\label{4d1}
\end{minipage}
$\wedge$
\begin{minipage}[htb]{0.4\linewidth}
\centering
 \begin{tabular}{|c|c|c|c|}
  \hline	
  \rowcolor[rgb]{0.9,0.9,0.9} { $p_2$} & { $r_2$} & { $s_2$} & {\bf Expressão} \\ \hline
   $F$   & $F$   & $F$   & $F$ \\ \hline
  \rowcolor[rgb]{1,0.2,0.2} $F$   & $F$   & $V$   & $V$ \\ \hline
  \rowcolor[rgb]{1,0.2,0.2} $F$   & $V$   & $F$   & $V$ \\ \hline
   $F$   & $V$   & $V$   & $F$ \\ \hline
  \rowcolor[rgb]{1,0.2,0.2} $V$   & $F$   & $F$   & $V$ \\ \hline
   $V$   & $F$   & $V$   & $F$ \\ \hline
   $V$   & $V$   & $F$   & $F$ \\ \hline
   $V$   & $V$   & $V$   & $F$ \\ \hline
\end{tabular}
\caption{Jogador 2}
\label{4d2}
\end{minipage}
\end{center}
\end{table}
 
 \begin{itemize}
    \item Verifica-se que nas tabelas ~\ref{4d1} e ~\ref{4d2} temos 3 linhas cujo a expressão é verdadeira, portanto trabalharemos com elas.
    \item Ao chegar à expressão da opção do jogador 1, deve-se fazer a expressão da opção do jogador 2 (que tem a mesma forma).
    \item Una as três jogadas por uma disjunção.
 \end{itemize}
 
 \begin{center}
 $(((\sim p_1 \wedge \sim r_1 \wedge s_1) \wedge (\sim p_2 \wedge \sim r_2 \wedge s_2))$ = os dois jogam tesoura;\\
\parindent=29,2mm $\vee$ = ou \parindent=0mm\\
 $((\sim p_1 \wedge r_1 \wedge \sim s_1) \wedge (\sim p_2 \wedge r_2 \wedge \sim s_2))$ = os dois jogam pedra;\\
\parindent=29,2mm $\vee$ = ou \parindent=0mm\\
 $(( p_1 \wedge \sim r_1 \wedge \sim s_1) \wedge ( p_1 \wedge \sim r_1 \wedge \sim s_1)))$ = os dois jogam papel.\\
 \end{center}
Observe que a expressão a seguir não funciona pois há muitas verdades na tabela que não deveriam existir. $ (r_1 \wedge r_2) \vee (p_1 \wedge p_2) \vee (s_1 \wedge s_2)$


\item {\bf Expresse como uma proposição: \textit{``O jogador 1 ganha''}. Assuma que você não tenha que se preocupar com a regra em parte (c).} % you don't have to worry about the rule in part (c).

  \textcolor{red}{Resposta:} Considere as tabelas e as observações a seguir.
  
 \begin{table}[htb]
\begin{center}
\begin{minipage}[htb]{0.4\linewidth}
\centering
 \begin{tabular}{|c|c|c|c|}
  \hline	
  \rowcolor[rgb]{0.9,0.9,0.9} { $p_1$} & { $r_1$} & { $s_1$} & {\bf Expressão} \\ \hline
   $F$   & $F$   & $F$   & $F$ \\ \hline
  \rowcolor[rgb]{1,0.2,0.2} $F$   & $F$   & $V$   & $V$ \\ \hline
  \rowcolor[rgb]{1,0.2,0.2} $F$   & $V$   & $F$   & $V$ \\ \hline
   $F$   & $V$   & $V$   & $F$ \\ \hline
  \rowcolor[rgb]{1,0.2,0.2} $V$   & $F$   & $F$   & $V$ \\ \hline
   $V$   & $F$   & $V$   & $F$ \\ \hline
   $V$   & $V$   & $F$   & $F$ \\ \hline
   $V$   & $V$   & $V$   & $F$ \\ \hline
\end{tabular}
\caption{Jogador 1}
\label{4e1}
\end{minipage}
\begin{minipage}[htb]{0.4\linewidth}
\centering
 \begin{tabular}{|c|c|c|c|}
  \hline	
  \rowcolor[rgb]{0.9,0.9,0.9} { $p_2$} & { $r_2$} & { $s_2$} & {\bf Expressão} \\ \hline
   $F$   & $F$   & $F$   & $F$ \\ \hline
  \rowcolor[rgb]{1,0.2,0.2} $F$   & $F$   & $V$   & $V$ \\ \hline
  \rowcolor[rgb]{1,0.2,0.2} $F$   & $V$   & $F$   & $V$ \\ \hline
   $F$   & $V$   & $V$   & $F$ \\ \hline
  \rowcolor[rgb]{1,0.2,0.2} $V$   & $F$   & $F$   & $V$ \\ \hline
   $V$   & $F$   & $V$   & $F$ \\ \hline
   $V$   & $V$   & $F$   & $F$ \\ \hline
   $V$   & $V$   & $V$   & $F$ \\ \hline
\end{tabular}
\caption{Jogador 2}
\label{4e2}
\end{minipage}
\end{center}
\end{table}

\begin{itemize}
   \item Para o jogador 1 ganhar, deve seguir a regra do jogo, obedecendo a precedência das opções:
   \begin{enumerate}
      \item Tesoura ganha do Papel;
      \item Pedra ganha da Tesoura;
      \item Papel ganha da Pedra;
   \end{enumerate}
    \item Se desconsiderar a regra do item (c), não teremos que nos preocupar com o OU EXCLUSIVO (XOR).
    \item Com isso, chegamos à seguinte conclusão:
\end{itemize}
\begin{center}
   $(((\sim p_1 \wedge \sim r_1 \wedge s_1) \wedge (p_2 \wedge \sim r_2 \wedge \sim s_2))$ = regra i;\\
   $ \vee $ ou \\
   $((\sim p_1 \wedge r_1 \wedge \sim s_1) \wedge (\sim p_2 \wedge \sim r_2 \wedge s_2))$ = regra ii;\\
   $ \vee $ ou  \\
   $((p_1 \wedge \sim r_1 \wedge \sim s_1) \wedge (\sim p_2 \wedge r_2 \wedge \sim s_2)))$ = regra iii;\\
\end{center}


\end{enumerate}

\item {\bf Determine o valor verdade $\{V, F \}$ (a interpretação $\Phi $)
de cada uma das fórmulas abaixo, em seu respectivo domínio.
Faça os cálculos em separado e preencha a tabela abaixo.}
%(Determine the truth value of each statement for each domain.)

\begin{center}
\begin{tabular}{|c|l|l|l|l|} \hline
 & \multicolumn{4}{c|}{Domínios} \\ \hline
 & Num. Reais & Reais Positivos & Inteiros & Inteiros Positivos \\ \hline
$\exists x (x = -x)$ & & & & \\ \hline
$\forall x (2x \leq 3x)$ & & & & \\ \hline
$\exists x (x^2 = 2)$ & & & & \\ \hline
$\forall x (x \leq x^2)$ & & & & \\ \hline
$\forall x \exists y (xy = 1)$ & & & & \\ \hline
\end{tabular}
\end{center}


\item {\bf Seja o enunciado: ``{\em \ldots para todo caminho definido de $x$ até $z$ e arco entre $z$ e $y$, então há um caminho entre $x$ e $y$. Sabe-se que todo arco 
entre $x$ e $y$ é também um caminho entre $x$ e $y$}''. Sabe-se ainda que há arcos
definidos pelas fórmulas: $arco(a,b)$, $arco(a,c)$, $arco(b,d)$,  e  $arco(c,d)$. 
Prove que é possível ir de um ponto $a$ a $e$ definido por  um 
$caminho(a,e)$ como verdade. Desta vez vamos fornecer
a fórmulas referente ao texto acima, as quais são dadas por:}


\begin{enumerate}
\setlength{\itemsep}{-2pt} 
  \item  $\forall x \forall y \forall z ( caminho(x,z) \wedge arco(z,y) \rightarrow  caminho(x,y) )$
\item $\forall x \forall y ( arco(x,y)  \rightarrow  caminho(x,y) )$
\item   $arco(a,b)$
\item   $arco(a,c)$
\item   $arco(b,d)$
\item   $arco(c,d)$
\item   $arco(d,e)$
\end{enumerate}
{\bf Deduza tal caminho como verdade, indicando todas instâncias
das variáveis,  PU's, PE's e regras de inferências
utilizadas. Faça um grafo (flechas e nós) orientado para ver o que estás calculando.}

\end{enumerate}

\newpage 
%%%%%%%%%%%%%%%%%%%%%%%%%%%%%%%
\underline{{\Large Equivalências Notáveis}}:
\begin{description}
\setlength{\itemsep}{-4pt}

\item[Idempotência (ID):] $P\Leftrightarrow P\wedge P$ ou $P\Leftrightarrow P\vee P$
\item[Comutação (COM):] $P\wedge Q\Leftrightarrow Q\wedge P$ ou $P\vee Q\Leftrightarrow Q\vee P$
\item[Associação (ASSOC):] $P\wedge(Q\wedge R)\Leftrightarrow (P\wedge Q)\wedge R$ ou $P\vee(Q\vee R)\Leftrightarrow (P\vee Q)\vee R$ 
\item[Distribuição (DIST):] $P\wedge(Q\vee R)\Leftrightarrow (P\wedge Q)\vee (P \wedge R)$ ou $P\vee(Q\wedge R)\Leftrightarrow (P\vee Q)\wedge (P\vee R)$
\item[Dupla Negação (DN):] $P\Leftrightarrow\sim\sim P$
\item[De Morgan (DM):] $\sim(P \wedge Q) \Leftrightarrow \sim P \vee\sim Q$ ou $\sim(P \vee Q) \Leftrightarrow \sim P \wedge\sim Q$
\item[Equivalência da Condicional (COND):] $P\rightarrow Q \Leftrightarrow\sim P \vee Q$

\item[Bicondicional (BICOND):] $P\leftrightarrow Q \Leftrightarrow (P\rightarrow Q)\wedge(Q\rightarrow P)$

\item[Contraposição (CP):] $P\rightarrow Q \Leftrightarrow \sim Q\rightarrow\sim P$

\item[Exportação-Importação (EI):] $P\wedge Q\rightarrow R \Leftrightarrow P\rightarrow(Q\rightarrow R)$

\item[Contradição:] $P\wedge \sim P \Leftrightarrow \square $

\item[Tautologia:] $ P\vee \sim P \Leftrightarrow \blacksquare    $


\end{description}

\underline{{\Large Regras Inferencias Válidas (Teoremas)}}:
\begin{description}
\setlength{\itemsep}{-4pt}
\item[Adição (AD):] $P \vdash P \vee Q$ ou $P \vdash Q \vee P$
\item[Simplificação (SIMP):] $P \wedge Q \vdash P$ ou $P \wedge Q \vdash Q$
\item[Conjunção (CONJ)] $P, Q \vdash P \wedge Q$ ou $P, Q \vdash Q \wedge P$
\item[Absorção (ABS):] $P \rightarrow Q \vdash P \rightarrow (P \wedge Q)$
\item[Modus Ponens (MP):] $P \rightarrow Q, P \vdash Q$
\item[Modus Tollens (MT):] $P \rightarrow Q, \sim Q \vdash \sim P$
\item[Silogismo Disjuntivo (SD):] $P \vee Q, \sim P \vdash Q$ ou $P \vee Q, \sim Q \vdash P$
\item[Silogismo Hipotético (SH):] $P \rightarrow Q, Q\rightarrow R \vdash P\rightarrow R$
\item[Dilema Construtivo (DC):] $P\rightarrow Q, R\rightarrow S, P \vee R \vdash Q\vee S$
\item[Dilema Destrutivo (DD):] $P\rightarrow Q, R\rightarrow S, \sim Q\vee\sim S \vdash \sim P \vee\sim R$
\end{description}
%\end{enumerate}

\begin{flushleft}
\underline{Observações}:
\begin{enumerate}
\setlength{\itemsep}{-2pt}
\item Qualquer dúvida, desenvolva a questão e deixe tudo
explicado, detalhadamente, que avaliaremos o seu conhecimentos sobre
 o assunto;
 \item \underline{Clareza e legibilidade};

\end{enumerate}
\end{flushleft}

\vskip 2cm

\begin{flushright}
\textit{In formal logic, a contradiction is the signal of defeat, but in the evolution\\
of real knowledge it marks the first step in progress toward a victory.} 

Alfred North Whitehead
\end{flushright}
\end{document}
