
\documentclass[10pt, a4paper,final]{article}
\usepackage{t1enc}
\usepackage[utf8]{inputenc} %%% garante mactosh
\usepackage[portuges]{babel}
\usepackage{amsmath}
\usepackage{amsfonts}
\usepackage{amssymb}
\usepackage{comment, color} %%% 

%\usepackage{graphicx}
\topmargin       -1.5cm
\headheight      0pt
\headsep  0cm
\textheight      27cm
\textwidth       16.7cm
\oddsidemargin   -5mm
\evensidemargin  -5mm
\pagestyle{empty}


\begin{document}
\begin{center}
\begin{tabular}{||c||} \hline \hline 
{\Large Logica Matemática  (LMA)}  \\
\mbox{\hskip 2cm  UDESC/DCC -- \today  \hskip 2cm }
\\
Exame Final   \\ \hline \hline
\end{tabular}
\end{center}
%\vskip1cm 
\textbf{Aluno(a)/TURMA}: \hrulefill
%%%\noindent

\begin{flushright}
``{\em Educação é aquilo que fica depois que você esquece o que a escola ensinou.}''\\
Albert Einstein
\end{flushright}

%\textcolor{red}{Rogério ... a prova está bem carregada, mas bem DIMENSIONADA. Confira
%e corte o que achares excesso, e atribua a pontuação, pois está faltando.}

\begin{enumerate}
%\setlength{\itemsep}{-5pt}

\item {\bf (2.0 pt)} Determinar as Formas Normais Disjuntiva e Conjuntiva para:
\begin{description}
\setlength{\itemsep}{-5pt}
 \item [1.]  $\sim (p \rightarrow q) \wedge (q \rightarrow p) $
 \item [2.]  $(\sim p \rightarrow q) \vee (q \rightarrow \sim p) $
\end{description}
Qual o tipo dessas fórmulas? (consistentes, ... etc)

%% OK
\item {\bf (1.0 pt)} Efetuar a prova (ou demonstração) direta para validade dos argumentos
que se seguem: \\

 $\{p\rightarrow \sim q \: , \:\:\: \sim p \rightarrow (r \rightarrow \sim q)  \: , \:\:\:
 (\sim s \vee \sim r)\rightarrow \sim \sim q  \: , \:\:\: \sim s  \} \vdash  r $ \\
 
Caso esse conjunto não derive um teorema, que mudanças
você faria nas premissas para derivar $r$?

%% OK
\item {\bf (1.0 pt)} Utilizando o método de  {\em demonstração por absurdo} ou {\em indireta},
 demonstre a validade do   argumento $ \sim p \wedge \sim r $, a partir 
 das premissas: \\
1. $  p \rightarrow  q $ \\
2. $ q \rightarrow r $ \\
3. $ r \rightarrow p $ \\
4. $ p \rightarrow \sim r $ \\
Isto é, esta sequência deduz ( $\vdash $, consiste
de um teorema) $ \sim p \wedge \sim r$?

%%%%
\item {\bf (2.0 pt)} Aplicando a negação nas fórmulas abaixo, obtenha
 as novas  sentenças lógicas equivalentes:
\begin{enumerate}
\setlength{\itemsep}{-2pt}
 \item $ \forall x \exists y ((\sim p(x) \wedge \sim q(y)) \vee \exists z (r(z)))$
% \item $ \exists x \forall y (p(x) \vee \sim q(y))$
 \item $ \exists x \forall y ((p(x) \vee q(y)) \wedge \forall z (r(z)))$
% \item $ \forall x \exists y (\sim p(x) \vee \sim q(y))$
  \item  $ \exists x \forall y (p(y) \leftrightarrow q(y))$
  \item  $ \forall x \exists y (p(y) \leftrightarrow q(y))$
\end{enumerate}


\item {\bf (1.5 pts)} Considere o seguinte conjunto de f\'ormulas: 

\begin{tabular}{ll}
  % after \\: \hline or \cline{col1-col2} \cline{col3-col4} ...
1 &  $\forall x\forall y (q(x,y) \wedge r(y) \rightarrow p(y)) $ \\
2 &  $\forall x (q(x,x) \rightarrow p(x))  $ \\
3 &  $\forall x (s(x) \rightarrow q(x,x)) $ \\
4 &  $r(b)$ \\ 
5 &  $s(a)$ \\
6 &  $s(b)$ \\
\end{tabular}\\
Utilizando as propriedades da LPO, tais como PU, GU, GE e PE, detalhando passo-a-passo, verifique se há uma resposta para $\sim p(X)$ ou $p(X)$, 
consequentemente para $\sim p(a)$ ou $p(a)$ e $\sim p(b)$ ou $p(b)$.
O domínio é dado por $D=\{a,b\}$. 


\item {\bf (1.5 pts)} Seja o seguine texto: Sam, Clyde e Oscar são elefantes. N\'os sabemos os seguintes  fatos sobre eles:
\begin{itemize}
\setlength{\itemsep}{-2pt}
    \item Sam é rosa;
    \item Clyde é cinza e gosta de Oscar;
    \item Oscar é ou rosa ou cinza, mas não tem as duas cores, e gosta de Sam.
\end{itemize}
Usando as propriedades da LPO como prova, demonstre que um elefante cinza gosta de um elefante rosa. Construa passo-a-passo as expressões acima, e prove que: $\exists x \:\: \exists y \:\:(cinza(x) \wedge rosa(y) \wedge gosta(x,y))$


\item {\bf (1.0 pt)} Expresse {\bf em Prolog} o seguinte texto e resolva: 
``\textit{Tweety é um pássaro. Goldie é um peixe. Molie é uma minhoca. Pássaros gostam de minhocas. Gatos gostam de peixes. Gatos gostam de pássaros. Amigos gostam uns dos outros. O meu gato é meu amigo. O meu gato come tudo o que gosta. O meu gato chama-se Silvester.}''

\begin{enumerate}
  \item Escreva o texto acima em Prolog (era o enunciado);
  \item Use Prolog para determinar tudo o que come o Silvester?
  \item Da resposta do item anterior, a resposta é razoável ? Se não for, verifique se o problema está na especificação original ou na sua tradução para Prolog, corrija o seu programa e veja uma nova resposta para o que Silvester come.
\end{enumerate}




\end{enumerate}


\newpage
%%%%%%%%%%%%%%%%%%%%%%%%%%%%%%%
\underline{{\large Equivalências Notáveis}}:
\begin{description}
\setlength{\itemsep}{-2pt}

\item[Idempotência (ID):] $P\Leftrightarrow P\wedge P$ ou $P\Leftrightarrow P\vee P$
\item[Comutação (COM):] $P\wedge Q\Leftrightarrow Q\wedge P$ ou $P\vee Q\Leftrightarrow Q\vee P$
\item[Associação (ASSOC):] $P\wedge(Q\wedge R)\Leftrightarrow (P\wedge Q)\wedge R$ ou $P\vee(Q\vee R)\Leftrightarrow (P\vee Q)\vee R$ 
\item[Distribuição (DIST):] $P\wedge(Q\vee R)\Leftrightarrow (P\wedge Q)\vee (P \wedge R)$ ou $P\vee(Q\wedge R)\Leftrightarrow (P\vee Q)\wedge (P\vee R)$
\item[Dupla Negação (DN):] $P\Leftrightarrow\sim\sim P$
\item[De Morgan (DM):] $\sim(P \wedge Q) \Leftrightarrow \sim P \vee\sim Q$ ou $\sim(P \vee Q) \Leftrightarrow \sim P \wedge\sim Q$
\item[Equivalência da Condicional (COND):] $P\rightarrow Q \Leftrightarrow\sim P \vee Q$

\item[Bicondicional (BICOND):] $P\leftrightarrow Q \Leftrightarrow (P\rightarrow Q)\wedge(Q\rightarrow P)$

\item[Contraposição (CP):] $P\rightarrow Q \Leftrightarrow \sim Q\rightarrow\sim P$

\item[Exportação-Importação (EI):] $P\wedge Q\rightarrow R \Leftrightarrow P\rightarrow(Q\rightarrow R)$

\item[Contradição:] $P\wedge \sim P \Leftrightarrow \square $

\item[Tautologia:] $ P\vee \sim P \Leftrightarrow \blacksquare    $

\item[Negações para LPO:] $ \sim \forall px \Leftrightarrow \exists \sim px $

\item[Negações para LPO:] $ \sim \exists px \Leftrightarrow \forall \sim px $

\end{description}

\underline{{\large Regras Inferencias Válidas (Teoremas)}}:
\begin{description}
\setlength{\itemsep}{-2pt}
\item[Adição (AD):] $P \vdash P \vee Q$ ou $P \vdash Q \vee P$
\item[Simplificação (SIMP):] $P \wedge Q \vdash P$ ou $P \wedge Q \vdash Q$
\item[Conjunção (CONJ)] $P, Q \vdash P \wedge Q$ ou $P, Q \vdash Q \wedge P$
\item[Absorção (ABS):] $P \rightarrow Q \vdash P \rightarrow (P \wedge Q)$
\item[Modus Ponens (MP):] $P \rightarrow Q, P \vdash Q$
\item[Modus Tollens (MT):] $P \rightarrow Q, \sim Q \vdash \sim P$
\item[Silogismo Disjuntivo (SD):] $P \vee Q, \sim P \vdash Q$ ou $P \vee Q, \sim Q \vdash P$
\item[Silogismo Hipotético (SH):] $P \rightarrow Q, Q\rightarrow R \vdash P\rightarrow R$
\item[Dilema Construtivo (DC):] $P\rightarrow Q, R\rightarrow S, P \vee R \vdash Q\vee S$
\item[Dilema Destrutivo (DD):] $P\rightarrow Q, R\rightarrow S, \sim Q\vee\sim S \vdash \sim P \vee\sim R$
\end{description}
%\end{enumerate}

\begin{flushleft}
\underline{Observações}:
\begin{enumerate}
\setlength{\itemsep}{-2pt}
\item Qualquer dúvida, desenvolva a questão e deixe tudo
explicado, detalhadamente, que avaliaremos o seu conhecimentos sobre
 o assunto;
 \item \underline{Clareza e legibilidade};

\end{enumerate}
\end{flushleft}
\noindent Boas férias!

%% detalhamento.
%\noindent

\end{document}
