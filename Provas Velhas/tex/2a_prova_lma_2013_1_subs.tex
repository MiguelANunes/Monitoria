\documentclass[12pt, a4paper,final]{article}
\usepackage{t1enc}
\usepackage[latin1]{inputenc}
\usepackage[portuges,brazilian]{babel}

\usepackage{amsmath}
\usepackage{amsfonts}
\usepackage{amssymb}
\usepackage{comment,color} %%% 
%\usepackage{tikz}
%\usetikzlibrary{arrows}

%%%\usepackage{graphicx,url}
\topmargin       -1cm 
\headheight      0pt 
\headsep  0cm
\textheight      24cm
\textwidth       16.3cm
\oddsidemargin   2mm
\evensidemargin  2mm
\pagestyle{empty}

%%%\graphicspath{{/figures/}}   
%%\DeclareGraphicsExtensions{{.jpg},{.png}}


\begin{document}
\framebox[15cm][c]{$2^a$ Avalia\c c\~ao Substitutiva de L\'ogica Matem\'atica  (LMA) - Joinville, \today}

%\author{Rog�rio Eduardo da Silva e Claudio Cesar de S�}
%\date{\today}

\vskip 0.5cm Acad\^emico(a) : \rule{10cm}{0.4pt} Turma:  \rule{1cm}{0.4pt}
\noindent
\begin{enumerate}
\setlength{\itemsep}{-1pt}

\item Verificar a validade por dedu\c c\~ao natural os argumentos que se seguem (escolha duas para fazer das 3 abaixo):

\begin{enumerate}
\setlength{\itemsep}{-2pt} 

 \item $\{ p \vee q \rightarrow r \wedge s, \sim s \}$ {\bf $\vdash $} $\sim q$
 %% pagina 132 -- 3

\item $\{ \sim p \vee q, \sim q, \sim (q \wedge r) \rightarrow p \}$  {\bf $\vdash $} $r$
% pagina 141 -- 1g  = MUITO FACIL

\item $\{p \rightarrow q, q \leftrightarrow s, t \vee (r \wedge \sim s)\}$  {\bf $\vdash $} $p \rightarrow t$
% pagina 133 -- 8
\end{enumerate}

%\textcolor{red}{As demais estao iguais a do ano passado}

 \item Utilizando o m\'etodo de {\em demonstra\c c\~ao condicional}, demonstre a validade das consequ\^encias abaixo:
  
\begin{enumerate}
\setlength{\itemsep}{-4pt} 
\item
 \vskip 11pt
 \begin{tabular}{ll}  % after \\: \hline or \cline{col1-col2} \cline{col3-col4} ...    
 1 &  $  p \rightarrow \sim q$ \\    
 2 & $p \vee (r \wedge s) $ \\   
      $\vdash $ & Esta sequ\^encia deduz ( $\vdash $, consiste de um teorema) $ q \rightarrow s$
% pagina 144 - 3j
   
\end{tabular}
\item 
\vskip 11pt
\begin{tabular}{ll}  
    1 &  $ \sim p \vee \sim s $ \\   
    2 &  $ q \rightarrow \sim r$ \\   
     3 &  $ t \rightarrow s \wedge r$ \\ \hline
    $\vdash $  & Esta sequ\^encia deduz ( $\vdash $, consiste de um teorema) $ t \rightarrow \sim (p \vee q) $\end{tabular}
% pagina 153 -- 1g
\end{enumerate}

\item Demonstrar que o conjunto das proposi\c c\~oes abaixo geram uma contradi\c c\~ao, ou
 {\em demonstra\c c\~ao por absurdo},  (isto \'e,derivam uma inconsist\^encia do tipo: ($\Box \Leftrightarrow (\sim x \wedge x)$)
Escolha duas provas para fazer das 3 que seguem  abaixo:

\begin{enumerate}
\setlength{\itemsep}{-4pt}
\item 
\vskip 11pt
\begin{tabular}{ll}  
  1 &  $p \rightarrow \sim q $ \\  
  2 &  $ r \rightarrow \sim p$ \\ 
  3 &  $q \vee r$ \\ \hline
  $\vdash $  & $\sim p$   
  % pagina 154 -- 4b
\end{tabular}

\item 
\vskip 11pt
\begin{tabular}{ll}  % after \\: \hline or \cline{col1-col2} \cline{col3-col4} ...   
   1. &  $ p \rightarrow q \vee r $ \\   
    2. &   $ q \rightarrow \sim p$  \\
    3. & $s \rightarrow \sim r $ \\    
    $\vdash $  & $ \sim (p \wedge s)$ 
  % pagina 154 -- 4d
\end{tabular}

\item 
\vskip 11pt
\begin{tabular}{ll}  % after \\: \hline or \cline{col1-col2} \cline{col3-col4} ...   
 1. &  $(p \rightarrow q) \rightarrow r $ \\   
  2. &  $ r \vee s \rightarrow \sim t $ \\    
  3. &  $t$ \\   \hline
     $\vdash $ & $ \sim q $   
     \end{tabular}
% pagina 155 -- 4n
\end{enumerate}

\end{enumerate}

\newpage
\underline{{\Large Equival\^encias Not\'aveis}}:
\begin{description}
\setlength{\itemsep}{-4pt}

\item[Idempot\^encia (ID):] $P\Leftrightarrow P\wedge P$ ou $P\Leftrightarrow P\vee P$
\item[Comuta\c c\~ao (COM):] $P\wedge Q\Leftrightarrow Q\wedge P$ ou $P\vee Q\Leftrightarrow Q\vee P$
\item[Associa\c c\~ao (ASSOC):] $P\wedge(Q\wedge R)\Leftrightarrow (P\wedge Q)\wedge R$ ou $P\vee(Q\vee R)\Leftrightarrow (P\vee Q)\vee R$ 
\item[Distribui\c c\~ao (DIST):] $P\wedge(Q\vee R)\Leftrightarrow (P\wedge Q)\vee (P \wedge R)$ ou $P\vee(Q\wedge R)\Leftrightarrow (P\vee Q)\wedge (P\vee R)$
\item[Dupla Nega\c c\~ao (DN):] $P\Leftrightarrow\sim\sim P$
\item[De Morgan (DM):] $\sim(P \wedge Q) \Leftrightarrow \sim P \vee\sim Q$ ou $\sim(P \vee Q) \Leftrightarrow \sim P \wedge\sim Q$
\item[Equival\^encia da Condicional (COND):] $P\rightarrow Q \Leftrightarrow\sim P \vee Q$

\item[Bicondicional (BICOND):] $P\leftrightarrow Q \Leftrightarrow (P\rightarrow Q)\wedge(Q\rightarrow P)$

\item[Contraposi\c c\~ao (CP):] $P\rightarrow Q \Leftrightarrow \sim Q\rightarrow\sim P$

\item[Exporta\c c\~ao-Importa\c c\~ao (EI):] $P\wedge Q\rightarrow R \Leftrightarrow P\rightarrow(Q\rightarrow R)$

\item[Contradi\c c\~ao:] $P\wedge \sim P \Leftrightarrow \square $

\item[Tautologia:] $ P\vee \sim P \Leftrightarrow \blacksquare    $
\end{description}

\underline{{\Large Regras Infer\^encias V\'alidas (Teoremas)}}:
\begin{description}
\setlength{\itemsep}{-4pt}
\item[Adi\c c\~ao (AD):] $P \vdash P \vee Q$ ou $P \vdash Q \vee P$
\item[Simplifica\c c\~ao (SIMP):] $P \wedge Q \vdash P$ ou $P \wedge Q \vdash Q$
\item[Conjun\c c\~ao (CONJ)] $P, Q \vdash P \wedge Q$ ou $P, Q \vdash Q \wedge P$
\item[Absor\c c\~ao (ABS):] $P \rightarrow Q \vdash P \rightarrow (P \wedge Q)$
\item[Modus Ponens (MP):] $P \rightarrow Q, P \vdash Q$
\item[Modus Tollens (MT):] $P \rightarrow Q, \sim Q \vdash \sim P$
\item[Silogismo Disjuntivo (SD):] $P \vee Q, \sim P \vdash Q$ ou $P \vee Q, \sim Q \vdash P$
\item[Silogismo Hipot\'etico (SH):] $P \rightarrow Q, Q\rightarrow R \vdash P\rightarrow R$
\item[Dilema Construtivo (DC):] $P\rightarrow Q, R\rightarrow S, P \vee R \vdash Q\vee S$
\item[Dilema Destrutivo (DD):] $P\rightarrow Q, R\rightarrow S, \sim Q\vee\sim S \vdash \sim P \vee\sim R$
\end{description}

%\end{enumerate}

\begin{flushleft}
\underline{Observa\c c\~oes}:
\begin{enumerate}
\setlength{\itemsep}{-2pt}
\item Qualquer d\'uvida, desenvolva a quest\~ao e deixe tudo
explicado, detalhadamente,
 que avaliaremos o seu conhecimentos sobre
 o assunto;\item \underline{Clareza e legibilidade};

\end{enumerate}
\end{flushleft}\noindent\end{document}
