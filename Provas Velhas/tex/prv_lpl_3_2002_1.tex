
\documentclass[12pt, a4paper,final]{article}
\usepackage{t1enc}
\usepackage[latin1]{inputenc}
\usepackage[portuges]{babel}
\usepackage{amsmath}
\usepackage{amsfonts}
\usepackage{amssymb}

%\usepackage{graphicx}
\topmargin       -1cm
%\headheight      17pt
 \headsep  1cm

\textheight      24cm

\textwidth       16.7cm

\oddsidemargin   2mm

\evensidemargin  2mm

\pagestyle{empty}

\begin{document}
\begin{center}
\framebox[\textwidth][c]{L�gica e Programa��o em L�gica  (LPL)-
Joinville, \today}
%%\newline
\end{center}

\vskip1cm Aluno(a): \hrulefill
%%%\noindent

\begin{enumerate}
\setlength{\itemsep}{-5pt}
 \item (cap. 15) Considere um dom�nio $A =
 \{1,2,3,4,5,6,7,8,9\}$, e dada as seguintes
 f�rmulas:
\begin{enumerate}
\setlength{\itemsep}{-5pt}
 \item $p(x): x^2 \subseteq A$ e $q(x):$ x � �mpar.
 \item $r(x): x + 2 \leq 9 $ e $s(x):$ x � par.
\end{enumerate}
Calcule os valores verdades
(\emph{conjunto-verdade}, isto �, onde a
${\mathbf\Phi } ($f�rmula$) = V$)
 para seguintes f�rmulas:
\begin{enumerate}
\setlength{\itemsep}{-5pt}
 \item $p \rightarrow q$ e $q \vee p$
 \item $r \rightarrow s$ e $s \wedge r$
\end{enumerate}

\item (cap. 16) Determine e justifique (explique)
o valor l�gico das f�rmulas abaixo:
\begin{enumerate}
\setlength{\itemsep}{-3pt}
 \item $ \forall x (2^x > x^2)$ para $x \in N$
 \item $ \forall x (x^2 + 3x + 2 = 0)$ para $x \in R$
 \item $ \exists x (x+2 = x)$ para $x \in R$
 \item $ \exists x (3x^2 - 2x -1 = 0)$ para $x \in R$
\end{enumerate}

\item (cap. 16) Sendo $A = \{2,3,4,5,6,7,8,9\}$
encontrar o(s) valor(es) do contra-exemplo de:
\begin{enumerate}
\setlength{\itemsep}{-2pt}
 \item $ \forall x (2x + 5 > 12)$
 \item $ \forall x (3x^2 - 7 > 17)$
\end{enumerate}

\item (cap. 17) Dar a nega��o das seguintes
proposi��es:
\begin{enumerate}
\setlength{\itemsep}{-2pt}
 \item $ \forall x \exists y (\sim p(x) \wedge \sim q(y))$
 \item $ \exists x \forall y (p(x) \vee \sim q(y))$
 \item $ \exists x \forall y (p(x) \rightarrow q(y))$
 \item $ \forall x \exists y (\sim p(x) \vee \sim q(y))$
\end{enumerate}

\item Escreva as seguintes proposi��es sob a
nota��o da l�gica de primeira ordem (explique os
predicados assumidos):
\begin{enumerate}
\setlength{\itemsep}{-2pt}

\item ``{\em H� pessoas boas, mas nem todas
pessoas s�o boas}''. Assuma D=\{pessoas\};

\item ``{\em Se Judas � bom, ent�o nenhum homem �
bom}''. Assuma D=\{homem\};

\item ``{\em Todas as abelhas gostam de algumas
flores}''. Assuma D=\{conjunto das abelhas e das
flores\}'';

\item ``{\em Existem algumas abelhas que gostam
de todo tipo de flor}''. Idem.

\end{enumerate}

\item  Desenvolva as seguintes regras  em Prolog,
que implemente uma fun\c{c}\~{a}o recursiva,
definida por
\begin{center}
\vskip -10pt
$$
f(x)=\left\{
\begin{array}[l]{l}
x = 1 \mbox{  ent\~{a}o   } f(1)=5\\
x > 1 \mbox{  ent\~{a}o  } f(x) = f(x-1)\ast(x+7)
\end{array}
\right.
$$
\end{center}
\item Exemplifique os seguintes conceitos:
``\emph{casamento de padr�es}'' e ``{\em
backtracking}''.
\end{enumerate}
Observa��o: Clareza e legibilidade.
%\noindent

\end{document}
