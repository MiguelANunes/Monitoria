\documentclass[12pt]{article}
\usepackage[a4paper,left=30mm,right=30mm,top=25mm,bottom=19mm]{geometry}
\usepackage{graphicx,url}
\usepackage{titlesec}
\usepackage{amssymb}
\usepackage[utf8]{inputenc}
\usepackage[brazilian]{babel}
\usepackage[T1]{fontenc}

% Setting configuration for the text format
%\renewcommand{\contentsname}{Table of Contents}
%\renewcommand{\bibname}{References}
%\titleformat{\chapter}[display]{\normalfont\huge\bfseries}{\filleft\chaptername\ \thechapter}{5pt}{\filleft\Huge}
%\sloppy

\title{Lógica Matemática - Exame Final}
\author{Rogério Eduardo da Silva}
\date{\today}

\graphicspath{{/figures/}}   
\DeclareGraphicsExtensions{{.jpg},{.png}}


\begin{document}
\maketitle
\begin{flushright}
``\textit{A educação exige os maiores cuidados porque influi sobre toda a vida.}''\\ (Sêneca)
\end{flushright}

\small
Sobre Lógica Proposicional responda:
\begin{enumerate}
\item (1.0pt) Determinar por tabela-verdade se a fórmula abaixo é uma tautologia, contradição ou contingência: $(P \vee \sim R) \rightarrow (Q \wedge \sim R)$

\item (1.0pt/cada) Demonstre a validade dos seguintes argumentos através das regras de inferência e equivalência \begin{scriptsize}(indique a regra sendo aplicada a cada passo da prova)\end{scriptsize}:
\begin{enumerate}
\item $\{ P\rightarrow Q, Q \rightarrow R, R \rightarrow P, P \rightarrow\sim R \}\vdash \sim P \wedge\sim R$
\item $\{ T\rightarrow P\wedge S, Q\rightarrow\sim P, R \rightarrow\sim S, R \vee Q \}\vdash \sim T$
\item $\{ R\vee S, \sim T\rightarrow\sim P, R\rightarrow\sim Q \}\vdash P\wedge Q\rightarrow S\wedge T$
\begin{scriptsize}(use demonstração condicional)\end{scriptsize}
\item $\{ P \wedge Q \leftrightarrow\sim R, \sim R\rightarrow\sim P, \sim Q \rightarrow\sim R \}\vdash Q$ \begin{scriptsize}(use demonstração por absurdo)\end{scriptsize}
\end{enumerate}

\item (1.0pt) Determine a forma normal conjuntiva e disjuntiva equivalentes: $(P \uparrow Q)\leftrightarrow P $
\end{enumerate}

Sobre Lógica de Primeira Ordem responda:

\begin{enumerate}

\item (1.0pt/cada) Determine o valor verdade $\{V, F \}$ (a interpretação $\Phi $)
de cada uma das fórmulas abaixo, em seu respectivo domínio.
Faça os cálculos em separado e preencha a tabela abaixo.
%(Determine the truth value of each statement for each domain.)

\begin{center}
\begin{tabular}{l|l|l|l} \hline \hline
 & \multicolumn{3}{c}{Domínios} \\ \hline
 & Num. Reais & Inteiros Positivos & \{-5, -3.5, 0, 1, 1.1, 5.6, 10, 100.1\} \\ \hline
$\forall x (x^2 \geq 2x)$ & & & \\ \hline
$\forall x \exists y (x^2y^2 = 1)$ & &  & \\ \hline \hline
\end{tabular}
\end{center}

\item (2.0pts) Dado o enunciado abaixo, prove que $ama(john, garfield)$:

\begin{itemize}
\item $\forall x \exists y (animal(x) \wedge pessoa(y) \rightarrow ama(y, x) ) $
\item $\forall x ( mamifero(x) \vee ave(x) \vee peixe(x) \rightarrow animal(x) )$
\item $\forall x ( cao(x) \vee gato(x) \rightarrow mamifero(x) )$
\item $gato(garfield)$
\item $cao(oddie)$
\item $pessoa(john)$
\item $pessoa(lizzie)$
\end{itemize}

\end{enumerate}

\newpage

Argumentos válidos fundamentais:
\begin{description}
\item[Adição (AD)] $P \vdash P \vee Q$ ou $P \vdash Q \vee P$
\item[Simplificação (SIMP)] $P \wedge Q \vdash P$ ou $P \wedge Q \vdash Q$
\item[Conjunção (CONJ)] $P, Q \vdash P \wedge Q$ ou $P, Q \vdash Q \wedge P$
\item[Absorção (ABS)] $P \rightarrow Q \vdash P \rightarrow (P \wedge Q)$
\item[Modus Ponens (MP)] $P \rightarrow Q, P \vdash Q$
\item[Modus Tollens (MT)] $P \rightarrow Q, \sim Q \vdash \sim P$
\item[Silogismo Disjuntivo (SD)] $P \vee Q, \sim P \vdash Q$ ou $P \vee Q, \sim Q \vdash P$
\item[Silogismo Hipotético (SH)] $P \rightarrow Q, Q\rightarrow R \vdash P\rightarrow R$
\item[Dilema Construtivo (DC)] $P\rightarrow Q, R\rightarrow S, P \vee R \vdash Q\vee S$
\item[Dilema Destrutivo (DD)] $P\rightarrow Q, R\rightarrow S, \sim Q\vee\sim S \vdash \sim P \vee\sim R$
\end{description}

Equivalências Notáveis:
\begin{description}
\item[Idempotência (ID)] $P\Leftrightarrow P\wedge P$ ou $P\Leftrightarrow P\vee P$
\item[Comutação (COM)] $P\wedge Q\Leftrightarrow Q\wedge P$ ou $P\vee Q\Leftrightarrow Q\vee P$
\item[Associação (ASSOC)] $P\wedge(Q\wedge R)\Leftrightarrow (P\wedge Q)\wedge R$ ou $P\vee(Q\vee R)\Leftrightarrow (P\vee Q)\vee R$ 
\item[Distribuição (DIST)] $P\wedge(Q\vee R)\Leftrightarrow (P\wedge Q)\vee (P \wedge R)$ ou $P\vee(Q\wedge R)\Leftrightarrow (P\vee Q)\wedge (P\vee R)$
\item[Dupla Negação (DN)] $P\Leftrightarrow\sim\sim P$
\item[De Morgan (DM)] $\sim(P \wedge Q) \Leftrightarrow \sim P \vee\sim Q$ ou $\sim(P \vee Q) \Leftrightarrow \sim P \wedge\sim Q$
\item[Condicional (COND)] $P\rightarrow Q \Leftrightarrow\sim P \vee Q$
\item[Bicondicional (BICOND)] $P\leftrightarrow Q \Leftrightarrow (P\rightarrow Q)\wedge(Q\rightarrow P)$
\item[Contraposição (CP)] $P\rightarrow Q \Leftrightarrow \sim Q\rightarrow\sim P$
\item[Exportação-Importação (EI)] $P\wedge Q\rightarrow R \Leftrightarrow P\rightarrow(Q\rightarrow R)$
\end{description}

\end{document}

