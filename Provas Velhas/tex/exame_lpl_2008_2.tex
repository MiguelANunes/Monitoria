
\documentclass[12pt, a4paper,final]{article}
\usepackage{t1enc}
\usepackage[latin1]{inputenc}
\usepackage[portuges]{babel}
\usepackage{amsmath}
\usepackage{amsfonts}
\usepackage{amssymb}

%\usepackage{graphicx}
\topmargin       -1cm
%\headheight      17pt
\headsep  0cm
\textheight      25cm
\textwidth       16.7cm
\oddsidemargin   0mm
\evensidemargin  0mm
\pagestyle{empty}

\begin{document}
\begin{center}
\begin{tabular}{||c||}\hline \hline {\Large L�gica e Programa��o em L�gica}  \\
\mbox{\hskip 2cm  UDESC/DCC -- \today  \hskip 2cm }\\
Exame Final \\ \hline \hline
\end{tabular}
\end{center}

\vskip1cm Aluno(a): \hrulefill
%%%\noindent

\begin{enumerate}
\setlength{\itemsep}{-5pt}
%% OK
 \item Determinar
as seguintes Formas Normais:
\begin{description}
\setlength{\itemsep}{-5pt}
 \item [Disjuntiva] para: $(p \rightarrow q) \wedge \sim(q \rightarrow p) $
 \item [Conjuntiva] para: $\sim (\sim p \rightarrow q) \vee (q \rightarrow \sim p) $
\end{description}

%% OK
\item Efetuar a prova (ou demonstra��o) direta para validade dos argumentos
que se seguem:
\begin{enumerate}
\setlength{\itemsep}{-2pt}
 \item $\{p\rightarrow \sim q \: , \:\:\: \sim p \rightarrow (r \rightarrow \sim q)  \: , \:\:\:
 (\sim s \vee \sim r)\rightarrow \sim \sim q  \: , \:\:\: \sim s  \} \vDash \sim r $

 \item Verifique a argumenta��o abaixo, e valide
 o resultado: \vskip 12pt

\begin{tabular}{ll}
  % after \\: \hline or \cline{col1-col2} \cline{col3-col4} ...
  Se & $x = y$, ent�o $x = z$ \\
  Se & $x = z$, ent�o $x = t$ \\
  Ou &  $x = y$, ou $x = 0$ \\
  Se & $x = 0$, ent�o $x + u = 1$ \\
  Mas & $x + u \not= 1$ \\ \hline
  Portanto & $x = t$ \\
\end{tabular}
\end{enumerate}
Conclua algo sobre.

%% OK

\item Aplicando o m�todo da Resolu��o e 
considerando as premissas dos item anterior, demonstre que:
\begin{enumerate}
\setlength{\itemsep}{-2pt}
\item $\sim r $ � consequente l�gico do item a)
\item $x = t$ � consequente l�gico  do item b)
\end{enumerate}
Construa a �rvore de expans�o apresentado cada termo $\lambda$

 \item  Considerando um dom�nio $A =
 \{1,2,3,4,5,6,7,8,9\}$, e dado os seguintes
 predicados:
\begin{enumerate}
\setlength{\itemsep}{-2pt}
 \item $p(x): x^3 \subseteq A$ e $q(x):$ x � par.
 \item $r(x): x^2 + 3 \leq 9 $ e $s(x):$ x � �mpar.
\end{enumerate}
Calcule a validade das seguintes f�rmulas:
 (a) $\forall x (q(x) \rightarrow p(x))$ \hskip 0.7cm  (b) $\exists x (qx \vee
 px)$  \hskip 0.7cm
 (c) $\exists x (sx \rightarrow rx)$ \hskip 0.7cm (d) $\forall x (sx \wedge rx)$\\
 Simplificando a nota��o, leia-se, $q(x) = qx$ e assim sucessivamente aos demais predicados.


% \item Dar a nega��o das seguintes
% proposi��es:
% \begin{enumerate}
% \setlength{\itemsep}{-2pt}
% \item $ \sim \forall x \exists y (\sim p(x,y) \wedge \sim q(y,x))$
% \item $ \exists x \sim \forall y (p(x) \vee \sim q(y))$
% \item $ \sim \exists x \forall y (p(x) \rightarrow q(y))$
% \item $ \forall x \sim \exists y (\sim p(x) \vee \sim q(y))$
% \end{enumerate}

%%%%
\item Escreva o seguinte enunciado sob a
nota��o da l�gica de primeira ordem, em um
conjunto de premissas e uma conclus�o:
\begin{quote}
``{\em Todo estudante de CC trabalha mais do
que algu�m. Se todo mundo que trabalha mais
do que uma pessoa, ent�o dorme menos que
essa pessoa. Maria � estudante de CC.
Portanto, Maria dorme menos do que algu�m}''
\end{quote}
Utilize os seguintes predicados: 
\vspace {-13pt}
\begin{itemize}
\setlength{\itemsep}{-5pt}
    \item trab$(x,y)$
    \item dorme$(z,w)$
    \item estudante\_cc$(x)$, e ``maria''
\end{itemize}


\end{enumerate}
Observa��o: Clareza,  legibilidade, e
justifique suas escolhas, etc,

%% detalhamento.
%\noindent

\end{document}
