\documentclass[a4paper,11pt]{article}
\usepackage[T1]{fontenc}
\usepackage[utf8]{inputenc} %% isto garante compatibilidade com seu MAC
\usepackage{lmodern}
\usepackage[brazil]{babel}
\usepackage{comment, color} %%% 
%\usepackage{graphicx, url}
\usepackage{amsmath}
\usepackage{amsfonts}
\usepackage{amssymb}
%%%\usepackage[normalem]{ulem}

\topmargin       0.15cm
\headheight      0pt
\headsep         -0.5cm
\textheight      25cm
\textwidth       16.7cm
\oddsidemargin   -5mm
\evensidemargin  -5mm
\baselineskip    -13pt

\begin{document}
\framebox[15cm][c]{$3^a$ Avaliação de Lógica Matemática  (LMA) - Joinville, \today}

%\author{Rogério Eduardo da Silva e Claudio Cesar de Sá}
%\date{\today}

\vskip 0.5cm Acad\^emico(a) : \rule{10cm}{0.4pt} Turma:  \rule{1cm}{0.4pt}
%%%\noindent Algumas questões desta prova vieram de \url{http://www.cs.utsa.edu/~bylander/cs2233/index.html}

\begin{enumerate}
\setlength{\itemsep}{-1pt}



\item {\bf (2.0 pts.)} Determine o valor verdade $\{V, F \}$ (a interpretação $\Phi $)
de cada uma das fórmulas abaixo em seu respectivo domínio. Dados: $A = \{ -3,  5 \}$, $B = \{ -10, 0, 10\}$ e  $C = \{ 2,  4 \}$.
Faça os cálculos em separado e preencha a tabela abaixo.

\begin{center}
\begin{tabular}{l|c|c|c|c} \hline \hline
 & \multicolumn{4}{c}{Domínios} \\ \hline
 & $ x \in A$ & $x \in A ~e~ y \in C$ & $x \in B ~e~ y \in C$ & $x \in C$ \\ \hline

$\forall x (2x \leq x^2)$ & & --xxx-- & --xxx--  & \\ \hline
$\exists x ((2x)^2 > 16)$ & &  --xxx-- & --xxx--  & \\ \hline
$\forall x (x^3 < 1)$ & &  --xxx-- & --xxx-  & \\ \hline
$\exists y \forall x (y = x^3)$ & --xxx-- &  & & --xxx-- \\ \hline
$\forall x \exists y (xy \leq 1)$ & --xxx-- &  & & --xxx-- \\ \hline \hline
\end{tabular}
\end{center}

%\item  {\bf (1.0 pt)} Reescreva a negação de todos os predicados 
%da questão acima. Exemplo: $\forall x \:\: p(x)$ negando
%fornece $\sim \forall x \:\: p(x) \equiv  \exists x \:\: \sim p(x)$.
%Isto é, elimine todas  negações em frente aos quantificadores.

\item {\bf (1.0 pt)} Traduza as expressões abaixo (apresentadas em linguagem natural) para uma representação equivalente em lógica de primeira ordem. Contudo, 
justifique a escolha dos predicados e átomos.

\begin{enumerate}
\setlength{\itemsep}{-2pt} 
\item Alguém está em casa %%% 
%%Errar é humano
\item Existe um número que é par
\item Quem estuda sempre alcança bons resultados
\item Todos os alunos têm exatamente um número e este é igual a sua matrícula
% \item A necessidade é a mãe de todas as invenções
\item A inteligência do homem resolve problemas que os músculos não conseguem
\end{enumerate}

%\item {\bf (1.0 pt)} Escreva claramente uma descrição para as fórmula abaixo, indicando
%quais são logicamente equivalentes entre elas (da ``a'' a ``e''), e quais não são.
%Explique suas respostas.
%%% pagina 371 do livro em ingles que estah no Dropbox
%%\textcolor{red}{Rogério: esta é nova ... veja se está bom}
%\begin{enumerate}
%\setlength{\itemsep}{-3pt}
%  \item  $\exists x \sim p(x)$ 
%  \item  $ \exists x \forall y (p(y) \rightarrow y = x)$
%  \item  $ \exists x \forall y (p(y) \leftrightarrow y = x)$
%   \item $ \forall x \forall y (p(x) \wedge p(y)) \rightarrow x = y)$
%   \item $ \forall x \forall y (p(x) \wedge p(y)) \leftrightarrow x = y)$
%\end{enumerate}

%\item {\bf (1.5 pt)} Aplicando De Morgan aos
%quantificadores das fórmulas de LPO, dar a
%negação das seguintes sentenças lógicas:
%\begin{enumerate}
%\setlength{\itemsep}{-2pt}
% \item $ \forall x \exists y (\sim p(x) \wedge \sim q(y))$
% \item $ \exists x \forall y (p(x) \vee \sim q(y))$
% \item $ \exists x \forall y (p(x)\rightarrow q(y))$
% \item $ \forall x \exists y (\sim p(x) \vee \sim q(y))$
%  \item  $ \exists x \forall y (p(y) \rightarrow y = x)$
%  \item  $ \forall x \exists y (p(y) \leftrightarrow y = x)$
%\end{enumerate}

\item {\bf (3.0 pts.)} Seja o conjunto das seguintes fórmulas em lógica de primeira-ordem (LPO):\\
\begin{tabular}{ll}
\\  \hline \hline
  % after \\: \hline or \cline{col1-col2} \cline{col3-col4} ...
    1. &  $\forall x \exists y ( come(x,y)  \rightarrow cadeia\_alimentar(x,y) )$ \\
    2. &  $ \forall x \exists y \exists z( cadeia\_alimentar(x,z) \wedge come(z,y) \rightarrow cadeia\_alimentar(x,y) ) $ \\
    3. &  $ come(jacare, gueopardo) $ \\
    4. &  $ come(jacare, leao) $ \\
    5. &  $ come(leao, gnus) $ \\
    6. &  $ come(leao, hienas) $ \\
   %% 7. &  $ come(gueopardo, gnus) $ \\
    7 &  $ come(gueopardo, gazelas) $ \\
    8. &  $ come(gnus, mata\_nativa) $ \\
    9. &  $ come(gazelas, mata\_nativa) $ \\
    \hline \hline
 \end{tabular}

Demonstre que $gnus$, $gazelas$, $hienas$ e $mata\_nativa$ estão na cadeia alimentar
do $jacare$. {\bf Uma} leitura  dos predicados é: $come(x,y)$, ``{\em x come y}'';  e 
$cadeia\_alimentar(x,y)$, ``{\em x está na cadeia alimentar de  y}''. Quem mais é alimentado
indiretamente por quem?
PS: Indique claramente cada passo realizado.

\begin{comment}
\begin{tabular}{ll}
 \hline \hline
  % after \\: \hline or \cline{col1-col2} \cline{col3-col4} ...
    1. &  $\forall y \exists x ( pessoa(y) \wedge pet(x) \wedge vacinado(x) \rightarrow ama(y, x) )$ \\
    2. &  $ \forall x ( pet(x) \wedge saudavel(x) \rightarrow vacinado(x) ) $ \\
    3. &  $ pessoa(mickey) $ \\
    4. &  $ pet(pluto) $ \\
    5. &  $saudavel(pluto)$ \\
    \hline \hline
 \end{tabular}

 Na sequência abaixo, resolva as seguintes questões:
\begin{enumerate}
\setlength{\itemsep}{-3pt}
\item {\bf (1.0 pt)} Interprete textualmente o significado de cada fórmula acima
\item {\bf (2.0 pts)} Utilizando as propriedades da LPO, PU's, PE's e regras de inferências, deduza se {\it Mickey} ama {\it Pluto}.
\end{enumerate}
\end{comment}


\item {\bf (2.0 pts.)} Em Prolog, imagine um predicado {\tt pessoa(Nome, Ano)}, onde {\it Ano} representa o ano de nascimento da pessoa. Crie um predicado para calcular a idade dessa pessoa, e depois utilize este predicado para classificar a pessoa em:
\begin{small}
\begin{description}
\setlength{\itemsep}{-3pt}
\item[Bebê:] $idade < 2$
\item[Criança:] $2 \leq idade < 12$
\item[Adolescente:] $12 \leq idade < 21$
\item[Adulto:] $21 \leq idade < 60$
\item[Idoso:] $idade \geq 60$
\end{description}
\end{small}

Assuma as suas convenções de seu código e justifique-as.

\item {\bf (2.0 pts.)} Analise o código Prolog apresentado abaixo e informe qual é  a sequência de respostas válidas para a inferência {\tt resultado(X,Y,Z).} (todas respostas) 
{\small (na mesma ordem que seria apresentada pelo Prolog)}:

\begin{verbatim}
a(1).
a(2).
a(3).
b(4).
b(5).
b(6).
resultado(X,Y,Z) :- a(X), 
                    b(Y), 
                    predicado(X,Y,Z).
predicado(X,X,0).
predicado(X,Y,Z) :- W is (Y-1), 
                    W > 0,  
                    predicado(X,W,R), 
                    Z is (R + 1).
\end{verbatim}
\end{enumerate}


\underline{{\large Equivalências Notáveis}}:
{\small
\begin{description}
\setlength{\itemsep}{-4pt}

\item[Idempotência (ID):] $P\Leftrightarrow P\wedge P$ ou $P\Leftrightarrow P\vee P$
\item[Comutação (COM):] $P\wedge Q\Leftrightarrow Q\wedge P$ ou $P\vee Q\Leftrightarrow Q\vee P$
\item[Associação (ASSOC):] $P\wedge(Q\wedge R)\Leftrightarrow (P\wedge Q)\wedge R$ ou $P\vee(Q\vee R)\Leftrightarrow (P\vee Q)\vee R$ 
\item[Distribuição (DIST):] $P\wedge(Q\vee R)\Leftrightarrow (P\wedge Q)\vee (P \wedge R)$ ou $P\vee(Q\wedge R)\Leftrightarrow (P\vee Q)\wedge (P\vee R)$
\item[Dupla Negação (DN):] $P\Leftrightarrow\sim\sim P$
\item[De Morgan (DM):] $\sim(P \wedge Q) \Leftrightarrow \sim P \vee\sim Q$ ou $\sim(P \vee Q) \Leftrightarrow \sim P \wedge\sim Q$
\item[Equivalência da Condicional (COND):] $P\rightarrow Q \Leftrightarrow\sim P \vee Q$

\item[Bicondicional (BICOND):] $P\leftrightarrow Q \Leftrightarrow (P\rightarrow Q)\wedge(Q\rightarrow P)$

\item[Contraposição (CP):] $P\rightarrow Q \Leftrightarrow \sim Q\rightarrow\sim P$

\item[Exportação-Importação (EI):] $P\wedge Q\rightarrow R \Leftrightarrow P\rightarrow(Q\rightarrow R)$

\item[Contradição:] $P\wedge \sim P \Leftrightarrow \square $

\item[Tautologia:] $ P\vee \sim P \Leftrightarrow \blacksquare    $

\item[Negações para LPO:] $ \sim \forall x: px \Leftrightarrow \exists x: \sim px $
\item[Negações para LPO:] $ \sim \exists x: px \Leftrightarrow \forall x: \sim px $
\end{description}
}

\underline{{\large Regras Inferencias Válidas (Teoremas)}}:
{\small
\begin{description}
\setlength{\itemsep}{-4pt}
\item[Adição (AD):] $P \vdash P \vee Q$ ou $P \vdash Q \vee P$
\item[Simplificação (SIMP):] $P \wedge Q \vdash P$ ou $P \wedge Q \vdash Q$
\item[Conjunção (CONJ)] $P, Q \vdash P \wedge Q$ ou $P, Q \vdash Q \wedge P$
\item[Absorção (ABS):] $P \rightarrow Q \vdash P \rightarrow (P \wedge Q)$
\item[Modus Ponens (MP):] $P \rightarrow Q, P \vdash Q$
\item[Modus Tollens (MT):] $P \rightarrow Q, \sim Q \vdash \sim P$
\item[Silogismo Disjuntivo (SD):] $P \vee Q, \sim P \vdash Q$ ou $P \vee Q, \sim Q \vdash P$
\item[Silogismo Hipotético (SH):] $P \rightarrow Q, Q\rightarrow R \vdash P\rightarrow R$
\item[Dilema Construtivo (DC):] $P\rightarrow Q, R\rightarrow S, P \vee R \vdash Q\vee S$
\item[Dilema Destrutivo (DD):] $P\rightarrow Q, R\rightarrow S, \sim Q\vee\sim S \vdash \sim P \vee\sim R$
\end{description}
%\end{enumerate}

\begin{flushleft}
\underline{Observações}:
\begin{enumerate}
\setlength{\itemsep}{-2pt}
\item Qualquer dúvida, desenvolva a questão e deixe tudo
explicado, detalhadamente, que avaliaremos o seu conhecimentos sobre
 o assunto;
 \item \underline{Clareza e legibilidade};
\end{enumerate}
\end{flushleft}
}
\end{document}
