
\documentclass[12pt, a4paper,final]{article}
\usepackage{t1enc}
\usepackage[latin1]{inputenc}
\usepackage[portuges]{babel}
\usepackage{amsmath}
\usepackage{amsfonts}
\usepackage{amssymb}

%\usepackage{graphicx}
\topmargin       -1cm
%\headheight      17pt
 \headsep  1cm

\textheight      24cm

\textwidth       16.7cm

\oddsidemargin   2mm

\evensidemargin  2mm

\pagestyle{empty}

\begin{document}
\begin{center}
\framebox[\textwidth][c]{L�gica e
Programa��o em L�gica  (Jlle, \today )}
%%\newline

\fbox{\hskip 7cm Exame Final \hskip 7cm }
\end{center}

\vskip1cm Aluno(a): \hrulefill
%%%\noindent

\begin{enumerate}
\setlength{\itemsep}{-5pt}

 \item Determinar
as seguintes Formas Normais:
\begin{description}
\setlength{\itemsep}{-5pt}
 \item [Disjuntiva] para: $(\sim p \rightarrow  \sim q) \wedge \sim(q \rightarrow p) $
 \item [Conjuntiva] para: $\sim (\sim p \vee \sim q) \rightarrow (\sim q \wedge \sim p) $
\end{description}

\item Demonstrar que o conjunto das
proposi��es abaixo geram uma contradi��o
(isto �, derivam ma inconsist�ncia, i. �:
$\Box $):

\begin{enumerate}
\setlength{\itemsep}{-2pt}

\item %%\vskip 11pt
\begin{tabular}{ll}
  % after \\: \hline or \cline{col1-col2} \cline{col3-col4} ...
    1 &  $ x = 0 \leftrightarrow y + x = y$ \\
    2 &  $ x > 1  \wedge  x = 0$ \\
    3 &  $ x + y = y \rightarrow x \ngtr 1 $
\end{tabular}

\item \vskip 11pt

%% fim este ... ok
\begin{tabular}{ll}
  % after \\: \hline or \cline{col1-col2} \cline{col3-col4} ...
    1 &  $ \sim p \vee \sim q$ \\
    2 &  $ p \wedge s$ \\
    3 &  $ r \rightarrow r \wedge q $ \\
    4 &  $ \sim s \vee r $
\end{tabular}
\end{enumerate}
Conclua algo sobre.

 \item  Considere um dom�nio $A =
 \{1,2,3,4,5,6,7,8,9\}$, e dada as seguintes
 f�rmulas:

\begin{enumerate}
\setlength{\itemsep}{-2pt}
 \item $p(x): x^3 \subseteq A$ e $q(x):$ x � �mpar.
 \item $r(x): x + 5 \leq 9 $ e $s(x):$ x � par.
\end{enumerate}
Calcule os valores verdades
(\emph{conjunto-verdade}, isto �, onde a
${\mathbf\Phi } ($f�rmula$) = V$)
 para seguintes f�rmulas:
 (a) $p \rightarrow q$ \hskip 0.7cm  (b) $q \vee
 p$  \hskip 0.7cm
 (c) $r \rightarrow s$ \hskip 0.7cm (d) $s \wedge r$

\item Dar a nega��o das seguintes
proposi��es:
\begin{enumerate}
\setlength{\itemsep}{-2pt}
 \item $ \sim \forall x \exists y (\sim p(x,y) \wedge \sim q(y,x))$
 \item $ \exists x \sim \forall y (p(x) \vee \sim q(y))$
 \item $ \sim \exists x \forall y (p(x) \rightarrow q(y))$
 \item $ \forall x \sim \exists y (\sim p(x) \vee \sim q(y))$
\end{enumerate}

%%%%
\item Escreva o seguinte enunciado sob a
nota��o da l�gica de primeira ordem, em um
conjunto de premissas e uma conclus�o:
\begin{quote}
``{\em Todo estudante de CC trabalha mais do
que algu�m. Se todo mundo que trabalha mais
do que uma pessoa, ent�o dorme menos que
essa pessoa. Maria � estudante de CC.
Portanto, Maria dorme menos do que algu�m}''
\end{quote}
Utilize os seguintes predicados:
\begin{itemize}
\setlength{\itemsep}{-5pt}
    \item trab$(x,y)$
    \item dorme$(z,w)$
    \item estuda\_cc$(x)$, e "maria"
\end{itemize}

\item  Resolva as seguintes regras  em
Prolog, que implemente uma fun\c{c}\~{a}o
recursiva, definida por:
\begin{enumerate}
\setlength{\itemsep}{-5pt}

 \item Imprima a
sequ�ncia de n�meros entre $N1$ e $N2$ em
ordem ascendente, tal que: $N1 < N2$;

\item Como se faz para for�ar um ``{\em
backtracking}'' sobre qualquer regra, e que
o resultado seja sempre ``{\em yes}''
(lembrar do exemplo do menu da apostila);

\item Regras que realizem as quatro
opera��es b�sicas ($+$, $-$, $/$, e
$\times$), para uma entrada do tipo $N1$ e
$N2$.

%Seja a formula��o:
%``{\em Todos objetos pertences h� uma
%classe. Todo componente � composto por uma
%classe, ou um componente � composto por
%outros componentes e suas classes.}''

\end{enumerate}
\end{enumerate}
Observa��o: Clareza,  legibilidade, e
detalhamento.
%\noindent

\end{document}
