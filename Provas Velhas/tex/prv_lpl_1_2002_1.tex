%% This document created by Scientific Word (R) Version 3.0



\documentclass[12pt, a4paper,portuges]{article}
\usepackage{t1enc}
\usepackage[latin1]{inputenc}
\usepackage[portuges]{babel}
\usepackage{amsmath}
\usepackage{amsfonts}
\usepackage{amssymb}

%\usepackage{graphicx}
\topmargin       -1cm
 \headheight      17pt
 \headsep  1cm

\textheight      24cm

\textwidth       16.3cm

\oddsidemargin   2mm

\evensidemargin  2mm

\pagestyle{empty}

\begin{document}
\begin{center}
\framebox[\textwidth][c]{L�gica e Programa��o em L�gica  (LPL)-
Joinville, \today}
%%\newline
\end{center}

\vskip1cm Aluno(a): \hrulefill

%%%\noindent

\begin{enumerate}
\setlength{\itemsep}{-5pt}
 \item Determine o valor l�gico para:
\begin{enumerate}
\setlength{\itemsep}{-5pt}
 \item $3+4=7$ se somente se $5^4=125 $;
 \item $3^2 + 4^2 = 5^2$ se
somente se $\pi $ � irracional; \item $5^2=10$ ou  $\pi $ �
irracional.
\end{enumerate}

 \item Identificar e simbolizar as seguintes proposi��es matem�ticas:

 \begin{enumerate}
\setlength{\itemsep}{-5pt}
 \item ``{\em x � maior que 5 e menor
que 7 ou x n�o � igual a 6}'';
\item ``{\em Se x � menor que 5 e
maior que 3, ent�o x � igual a 4}'';
\item ``{\em  � falso que
Carlos fala ingl�s ou alem�o, mas que n�o fala franc�s}''.

\end{enumerate}
\item  Identifique as proposi��es e escreva as f�rmulas
proposicionais:
\begin{enumerate}
\setlength{\itemsep}{-5pt}
\item  ``\emph{Se o programa \'{e}
eficiente, ent\~{a}o ele executar\'{a} eficientemente. Ou o
programa \'{e} eficiente ou ele tem um erro. O programa n\~{a}o
executa eficientemente. O programa tem um erro}''.
%%%Portanto, programa tem um erro}''.
%%Encontre  ou demonstre esta \'{u}ltima conclus\~{a}o.

\item ``\emph{A colheita \'{e} boa, e n\~{a}o h\'{a} \'{a}gua
suficiente. Se tivesse bastante \'{a}gua ou n\~{a}o tivesse
bastante sol, ent\~{a}o \ haveria \'{a}gua suficiente. A colheita
\'{e} boa e  h\'{a} bastante sol}''.

%%Portanto, a colheita \'{e} boa e  h\'{a} bastante sol}''.
%%Verifique esta \'{u}ltima conclus\~{a}o.
\end{enumerate}

\item Identifique se a f�rmula �  tautol\'{o}gica, contingente (
satisfat\'{i}vel, consistente), ou inv\'{a}lida (contradit�ria,
insatisfat�vel):
\begin{enumerate}
\setlength{\itemsep}{-5pt}

\item $(A \leftrightarrow B) \wedge (A \vee B)$

\item $(A \vee B)\rightarrow (A \wedge  B)$

\item $(B \rightarrow A)\rightarrow (A \rightarrow  B)$

\item $(A \rightarrow (A \rightarrow B)) \rightarrow  B $

\item $(A \rightarrow (B \rightarrow (B \rightarrow  A))) $
\end{enumerate}

\item Demonstre as seguintes equival�ncias l�gicas:
\begin{enumerate}
\setlength{\itemsep}{-5pt}

\item $A \leftrightarrow B \equiv (\sim A \wedge \sim B)\vee( A
\wedge B)$

\item $(A \rightarrow (A \rightarrow (A \rightarrow B ))))\equiv A
\rightarrow B $
\end{enumerate}


\item Sabendo que uma f�rmula \textbf{$A \rightarrow B$}, tem por
defini��o como sua f�rmula contrapositiva: \textbf{$\sim B
\rightarrow \sim A$}. Qual � a contrapositiva de uma
contrapositiva? Explique passo-a-passo porqu�?

\item Sabendo que uma f�rmula \textbf{$A \rightarrow B$}, tem por
defini��o como sua f�rmula contr�ria: \textbf{$\sim A \rightarrow
\sim B$}. Qual � a contrapositiva dessa f�rmula? Explique
passo-a-passo porqu�?



\end{enumerate}



%\vskip1cm

%\noindent

%

\end{document}
