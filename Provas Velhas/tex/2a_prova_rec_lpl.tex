\documentclass[12pt, a4paper,final]{article}
\usepackage{t1enc}
\usepackage[latin1]{inputenc}
\usepackage[portuges]{babel}
\usepackage{amsmath}
\usepackage{amsfonts}
\usepackage{amssymb}

%\usepackage{graphicx}
\topmargin       -1cm
%\headheight      17pt
 \headsep  1cm

\textheight      24cm

\textwidth       16.7cm

\oddsidemargin   2mm

\evensidemargin  2mm

\pagestyle{empty}

\begin{document}
\begin{center}
\framebox[\textwidth][c]{L�gica e
Programa��o em L�gica  (Jlle, \today )}
%%\newline

\fbox{\hskip 2cm Prova de Recupera��o \hskip 2cm }
\end{center}

\vskip1cm Aluno(a): \hrulefill
%%%\noindent

\begin{enumerate}
\setlength{\itemsep}{-5pt}

 \item Determinar
as seguintes Formas Normais:
\begin{description}
\setlength{\itemsep}{-5pt}
 \item [Disjuntiva] para: $(\sim p \rightarrow  \sim q) \wedge \sim(q 
\rightarrow p) $
 \item [Conjuntiva] para: $\sim (\sim p \vee \sim q) \rightarrow (\sim q \wedge 
\sim p) $
\end{description}

\item Demonstrar que o conjunto das
proposi��es abaixo geram uma contradi��o
(isto �, derivam ma inconsist�ncia, i. �:
$\Box $):

\begin{enumerate}
\setlength{\itemsep}{-2pt}

\item %%\vskip 11pt
\begin{tabular}{ll}
  % after \\: \hline or \cline{col1-col2} \cline{col3-col4} ...
    1 & $ x = 0 \leftrightarrow y + x = y$ \\
    2 & $ x > 1  \wedge  x = 0$ \\
    3 & $ x + y = y \rightarrow x \ngtr 1 $
\end{tabular}

\item \vskip 11pt

%% fim este ... ok
\begin{tabular}{ll}
  % after \\: \hline or \cline{col1-col2} \cline{col3-col4} ...
    1 & $ \sim p \vee \sim q$ \\
    2 & $ p \wedge s$ \\
    3 & $ r \rightarrow r \wedge q $ \\
    4 & $ \sim s \vee r $
\end{tabular}
\end{enumerate}
Conclua algo sobre.


\item  Verificar a validade dos
argumentos que se seguem:
\begin{enumerate}
\setlength{\itemsep}{-2pt}
 \item $p\rightarrow q$, $q \leftrightarrow s$, $t\vee (r\wedge \sim s)$
{\bf $\vdash $} $p \rightarrow t$

\item $\sim p\vee q \rightarrow r$,  $r \vee s
\rightarrow \sim t$, t {\bf $\vdash $} $\sim q$

\end{enumerate}

 \item  Demonstre a equival�ncia l�gica de :
\begin{enumerate}
\setlength{\itemsep}{-5pt}
 \item $p \wedge \sim q \rightarrow \Box \Leftrightarrow   p \rightarrow  q$
 \item $ p \rightarrow  (q\rightarrow  r) \Leftrightarrow p \wedge q \rightarrow r$
\end{enumerate}

\item Verifique a argumenta��o abaixo, e valide
 o resultado: \vskip 12pt

\begin{tabular}{ll}
  % after \\: \hline or \cline{col1-col2} \cline{col3-col4} ...
  Se & $x = y$, ent�o $x = z$ \\
  Se & $x = z$, ent�o $x = t$ \\
  Ou & $x = y$, ou $x = 0$ \\
  Se & $x = 0$, ent�o $x + u = 1$ \\
  Mas & $x + u \not= 1$ \\ \hline
  Portanto & $x = t$ \\
\end{tabular}

Conclua algo sobre.

\end{enumerate}

Observa��o: Clareza,  legibilidade, e
detalhamento. Caso tenha algum erro em algum
enunciado, corrija-o e justifique.
%\noindent

\end{document}

