
\documentclass[12pt, a4paper,final]{article}
\usepackage{t1enc}
\usepackage[latin1]{inputenc}
\usepackage[portuges]{babel}
\usepackage{amsmath}
\usepackage{amsfonts}
\usepackage{amssymb}

%\usepackage{graphicx}
\topmargin       -1cm
 \headheight      17pt
 \headsep  1cm

\textheight      24cm

\textwidth       16.3cm

\oddsidemargin   2mm

\evensidemargin  2mm

\pagestyle{empty}

\begin{document}
\begin{center}
\framebox[\textwidth][c]{L�gica e Programa��o em L�gica  (LPL)-
Joinville, \today}
%%\newline
\end{center}

\vskip1cm Aluno(a): \hrulefill
%%%\noindent

\begin{enumerate}
\setlength{\itemsep}{-5pt}
 \item (cap. 8) Demonstre a equival�ncia l�gica de :
\begin{enumerate}
\setlength{\itemsep}{-5pt}
 \item $p \wedge \sim q \rightarrow \Box \Leftrightarrow   p \rightarrow  q$
 \item $ p \rightarrow  (q\rightarrow  r) \Leftrightarrow p \wedge q \rightarrow r$
\end{enumerate}

\item (cap. 8) Determinar as seguintes Formas
Normais:
\begin{description}
\setlength{\itemsep}{-5pt}
 \item [Disjuntiva] para: $(p \rightarrow q) \wedge \sim(q \rightarrow p) $
 \item [Conjuntiva] para: $\sim (\sim p \rightarrow q) \vee (q \rightarrow \sim p) $
\end{description}

\item (cap. 11) Verificar a validade dos
argumentos que se seguem:
\begin{enumerate}
\setlength{\itemsep}{-2pt}
 \item $p\rightarrow \sim q$, $\sim p \rightarrow (r \rightarrow \sim q)$,
 $(\sim s \vee \sim r)\rightarrow \sim \sim q$,
 $\sim s$ $\vdash $ $\sim r$
 \item Verifique a argumenta��o abaixo, e valide
 o resultado: \vskip 12pt

\begin{tabular}{ll}
  % after \\: \hline or \cline{col1-col2} \cline{col3-col4} ...
  Se & $x = y$, ent�o $x = z$ \\
  Se & $x = z$, ent�o $x = t$ \\
  Ou &  $x = y$, ou $x = 0$ \\
  Se & $x = 0$, ent�o $x + u = 1$ \\
  Mas & $x + u \not= 1$ \\ \hline
  Portanto & $x = t$ \\
\end{tabular}
\end{enumerate}


\item (cap. 12) Verificar a validade dos
argumentos que se seguem:
\begin{enumerate}
\setlength{\itemsep}{-2pt}
 \item $p\rightarrow q$, $q \leftrightarrow s$, $t\vee (r\wedge \sim s)$
{\bf $\vdash $} $p \rightarrow t$

\item $\sim p\vee q \rightarrow r$,  $r \vee s
\rightarrow \sim t$, t {\bf $\vdash $} $\sim q$

\end{enumerate}

\item (cap. 12) Demonstrar que o conjunto das
proposi��es abaixo geram uma contradi��o (isto �,
derivam ma inconsist�ncia, i. �: $\Box $):
\begin{enumerate}
\setlength{\itemsep}{-2pt}

\item %%\vskip 11pt
\begin{tabular}{ll}
  % after \\: \hline or \cline{col1-col2} \cline{col3-col4} ...
    1 &  $x = 1 \rightarrow y < x$ \\
    2 &  $y < x \rightarrow  y = 0$ \\
    3 &  $\sim (y=0 \vee x \not= 1)$
\end{tabular}
\item \vskip 11pt
\begin{tabular}{ll}
  % after \\: \hline or \cline{col1-col2} \cline{col3-col4} ...
    1 &  $ p \vee s \rightarrow q$ \\
    2 &  $q \rightarrow \sim r$ \\
    3 &  $t \rightarrow p$ \\
    4 &  $t \wedge r $
\end{tabular}

\end{enumerate}

\end{enumerate}

\vskip 1cm Observa��o: Nas demonstra��es, detalhe
todos os passos feitos, isto qual a RI utilizada,
qual a equival�ncia \ldots, etc. Clareza e
legibilidade.
%\noindent

\end{document}
