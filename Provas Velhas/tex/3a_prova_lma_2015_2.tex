\documentclass[a4paper,11pt]{article}
\usepackage[T1]{fontenc}
\usepackage[utf8]{inputenc} %% isto garante compatibilidade com seu MAC
\usepackage{lmodern}
\usepackage[brazil]{babel}
\usepackage{comment, color} %%% 
\usepackage{graphicx, url}
\usepackage{amsmath}
\usepackage{amsfonts}
\usepackage{amssymb}
%%%\usepackage[normalem]{ulem}

\topmargin       0.15cm
\headheight      0pt
\headsep         -0.5cm
\textheight      25cm
\textwidth       16.7cm
\oddsidemargin   -5mm
\evensidemargin  -5mm
\baselineskip    -13pt

\begin{document}
\framebox[15cm][c]{$3^a$ Avaliação de Lógica Matemática  (LMA) - Joinville, \today}

%\author{Rogério Eduardo da Silva e Claudio Cesar de Sá}
%\date{\today}

\vskip 0.5cm Acad\^emico(a) : \rule{10cm}{0.4pt} Turma:  \rule{1cm}{0.4pt}
%%%\noindent Algumas questões desta prova vieram de \url{http://www.cs.utsa.edu/~bylander/cs2233/index.html}

\begin{enumerate}
%\setlength{\itemsep}{-1pt}

%{\Huge Rogerio pegar exercicios daqui }
%\url{http://cnx.org/contents/46af3ad0-8636-4653-b64a-ea5f934df9d6@28/Exercises_for_First-Order_Logi}
\item Atenção: Exame Final dia 14/12 (2a. feira às 17:00 hrs.)


\item {\bf (1.0 pt.)} Ao contrário  do que voce vai fazer na próxima questão,
seja o conjunto $N^\ast$ dos números naturais. Determine o 
conjunto-verdade ou domínio para o qual a fórmula é \textbf{verdadeira},  
para cada  uma das fórmulas abaixo:

\begin{enumerate}
\itemsep -2pt
   \item $\forall x . ((2x = 6) \vee (2x = 8)) $, por exemplo (este é o mais difícil), aqui o valor da resposta é $D=\{3, 4\}$
    \item $\forall x . (x^2 - 5x + 6 = 0)$
        \item $\exists x . (x^2 - 5x + 6 = 0)$
   \item $\exists x . (x^2 - 3x = 0)$ 
      \item $\forall x . (x-1 < 4 )$
     \item $\exists x . \sim (\text{ x é ímpar})$ 
     %%% \item REPENSAR AS FORMULAS.....
 \end{enumerate} 



\item {\bf (2.5 pts.)} Determine o valor verdade $\{V, F \}$ (a interpretação $\Phi $)
de cada uma das fórmulas abaixo em seu respectivo domínio. Dados: $A = \{ 3,  5 \}$, $B = \{ -15, 1, 15\}$ e  $C = \{ 6,  7 \}$.
As questões serão \textbf{apenas} validadas mediante os cálculos em separado. Em seguida preencha a tabela abaixo:

%{\Huge Veja pagina 42 exercicio 5 do livro do CHANG-LEE}\\
%trocarimos por esta daqui ....

\begin{center}
\begin{tabular}{l|c|c|c|c} \hline \hline
 & \multicolumn{4}{c}{Domínios} \\ \hline
 & $ x \in A$ & $x \in A ~e~ y \in C$ & $x \in B ~e~ y \in A$ & $x \in B$ \\ \hline

$\forall x (2x \leq x^2)$ & & --xxx-- & --xxx--  & \\ \hline
$\exists x \exists y ((2+x)^2 \geq 24 - y)$ &  --xxx-- & & & --xxx-- \\ \hline
$\forall x (x^2 \geq 5)$ & & --xxx-- & --xxx-  & \\ \hline
$\exists y \forall x (3x \neq y^2)$ & --xxx-- &  & & --xxx-- \\ \hline
$\forall x \exists y (xy \leq 50)$ & --xxx-- &  & & --xxx-- \\ \hline \hline
\end{tabular}
\end{center}
PS: esta questão é longa (10 cálculos a serem feitos), preste atenção, seja organizado.

%\item  {\bf (1.0 pt)} Reescreva a negação de todos os predicados 
%da questão acima. Exemplo: $\forall x \:\: p(x)$ negando
%fornece $\sim \forall x \:\: p(x) \equiv  \exists x \:\: \sim p(x)$.
%Isto é, elimine todas  negações em frente aos quantificadores.

%\item {\bf (1.0 pt)} Escreva claramente uma descrição para as fórmula abaixo, indicando
%quais são logicamente equivalentes entre elas (da ``a'' a ``e''), e quais não são.
%Explique suas respostas.
%%% pagina 371 do livro em ingles que estah no Dropbox
%%\textcolor{red}{Rogério: esta é nova ... veja se está bom}
%\begin{enumerate}
%\setlength{\itemsep}{-3pt}
%  \item  $\exists x \sim p(x)$ 
%  \item  $ \exists x \forall y (p(y) \rightarrow y = x)$
%  \item  $ \exists x \forall y (p(y) \leftrightarrow y = x)$
%   \item $ \forall x \forall y (p(x) \wedge p(y)) \rightarrow x = y)$
%   \item $ \forall x \forall y (p(x) \wedge p(y)) \leftrightarrow x = y)$
%\end{enumerate}

\item {\bf (1.5 pts)} Aplicando De Morgan aos
quantificadores das fórmulas de LPO, dar a
negação das seguintes sentenças lógicas:
\begin{enumerate}
\setlength{\itemsep}{-2pt}
 \item $ \forall x \:\: \exists y \:\:(p(x) \wedge \sim q(y))$
 \item $ \forall x\:\: \forall y \:\: \sim (\sim p(x) \vee \sim q(y))$
 \item $ \exists x \:\:\forall y\:\: (p(x)\rightarrow q(y))$
 \item $ \forall x\:\: \exists y\:\: (\sim p(x) \vee \sim q(y))$
  \item  $ \forall y\:\: (p(y) \rightarrow \exists x \:\: q(x))$
  \item  $ \forall x\:\: (p(x) \leftrightarrow \sim \exists y \:\: r(y) )$
\end{enumerate}
PS: Lembre que De Morgan não se aplica com os conectivos $\leftrightarrow$ e $\rightarrow$.

\item {\bf (2.5 pts.)} Seja o conjunto das seguintes fórmulas em lógica de primeira-ordem (LPO), as quais descrevem o comportamento de um adversário 
autônomo (NPC--\textit{nerd por computador}) em um videogame :

\begin{tabular}{ll}
\\  \hline \hline
  % after \\: \hline or \cline{col1-col2} \cline{col3-col4} ...
  1. & agente(oponente) \\
  2. & estado(oponente, fome) \\
  3. & fruta(banana) \\
  4. & fruta(laranja) \\
  5. & sanduiche(bigmac) \\
  6. & $\forall X \exists Y: (fruta(X) \vee sanduiche(Y) \rightarrow alimento(X) )$ \\
  7. & $\exists A \forall X: (agente(A) \wedge estado(A, fome) \wedge alimento(X) \rightarrow decisao(A, comer, X))$ \\
    \hline \hline
 \end{tabular}

Demonstre as possíveis decisões que o \textbf{agente} pode executar quando está no estado ``{\tt fome}''.



\item {\bf (2.5 pts.)} Seja o conjunto das seguintes fórmulas em lógica de primeira-ordem (LPO), as quais descrevem uma estória:
\\
\begin{tabular}{ll}
 \hline \hline
  % after \\: \hline or \cline{col1-col2} \cline{col3-col4} ...
    1. &  $\forall y \exists x ( pessoa(y) \wedge pet(x) \wedge vacinado(x) \rightarrow ama(y, x) )$ \\
    2. &  $ \forall x ( pet(x) \wedge saudavel(x) \rightarrow vacinado(x) ) $ \\
    3. &  $ pessoa(mickey) $ \\
    4. &  $ pet(pluto) $ \\
    5. &  $ pet(garfield) $ \\
    6. &  $ saudavel(pluto)$ \\
    7. &  $ saudavel(garfield)$ \\

    \hline \hline
 \end{tabular}

 Na sequência abaixo, resolva as seguintes questões:
\begin{enumerate}
\setlength{\itemsep}{-3pt}
\item {\bf (1.0 pt)} Interprete textualmente o significado de cada fórmula acima
\item {\bf (1.5 pts)} Utilizando as propriedades da LPO, PU's, PE's e regras de inferências, demonstre quem  {\it Mickey} ama.
\end{enumerate}


\end{enumerate}


\underline{{\large Equivalências Notáveis}}:
{\small
\begin{description}
\setlength{\itemsep}{-2pt}

\item[Idempotência (ID):] $P\Leftrightarrow P\wedge P$ ou $P\Leftrightarrow P\vee P$
\item[Comutação (COM):] $P\wedge Q\Leftrightarrow Q\wedge P$ ou $P\vee Q\Leftrightarrow Q\vee P$
\item[Associação (ASSOC):] $P\wedge(Q\wedge R)\Leftrightarrow (P\wedge Q)\wedge R$ ou $P\vee(Q\vee R)\Leftrightarrow (P\vee Q)\vee R$ 
\item[Distribuição (DIST):] $P\wedge(Q\vee R)\Leftrightarrow (P\wedge Q)\vee (P \wedge R)$ ou $P\vee(Q\wedge R)\Leftrightarrow (P\vee Q)\wedge (P\vee R)$
\item[Dupla Negação (DN):] $P\Leftrightarrow\sim\sim P$
\item[De Morgan (DM):] $\sim(P \wedge Q) \Leftrightarrow \sim P \vee\sim Q$ ou $\sim(P \vee Q) \Leftrightarrow \sim P \wedge\sim Q$
\item[Equivalência da Condicional (COND):] $P\rightarrow Q \Leftrightarrow\sim P \vee Q$

\item[Bicondicional (BICOND):] $P\leftrightarrow Q \Leftrightarrow (P\rightarrow Q)\wedge(Q\rightarrow P)$

\item[Contraposição (CP):] $P\rightarrow Q \Leftrightarrow \sim Q\rightarrow\sim P$

\item[Exportação-Importação (EI):] $P\wedge Q\rightarrow R \Leftrightarrow P\rightarrow(Q\rightarrow R)$

\item[Contradição:] $P\wedge \sim P \Leftrightarrow \square $

\item[Tautologia:] $ P\vee \sim P \Leftrightarrow \blacksquare    $

\item[Negações para LPO:] $ \sim \forall x: px \Leftrightarrow \exists x: \sim px $
\item[Negações para LPO:] $ \sim \exists x: px \Leftrightarrow \forall x: \sim px $
\end{description}
}

\underline{{\large Regras Inferencias Válidas (Teoremas)}}:
{\small
\begin{description}
\setlength{\itemsep}{-2pt}
\item[Adição (AD):] $P \vdash P \vee Q$ ou $P \vdash Q \vee P$
\item[Simplificação (SIMP):] $P \wedge Q \vdash P$ ou $P \wedge Q \vdash Q$
\item[Conjunção (CONJ)] $P, Q \vdash P \wedge Q$ ou $P, Q \vdash Q \wedge P$
\item[Absorção (ABS):] $P \rightarrow Q \vdash P \rightarrow (P \wedge Q)$
\item[Modus Ponens (MP):] $P \rightarrow Q, P \vdash Q$
\item[Modus Tollens (MT):] $P \rightarrow Q, \sim Q \vdash \sim P$
\item[Silogismo Disjuntivo (SD):] $P \vee Q, \sim P \vdash Q$ ou $P \vee Q, \sim Q \vdash P$
\item[Silogismo Hipotético (SH):] $P \rightarrow Q, Q\rightarrow R \vdash P\rightarrow R$
\item[Dilema Construtivo (DC):] $P\rightarrow Q, R\rightarrow S, P \vee R \vdash Q\vee S$
\item[Dilema Destrutivo (DD):] $P\rightarrow Q, R\rightarrow S, \sim Q\vee\sim S \vdash \sim P \vee\sim R$
\end{description}
%\end{enumerate}

\begin{flushleft}
\underline{Observações}:
\begin{enumerate}
\setlength{\itemsep}{-2pt}
\item Qualquer dúvida, desenvolva a questão e deixe tudo
explicado, detalhadamente, que avaliaremos o seu conhecimentos sobre
 o assunto;
 \item \underline{Clareza e legibilidade};
\end{enumerate}
\end{flushleft}
}
\end{document}
