
\documentclass[10pt, a4paper,final]{article}
\usepackage{t1enc}
\usepackage[utf8]{inputenc}
\usepackage[portuges]{babel}
\usepackage{amsmath}
\usepackage{amsfonts}
\usepackage{amssymb}

%\usepackage{graphicx}
\topmargin       -1.5cm
\headheight      0pt
\headsep  0cm
\textheight      27cm
\textwidth       16.7cm
\oddsidemargin   -5mm
\evensidemargin  -5mm
\pagestyle{empty}


\begin{document}
\begin{center}
\begin{tabular}{||c||} \hline \hline 
{\Large Logica Matemática  (LMA)}  \\
\mbox{\hskip 2cm  UDESC/DCC -- \today  \hskip 2cm }
\\
Exame Final \\ \hline \hline
\end{tabular}
\end{center}

%\vskip1cm 
Aluno(a): \hrulefill
%%%\noindent

\begin{enumerate}
%\setlength{\itemsep}{-5pt}

\item Determinar as Formas Normais Disjuntiva e Conjuntiva para:
\begin{description}
\setlength{\itemsep}{-5pt}
 \item [1.] para: $(p \rightarrow q) \wedge \sim(q \rightarrow p) $
 \item [2.] para: $(\sim p \rightarrow q) \vee (q \rightarrow \sim p) $
\end{description}
Qual o tipo dessas fórmulas? (consistentes, ... etc)

%% OK
\item Efetuar a prova (ou demonstração) direta para validade dos argumentos
que se seguem: 
%\begin{enumerate}
%\setlength{\itemsep}{-2pt}
% \item 
\begin{center}
 $\{p\rightarrow \sim q \: , \:\:\: \sim p \rightarrow (r \rightarrow \sim q)  \: , \:\:\:
 (\sim s \vee \sim r)\rightarrow \sim \sim q  \: , \:\:\: \sim s  \} \vdash  r $ \\
 \end{center}
 %\end{enumerate}
%%Conclua algo sobre este teorema. 
Caso esse conjunto não derive um teorema, que mudanças
voce faria nas premissas?

%% OK
\item Utilizando o método de
 {\em demonstração por absurdo} ou {\em indireta},
 demonstre a validade do 
 argumento $ q $, a partir das premissas: \\
1. \hskip 0.2cm $ \sim p \vee \sim q $ \\
2. \hskip 0.2cm $  p \wedge s $ \\
3. \hskip 0.2cm $ r \vee \sim s $ \\
4. \hskip 0.2cm $ r \rightarrow (r \wedge q) $ \\
Isto é, esta sequência deduz ( $\vdash $, consiste
de um teorema) $ q $? Caso esse conjunto não derive um teorema, que mudanças
voce faria nas premissas?


\item Aplicando o método da Resolução e considerando os dois exercícios anteriores (2 e 3) como teoremas,  demonstre que:
\begin{enumerate}
\setlength{\itemsep}{-2pt}
\item $ r $ é um consequente lógico
\item $ q $ é um consequente lógico
\end{enumerate}

Construa passo-a-passo todas derivaç\~oes das novas cláusulas e faça a  árvore de expansão,  apresentando cada termo $\lambda$.

%%%%

\item Considere o seguinte conjunto de f\'ormulas: \\

\begin{tabular}{ll}
  % after \\: \hline or \cline{col1-col2} \cline{col3-col4} ...
1 &  $\forall x\forall y (q(x,y) \wedge r(y) \rightarrow p(y)) $ \\
2 &  $\forall x (q(x,x) \rightarrow p(x))  $ \\
3 &  $\forall x (s(x) \rightarrow q(x,x)) $ \\
4 &  $r(b)$ \\ 
5 &  $s(a)$ \\
6 &  $s(b)$ \\
\end{tabular}\\
Utilizando o método da Resolução, encontre uma resposta para $\sim p(X)$ e
pelas instâncias átomicas, onde o domínio é dado por $D=\{a,b\}$. Construa 
a árvore desta dedução/expansão.

\item Converta as seguintes fórmulas em cláusulas:
\begin{tabular}{ll}
  % after \\: \hline or \cline{col1-col2} \cline{col3-col4} ...
1 &  $ \forall x (Px \vee (\exists x Qx )) \rightarrow (\forall x (Px \vee Qx )) $ \\
2 &  $ \exists x Px  \rightarrow (\exists x  \forall z Qxz ) \vee (\forall z (Rxyz))$ \\
3 &  $ \forall x (Px \rightarrow (\exists y Qxy )) \rightarrow \sim (\forall z  Rxyz ) $ \\
\end{tabular}\\
Notação exemplo: $Rxyz \equiv R(x,y,z)$

\item Sam, Clyde e Oscar são elefantes. N\'os sabemos os seguintes 
fatos sobre eles:
\begin{itemize}
\setlength{\itemsep}{-2pt}
    \item Sam é rosa;
    \item Clyde é cinza e gosta de Oscar;
    \item Oscar é tanto rosa ou cinza, mas não tem as duas cores, e gosta de Sam.
\end{itemize}
Usando a Resolução como prova, demonstre que um elefante cinza gosta de um elefante rosa.
Isto é, prove que: $\exists x \:\: \exists y \:\:(cinza(x) \wedge rosa(y) \wedge gosta(x,y))$




\end{enumerate}

\vskip 2cm
\noindent Observação: Clareza,  legibilidade, em caso de dúvida {\bf faça} e
justifique suas escolhas !\\
\noindent Boas férias!

%% detalhamento.
%\noindent

\end{document}
