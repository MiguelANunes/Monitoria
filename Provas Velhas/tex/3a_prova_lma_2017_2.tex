\documentclass[a4paper,11pt]{article}
\usepackage[T1]{fontenc}
\usepackage[utf8]{inputenc} %% isto garante compatibilidade com seu MAC
%\usepackage{lmodern}
\usepackage[brazil]{babel}
\usepackage{comment,color, fancybox} %%% 
\usepackage{graphicx, url}
\usepackage{amsmath}
\usepackage{amsfonts}
\usepackage{amssymb}
%%%\usepackage[normalem]{ulem}

\topmargin       0.1cm
\headheight      0pt
\headsep         -0.7cm
\textheight      25cm
\textwidth       16.7cm
\oddsidemargin   -5mm
\evensidemargin  -5mm
\baselineskip    -13pt

\begin{document}
%\framebox[15cm][c]{$3^a$ Avaliação de Lógica Matemática  (LMA) - Joinville, \today}

\begin{large}
\begin{center}

\shadowbox{
\begin{minipage}[c]{10cm}
\begin{center}
\sf
$3^{\underline{a}}$ Avalia\c c\~ao de L\'ogica Matem\'atica  (LMA)\\
Professores: Rogério  ($T_B$) e  Claudio ($T_A$)\\
Joinville, \today
\end{center}
\end{minipage}
} %% 
\end{center}
\end{large} 
%\author{Rogério Eduardo da Silva e Claudio Cesar de Sá}
%\date{\today}

\vskip 0.2cm Acad\^emico(a) : \rule{10cm}{0.7pt} Turma:  \rule{1cm}{0.7pt}
%%%\noindent Algumas questões desta prova vieram de \url{http://www.cs.utsa.edu/~bylander/cs2233/index.html}

%{\bf Atenção: Exame Final dia 07/12 (4a. feira às 17:00 hrs. -- Sala F101)}
%{\bf Atenção: Exame Final dia 06/07 (4a. feira às 17:00 hrs. -- Sala F101)}

\begin{enumerate}
\setlength{\itemsep}{13pt}

%{\Huge Rogerio pegar exercicios daqui }
%\url{http://cnx.org/contents/46af3ad0-8636-4653-b64a-ea5f934df9d6@28/Exercises_for_First-Order_Logi}

\item {\bf (1.0 pt.)} Determine o valor verdade $\{V, F \}$ (a interpretação $\Phi $)
de cada uma das fórmulas abaixo em seu respectivo domínio. Dados: $A = \{ 3,  5 \}$, $B = \{ -15, 1, 15\}$ e  $C = \{ 6,  7 \}$.
As questões serão \textbf{apenas} validadas mediante os cálculos em separado.
 Em seguida preencha a tabela abaixo:

\begin{center}
\begin{tabular}{l|c|c|c} \hline \hline
 & \multicolumn{2}{c}{Domínios} \\ \hline \hline
 & $x \in A ~e~ y \in C$ & $x \in B ~e~ y \in A$ & $x,z \in A ~e~ y \in C$  \\ \hline

$\exists x \forall y: ~ (x+y \neq 2.y)$ &  &  & ******** \\ \hline
$\forall x \exists y \exists z: ~ (x+y+z \leq 5)$ & ******** & ******** &   \\ \hline \hline
\end{tabular}

\end{center}

%\textcolor{red}{Rogerio ... aqui teria que dar uma pensada melhor ... rever as formula propostas}


\item \label{P1} {\bf (2.5 pts.)} Para a base de conhecimento abaixo, descrita em LPO, prove que {\em jonhsmith} come a {\em fruta}.

\vspace{-0.5cm}

\begin{tabular}{ll}
\\  \hline \hline
  % after \\: \hline or \cline{col1-col2} \cline{col3-col4} ...
  1. & $homem(johnsmith)$ \\
  2. & $mulher(janesmith)$ \\
  3. & $\forall x: homem(x) \vee mulher(x) \rightarrow humano(x)$ \\
  4. & $comida(fruta)$ \\
  5. & $\forall x: humano(x) \rightarrow sente\_fome(x)$ \\
  6. & $\forall x \exists y: humano(x) \wedge sente\_fome(x) \wedge comida(y) \rightarrow come(x,y)$ \\
    \hline \hline
 \end{tabular}
 
 \textcolor{red}{Rogerio ... já corrigi}

\item {\bf (1.5 pts.)} Dada as formulações em LPO do problema anterior, reescreva-as em cláusulas de  PICAT.
  
\item {\bf (1.5 pts.)} Dado o código abaixo, indique precisamente a sua saída, após   a execução do predicado \texttt{main}:
%\textcolor{red}{JA MODIFICADA E IMPLEMENTADA ... NESTE DIRETORIO DE PROVAS}

\begin{scriptsize}
\begin{verbatim}
index(-)      
     f1(3).
     f1(5).
     f1(6).
	
index(-)  
     f2(1).
     f2(2).
	
index(-)     
     f3(3).
     f3(6).
    
regra( X_1, Y_1, Resp ) => f1(X_1), f2(Y_1), Resp = (X_1 * Y_1), f3(Resp).
%%% esta regra tem backtracking
main ?=> regra(X,Y,R), printf("\n X: %d \tY: %d \tResp: %d", X,Y,R), false.
main =>  printf("\n\n FIM DOS FATOS \n\n") , true.
\end{verbatim}
\end{scriptsize}

%\vskip 0.3cm

OBS.: O predicado ``\texttt{false}''  é usado apenas para forçar o PICAT retornar todas as respostas usando o ({\em backtracking}). 

\newpage
\item Implementar em PICAT (usando notação da \textbf{programação em lógica}) as funções abaixo:

\begin{enumerate}

   \item {\bf (1.0 pt.)} Dada a implementação do predicado {\tt main} abaixo, implemente os predicados {\tt lado(X)} (fato) e 
   {\tt area\_quadrado(X,AREA)} (regra) a fim de determinar a área de um quadrado. Dado: 
   
   \begin{footnotesize}
   {\tt main => lado(X), area\_quadrado(X,AREA),\\ printf("A area do quadrado de lado \%d = \%d", X, AREA).}
   \end{footnotesize}
   
   
   \item {\bf (1.25 pts.)} Implemente um predicado que calcule $A \times B	$ sem usar o operador de multiplicação (apenas use somas sucessivas e não multiplicação, e A e B são números naturais, logo a resposta será um natural).
   
   Exemplo: $5 \times 3 = 5 + 5 + 5$

	\item {\bf (1.25 pts.)} Crie um predicado que converta um número decimal em seu correspondente binário de acordo com o procedimento descrito 
	a seguir:
		\begin{enumerate}
		\item Dado o número decimal $N$ a ser convertido divida-o sucessivas vezes por 2 até $N=0$
		\item Apresente todos os restos das divisões na ordem inversa que foram produzidos (ver exemplo para $N=25$ na figura abaixo)
		\item \underline{OBS}.: em PICAT o resto da divisão é dado pelo comando {\tt X mod Y}. 
		
		Exemplo: no {\tt R = 10 mod 3} a variável $R$ recebe o valor 1
		\end{enumerate}
 
	\begin{center}\includegraphics[width=6cm]{dec-bin.png}\end{center}
\end{enumerate}

\end{enumerate}
%\textcolor{red}{Removi as RIs ... etc ... nao sao mais necessarias neste estagio! colocamos no quadro o que precisarem}

\begin{comment}
  \item  Algo como ....:
  $$ 
  \operatorname{fact}(n) =
 \begin{cases}
  1 & \mbox{if } n = 0 \\
  n \cdot \operatorname{fact}(n-1) & \mbox{se } n > 0 \\
 \end{cases}
$$
   \end{comment}


\begin{comment}
\underline{{\large Equivalências Notáveis}}:
{\footnotesize
\begin{description}
\setlength{\itemsep}{-2pt}

\item[Idempotência (ID):] $P\Leftrightarrow P\wedge P$ ou $P\Leftrightarrow P\vee P$
\item[Comutação (COM):] $P\wedge Q\Leftrightarrow Q\wedge P$ ou $P\vee Q\Leftrightarrow Q\vee P$
\item[Associação (ASSOC):] $P\wedge(Q\wedge R)\Leftrightarrow (P\wedge Q)\wedge R$ ou $P\vee(Q\vee R)\Leftrightarrow (P\vee Q)\vee R$ 
\item[Distribuição (DIST):] $P\wedge(Q\vee R)\Leftrightarrow (P\wedge Q)\vee (P \wedge R)$ ou $P\vee(Q\wedge R)\Leftrightarrow (P\vee Q)\wedge (P\vee R)$
\item[Dupla Negação (DN):] $P\Leftrightarrow\sim\sim P$
\item[De Morgan (DM):] $\sim(P \wedge Q) \Leftrightarrow \sim P \vee\sim Q$ ou $\sim(P \vee Q) \Leftrightarrow \sim P \wedge\sim Q$
\item[Equivalência da Condicional (COND):] $P\rightarrow Q \Leftrightarrow\sim P \vee Q$

\item[Bicondicional (BICOND):] $P\leftrightarrow Q \Leftrightarrow (P\rightarrow Q)\wedge(Q\rightarrow P)$

\item[Contraposição (CP):] $P\rightarrow Q \Leftrightarrow \sim Q\rightarrow\sim P$

\item[Exportação-Importação (EI):] $P\wedge Q\rightarrow R \Leftrightarrow P\rightarrow(Q\rightarrow R)$

\item[Contradição:] $P\wedge \sim P \Leftrightarrow \square $

\item[Tautologia:] $ P\vee \sim P \Leftrightarrow \blacksquare    $

\item[Negações para LPO:] $ \sim \forall x: px \Leftrightarrow \exists x: \sim px $
\item[Negações para LPO:] $ \sim \exists x: px \Leftrightarrow \forall x: \sim px $
\end{description}
}

\underline{{\large Regras Inferencias Válidas (Teoremas)}}:
{\footnotesize
\begin{description}
\setlength{\itemsep}{-2pt}
\item[Adição (AD):] $P \vdash P \vee Q$ ou $P \vdash Q \vee P$
\item[Simplificação (SIMP):] $P \wedge Q \vdash P$ ou $P \wedge Q \vdash Q$
\item[Conjunção (CONJ)] $P, Q \vdash P \wedge Q$ ou $P, Q \vdash Q \wedge P$
\item[Absorção (ABS):] $P \rightarrow Q \vdash P \rightarrow (P \wedge Q)$
\item[Modus Ponens (MP):] $P \rightarrow Q, P \vdash Q$
\item[Modus Tollens (MT):] $P \rightarrow Q, \sim Q \vdash \sim P$
\item[Silogismo Disjuntivo (SD):] $P \vee Q, \sim P \vdash Q$ ou $P \vee Q, \sim Q \vdash P$
\item[Silogismo Hipotético (SH):] $P \rightarrow Q, Q\rightarrow R \vdash P\rightarrow R$
\item[Dilema Construtivo (DC):] $P\rightarrow Q, R\rightarrow S, P \vee R \vdash Q\vee S$
\item[Dilema Destrutivo (DD):] $P\rightarrow Q, R\rightarrow S, \sim Q\vee\sim S \vdash \sim P \vee\sim R$
\end{description}
%\end{enumerate}
\end{comment}

\begin{flushleft}
\underline{Observações}:
\begin{enumerate}
\setlength{\itemsep}{-2pt}
\item Qualquer dúvida, desenvolva a questão e deixe tudo explicado, detalhadamente, que avaliaremos o seu conhecimentos sobre  o assunto;

 
 \item Todas as questões devem ter seus passos precisamente identificados para validar a questão
 
  \item \underline{Clareza e legibilidade};
\end{enumerate}
\end{flushleft}

\end{document}
